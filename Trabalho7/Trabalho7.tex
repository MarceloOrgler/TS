\documentclass[11pt]{article}

    \usepackage[breakable]{tcolorbox}
    \usepackage{parskip} % Stop auto-indenting (to mimic markdown behaviour)
    
    \usepackage{iftex}
    \ifPDFTeX
    	\usepackage[T1]{fontenc}
    	\usepackage{mathpazo}
    \else
    	\usepackage{fontspec}
    \fi

    % Basic figure setup, for now with no caption control since it's done
    % automatically by Pandoc (which extracts ![](path) syntax from Markdown).
    \usepackage{graphicx}
    % Maintain compatibility with old templates. Remove in nbconvert 6.0
    \let\Oldincludegraphics\includegraphics
    % Ensure that by default, figures have no caption (until we provide a
    % proper Figure object with a Caption API and a way to capture that
    % in the conversion process - todo).
    \usepackage{caption}
    \DeclareCaptionFormat{nocaption}{}
    \captionsetup{format=nocaption,aboveskip=0pt,belowskip=0pt}

    \usepackage{float}
    \floatplacement{figure}{H} % forces figures to be placed at the correct location
    \usepackage{xcolor} % Allow colors to be defined
    \usepackage{enumerate} % Needed for markdown enumerations to work
    \usepackage{geometry} % Used to adjust the document margins
    \usepackage{amsmath} % Equations
    \usepackage{amssymb} % Equations
    \usepackage{textcomp} % defines textquotesingle
    % Hack from http://tex.stackexchange.com/a/47451/13684:
    \AtBeginDocument{%
        \def\PYZsq{\textquotesingle}% Upright quotes in Pygmentized code
    }
    \usepackage{upquote} % Upright quotes for verbatim code
    \usepackage{eurosym} % defines \euro
    \usepackage[mathletters]{ucs} % Extended unicode (utf-8) support
    \usepackage{fancyvrb} % verbatim replacement that allows latex
    \usepackage{grffile} % extends the file name processing of package graphics 
                         % to support a larger range
    \makeatletter % fix for old versions of grffile with XeLaTeX
    \@ifpackagelater{grffile}{2019/11/01}
    {
      % Do nothing on new versions
    }
    {
      \def\Gread@@xetex#1{%
        \IfFileExists{"\Gin@base".bb}%
        {\Gread@eps{\Gin@base.bb}}%
        {\Gread@@xetex@aux#1}%
      }
    }
    \makeatother
    \usepackage[Export]{adjustbox} % Used to constrain images to a maximum size
    \adjustboxset{max size={0.9\linewidth}{0.9\paperheight}}

    % The hyperref package gives us a pdf with properly built
    % internal navigation ('pdf bookmarks' for the table of contents,
    % internal cross-reference links, web links for URLs, etc.)
    \usepackage{hyperref}
    % The default LaTeX title has an obnoxious amount of whitespace. By default,
    % titling removes some of it. It also provides customization options.
    \usepackage{titling}
    \usepackage{longtable} % longtable support required by pandoc >1.10
    \usepackage{booktabs}  % table support for pandoc > 1.12.2
    \usepackage[inline]{enumitem} % IRkernel/repr support (it uses the enumerate* environment)
    \usepackage[normalem]{ulem} % ulem is needed to support strikethroughs (\sout)
                                % normalem makes italics be italics, not underlines
    \usepackage{mathrsfs}
    

    
    % Colors for the hyperref package
    \definecolor{urlcolor}{rgb}{0,.145,.698}
    \definecolor{linkcolor}{rgb}{.71,0.21,0.01}
    \definecolor{citecolor}{rgb}{.12,.54,.11}

    % ANSI colors
    \definecolor{ansi-black}{HTML}{3E424D}
    \definecolor{ansi-black-intense}{HTML}{282C36}
    \definecolor{ansi-red}{HTML}{E75C58}
    \definecolor{ansi-red-intense}{HTML}{B22B31}
    \definecolor{ansi-green}{HTML}{00A250}
    \definecolor{ansi-green-intense}{HTML}{007427}
    \definecolor{ansi-yellow}{HTML}{DDB62B}
    \definecolor{ansi-yellow-intense}{HTML}{B27D12}
    \definecolor{ansi-blue}{HTML}{208FFB}
    \definecolor{ansi-blue-intense}{HTML}{0065CA}
    \definecolor{ansi-magenta}{HTML}{D160C4}
    \definecolor{ansi-magenta-intense}{HTML}{A03196}
    \definecolor{ansi-cyan}{HTML}{60C6C8}
    \definecolor{ansi-cyan-intense}{HTML}{258F8F}
    \definecolor{ansi-white}{HTML}{C5C1B4}
    \definecolor{ansi-white-intense}{HTML}{A1A6B2}
    \definecolor{ansi-default-inverse-fg}{HTML}{FFFFFF}
    \definecolor{ansi-default-inverse-bg}{HTML}{000000}

    % common color for the border for error outputs.
    \definecolor{outerrorbackground}{HTML}{FFDFDF}

    % commands and environments needed by pandoc snippets
    % extracted from the output of `pandoc -s`
    \providecommand{\tightlist}{%
      \setlength{\itemsep}{0pt}\setlength{\parskip}{0pt}}
    \DefineVerbatimEnvironment{Highlighting}{Verbatim}{commandchars=\\\{\}}
    % Add ',fontsize=\small' for more characters per line
    \newenvironment{Shaded}{}{}
    \newcommand{\KeywordTok}[1]{\textcolor[rgb]{0.00,0.44,0.13}{\textbf{{#1}}}}
    \newcommand{\DataTypeTok}[1]{\textcolor[rgb]{0.56,0.13,0.00}{{#1}}}
    \newcommand{\DecValTok}[1]{\textcolor[rgb]{0.25,0.63,0.44}{{#1}}}
    \newcommand{\BaseNTok}[1]{\textcolor[rgb]{0.25,0.63,0.44}{{#1}}}
    \newcommand{\FloatTok}[1]{\textcolor[rgb]{0.25,0.63,0.44}{{#1}}}
    \newcommand{\CharTok}[1]{\textcolor[rgb]{0.25,0.44,0.63}{{#1}}}
    \newcommand{\StringTok}[1]{\textcolor[rgb]{0.25,0.44,0.63}{{#1}}}
    \newcommand{\CommentTok}[1]{\textcolor[rgb]{0.38,0.63,0.69}{\textit{{#1}}}}
    \newcommand{\OtherTok}[1]{\textcolor[rgb]{0.00,0.44,0.13}{{#1}}}
    \newcommand{\AlertTok}[1]{\textcolor[rgb]{1.00,0.00,0.00}{\textbf{{#1}}}}
    \newcommand{\FunctionTok}[1]{\textcolor[rgb]{0.02,0.16,0.49}{{#1}}}
    \newcommand{\RegionMarkerTok}[1]{{#1}}
    \newcommand{\ErrorTok}[1]{\textcolor[rgb]{1.00,0.00,0.00}{\textbf{{#1}}}}
    \newcommand{\NormalTok}[1]{{#1}}
    
    % Additional commands for more recent versions of Pandoc
    \newcommand{\ConstantTok}[1]{\textcolor[rgb]{0.53,0.00,0.00}{{#1}}}
    \newcommand{\SpecialCharTok}[1]{\textcolor[rgb]{0.25,0.44,0.63}{{#1}}}
    \newcommand{\VerbatimStringTok}[1]{\textcolor[rgb]{0.25,0.44,0.63}{{#1}}}
    \newcommand{\SpecialStringTok}[1]{\textcolor[rgb]{0.73,0.40,0.53}{{#1}}}
    \newcommand{\ImportTok}[1]{{#1}}
    \newcommand{\DocumentationTok}[1]{\textcolor[rgb]{0.73,0.13,0.13}{\textit{{#1}}}}
    \newcommand{\AnnotationTok}[1]{\textcolor[rgb]{0.38,0.63,0.69}{\textbf{\textit{{#1}}}}}
    \newcommand{\CommentVarTok}[1]{\textcolor[rgb]{0.38,0.63,0.69}{\textbf{\textit{{#1}}}}}
    \newcommand{\VariableTok}[1]{\textcolor[rgb]{0.10,0.09,0.49}{{#1}}}
    \newcommand{\ControlFlowTok}[1]{\textcolor[rgb]{0.00,0.44,0.13}{\textbf{{#1}}}}
    \newcommand{\OperatorTok}[1]{\textcolor[rgb]{0.40,0.40,0.40}{{#1}}}
    \newcommand{\BuiltInTok}[1]{{#1}}
    \newcommand{\ExtensionTok}[1]{{#1}}
    \newcommand{\PreprocessorTok}[1]{\textcolor[rgb]{0.74,0.48,0.00}{{#1}}}
    \newcommand{\AttributeTok}[1]{\textcolor[rgb]{0.49,0.56,0.16}{{#1}}}
    \newcommand{\InformationTok}[1]{\textcolor[rgb]{0.38,0.63,0.69}{\textbf{\textit{{#1}}}}}
    \newcommand{\WarningTok}[1]{\textcolor[rgb]{0.38,0.63,0.69}{\textbf{\textit{{#1}}}}}
    
    
    % Define a nice break command that doesn't care if a line doesn't already
    % exist.
    \def\br{\hspace*{\fill} \\* }
    % Math Jax compatibility definitions
    \def\gt{>}
    \def\lt{<}
    \let\Oldtex\TeX
    \let\Oldlatex\LaTeX
    \renewcommand{\TeX}{\textrm{\Oldtex}}
    \renewcommand{\LaTeX}{\textrm{\Oldlatex}}
    % Document parameters
    % Document title
    \title{Trabalho7}
    
    
    
    
    
% Pygments definitions
\makeatletter
\def\PY@reset{\let\PY@it=\relax \let\PY@bf=\relax%
    \let\PY@ul=\relax \let\PY@tc=\relax%
    \let\PY@bc=\relax \let\PY@ff=\relax}
\def\PY@tok#1{\csname PY@tok@#1\endcsname}
\def\PY@toks#1+{\ifx\relax#1\empty\else%
    \PY@tok{#1}\expandafter\PY@toks\fi}
\def\PY@do#1{\PY@bc{\PY@tc{\PY@ul{%
    \PY@it{\PY@bf{\PY@ff{#1}}}}}}}
\def\PY#1#2{\PY@reset\PY@toks#1+\relax+\PY@do{#2}}

\@namedef{PY@tok@w}{\def\PY@tc##1{\textcolor[rgb]{0.73,0.73,0.73}{##1}}}
\@namedef{PY@tok@c}{\let\PY@it=\textit\def\PY@tc##1{\textcolor[rgb]{0.24,0.48,0.48}{##1}}}
\@namedef{PY@tok@cp}{\def\PY@tc##1{\textcolor[rgb]{0.61,0.40,0.00}{##1}}}
\@namedef{PY@tok@k}{\let\PY@bf=\textbf\def\PY@tc##1{\textcolor[rgb]{0.00,0.50,0.00}{##1}}}
\@namedef{PY@tok@kp}{\def\PY@tc##1{\textcolor[rgb]{0.00,0.50,0.00}{##1}}}
\@namedef{PY@tok@kt}{\def\PY@tc##1{\textcolor[rgb]{0.69,0.00,0.25}{##1}}}
\@namedef{PY@tok@o}{\def\PY@tc##1{\textcolor[rgb]{0.40,0.40,0.40}{##1}}}
\@namedef{PY@tok@ow}{\let\PY@bf=\textbf\def\PY@tc##1{\textcolor[rgb]{0.67,0.13,1.00}{##1}}}
\@namedef{PY@tok@nb}{\def\PY@tc##1{\textcolor[rgb]{0.00,0.50,0.00}{##1}}}
\@namedef{PY@tok@nf}{\def\PY@tc##1{\textcolor[rgb]{0.00,0.00,1.00}{##1}}}
\@namedef{PY@tok@nc}{\let\PY@bf=\textbf\def\PY@tc##1{\textcolor[rgb]{0.00,0.00,1.00}{##1}}}
\@namedef{PY@tok@nn}{\let\PY@bf=\textbf\def\PY@tc##1{\textcolor[rgb]{0.00,0.00,1.00}{##1}}}
\@namedef{PY@tok@ne}{\let\PY@bf=\textbf\def\PY@tc##1{\textcolor[rgb]{0.80,0.25,0.22}{##1}}}
\@namedef{PY@tok@nv}{\def\PY@tc##1{\textcolor[rgb]{0.10,0.09,0.49}{##1}}}
\@namedef{PY@tok@no}{\def\PY@tc##1{\textcolor[rgb]{0.53,0.00,0.00}{##1}}}
\@namedef{PY@tok@nl}{\def\PY@tc##1{\textcolor[rgb]{0.46,0.46,0.00}{##1}}}
\@namedef{PY@tok@ni}{\let\PY@bf=\textbf\def\PY@tc##1{\textcolor[rgb]{0.44,0.44,0.44}{##1}}}
\@namedef{PY@tok@na}{\def\PY@tc##1{\textcolor[rgb]{0.41,0.47,0.13}{##1}}}
\@namedef{PY@tok@nt}{\let\PY@bf=\textbf\def\PY@tc##1{\textcolor[rgb]{0.00,0.50,0.00}{##1}}}
\@namedef{PY@tok@nd}{\def\PY@tc##1{\textcolor[rgb]{0.67,0.13,1.00}{##1}}}
\@namedef{PY@tok@s}{\def\PY@tc##1{\textcolor[rgb]{0.73,0.13,0.13}{##1}}}
\@namedef{PY@tok@sd}{\let\PY@it=\textit\def\PY@tc##1{\textcolor[rgb]{0.73,0.13,0.13}{##1}}}
\@namedef{PY@tok@si}{\let\PY@bf=\textbf\def\PY@tc##1{\textcolor[rgb]{0.64,0.35,0.47}{##1}}}
\@namedef{PY@tok@se}{\let\PY@bf=\textbf\def\PY@tc##1{\textcolor[rgb]{0.67,0.36,0.12}{##1}}}
\@namedef{PY@tok@sr}{\def\PY@tc##1{\textcolor[rgb]{0.64,0.35,0.47}{##1}}}
\@namedef{PY@tok@ss}{\def\PY@tc##1{\textcolor[rgb]{0.10,0.09,0.49}{##1}}}
\@namedef{PY@tok@sx}{\def\PY@tc##1{\textcolor[rgb]{0.00,0.50,0.00}{##1}}}
\@namedef{PY@tok@m}{\def\PY@tc##1{\textcolor[rgb]{0.40,0.40,0.40}{##1}}}
\@namedef{PY@tok@gh}{\let\PY@bf=\textbf\def\PY@tc##1{\textcolor[rgb]{0.00,0.00,0.50}{##1}}}
\@namedef{PY@tok@gu}{\let\PY@bf=\textbf\def\PY@tc##1{\textcolor[rgb]{0.50,0.00,0.50}{##1}}}
\@namedef{PY@tok@gd}{\def\PY@tc##1{\textcolor[rgb]{0.63,0.00,0.00}{##1}}}
\@namedef{PY@tok@gi}{\def\PY@tc##1{\textcolor[rgb]{0.00,0.52,0.00}{##1}}}
\@namedef{PY@tok@gr}{\def\PY@tc##1{\textcolor[rgb]{0.89,0.00,0.00}{##1}}}
\@namedef{PY@tok@ge}{\let\PY@it=\textit}
\@namedef{PY@tok@gs}{\let\PY@bf=\textbf}
\@namedef{PY@tok@gp}{\let\PY@bf=\textbf\def\PY@tc##1{\textcolor[rgb]{0.00,0.00,0.50}{##1}}}
\@namedef{PY@tok@go}{\def\PY@tc##1{\textcolor[rgb]{0.44,0.44,0.44}{##1}}}
\@namedef{PY@tok@gt}{\def\PY@tc##1{\textcolor[rgb]{0.00,0.27,0.87}{##1}}}
\@namedef{PY@tok@err}{\def\PY@bc##1{{\setlength{\fboxsep}{\string -\fboxrule}\fcolorbox[rgb]{1.00,0.00,0.00}{1,1,1}{\strut ##1}}}}
\@namedef{PY@tok@kc}{\let\PY@bf=\textbf\def\PY@tc##1{\textcolor[rgb]{0.00,0.50,0.00}{##1}}}
\@namedef{PY@tok@kd}{\let\PY@bf=\textbf\def\PY@tc##1{\textcolor[rgb]{0.00,0.50,0.00}{##1}}}
\@namedef{PY@tok@kn}{\let\PY@bf=\textbf\def\PY@tc##1{\textcolor[rgb]{0.00,0.50,0.00}{##1}}}
\@namedef{PY@tok@kr}{\let\PY@bf=\textbf\def\PY@tc##1{\textcolor[rgb]{0.00,0.50,0.00}{##1}}}
\@namedef{PY@tok@bp}{\def\PY@tc##1{\textcolor[rgb]{0.00,0.50,0.00}{##1}}}
\@namedef{PY@tok@fm}{\def\PY@tc##1{\textcolor[rgb]{0.00,0.00,1.00}{##1}}}
\@namedef{PY@tok@vc}{\def\PY@tc##1{\textcolor[rgb]{0.10,0.09,0.49}{##1}}}
\@namedef{PY@tok@vg}{\def\PY@tc##1{\textcolor[rgb]{0.10,0.09,0.49}{##1}}}
\@namedef{PY@tok@vi}{\def\PY@tc##1{\textcolor[rgb]{0.10,0.09,0.49}{##1}}}
\@namedef{PY@tok@vm}{\def\PY@tc##1{\textcolor[rgb]{0.10,0.09,0.49}{##1}}}
\@namedef{PY@tok@sa}{\def\PY@tc##1{\textcolor[rgb]{0.73,0.13,0.13}{##1}}}
\@namedef{PY@tok@sb}{\def\PY@tc##1{\textcolor[rgb]{0.73,0.13,0.13}{##1}}}
\@namedef{PY@tok@sc}{\def\PY@tc##1{\textcolor[rgb]{0.73,0.13,0.13}{##1}}}
\@namedef{PY@tok@dl}{\def\PY@tc##1{\textcolor[rgb]{0.73,0.13,0.13}{##1}}}
\@namedef{PY@tok@s2}{\def\PY@tc##1{\textcolor[rgb]{0.73,0.13,0.13}{##1}}}
\@namedef{PY@tok@sh}{\def\PY@tc##1{\textcolor[rgb]{0.73,0.13,0.13}{##1}}}
\@namedef{PY@tok@s1}{\def\PY@tc##1{\textcolor[rgb]{0.73,0.13,0.13}{##1}}}
\@namedef{PY@tok@mb}{\def\PY@tc##1{\textcolor[rgb]{0.40,0.40,0.40}{##1}}}
\@namedef{PY@tok@mf}{\def\PY@tc##1{\textcolor[rgb]{0.40,0.40,0.40}{##1}}}
\@namedef{PY@tok@mh}{\def\PY@tc##1{\textcolor[rgb]{0.40,0.40,0.40}{##1}}}
\@namedef{PY@tok@mi}{\def\PY@tc##1{\textcolor[rgb]{0.40,0.40,0.40}{##1}}}
\@namedef{PY@tok@il}{\def\PY@tc##1{\textcolor[rgb]{0.40,0.40,0.40}{##1}}}
\@namedef{PY@tok@mo}{\def\PY@tc##1{\textcolor[rgb]{0.40,0.40,0.40}{##1}}}
\@namedef{PY@tok@ch}{\let\PY@it=\textit\def\PY@tc##1{\textcolor[rgb]{0.24,0.48,0.48}{##1}}}
\@namedef{PY@tok@cm}{\let\PY@it=\textit\def\PY@tc##1{\textcolor[rgb]{0.24,0.48,0.48}{##1}}}
\@namedef{PY@tok@cpf}{\let\PY@it=\textit\def\PY@tc##1{\textcolor[rgb]{0.24,0.48,0.48}{##1}}}
\@namedef{PY@tok@c1}{\let\PY@it=\textit\def\PY@tc##1{\textcolor[rgb]{0.24,0.48,0.48}{##1}}}
\@namedef{PY@tok@cs}{\let\PY@it=\textit\def\PY@tc##1{\textcolor[rgb]{0.24,0.48,0.48}{##1}}}

\def\PYZbs{\char`\\}
\def\PYZus{\char`\_}
\def\PYZob{\char`\{}
\def\PYZcb{\char`\}}
\def\PYZca{\char`\^}
\def\PYZam{\char`\&}
\def\PYZlt{\char`\<}
\def\PYZgt{\char`\>}
\def\PYZsh{\char`\#}
\def\PYZpc{\char`\%}
\def\PYZdl{\char`\$}
\def\PYZhy{\char`\-}
\def\PYZsq{\char`\'}
\def\PYZdq{\char`\"}
\def\PYZti{\char`\~}
% for compatibility with earlier versions
\def\PYZat{@}
\def\PYZlb{[}
\def\PYZrb{]}
\makeatother


    % For linebreaks inside Verbatim environment from package fancyvrb. 
    \makeatletter
        \newbox\Wrappedcontinuationbox 
        \newbox\Wrappedvisiblespacebox 
        \newcommand*\Wrappedvisiblespace {\textcolor{red}{\textvisiblespace}} 
        \newcommand*\Wrappedcontinuationsymbol {\textcolor{red}{\llap{\tiny$\m@th\hookrightarrow$}}} 
        \newcommand*\Wrappedcontinuationindent {3ex } 
        \newcommand*\Wrappedafterbreak {\kern\Wrappedcontinuationindent\copy\Wrappedcontinuationbox} 
        % Take advantage of the already applied Pygments mark-up to insert 
        % potential linebreaks for TeX processing. 
        %        {, <, #, %, $, ' and ": go to next line. 
        %        _, }, ^, &, >, - and ~: stay at end of broken line. 
        % Use of \textquotesingle for straight quote. 
        \newcommand*\Wrappedbreaksatspecials {% 
            \def\PYGZus{\discretionary{\char`\_}{\Wrappedafterbreak}{\char`\_}}% 
            \def\PYGZob{\discretionary{}{\Wrappedafterbreak\char`\{}{\char`\{}}% 
            \def\PYGZcb{\discretionary{\char`\}}{\Wrappedafterbreak}{\char`\}}}% 
            \def\PYGZca{\discretionary{\char`\^}{\Wrappedafterbreak}{\char`\^}}% 
            \def\PYGZam{\discretionary{\char`\&}{\Wrappedafterbreak}{\char`\&}}% 
            \def\PYGZlt{\discretionary{}{\Wrappedafterbreak\char`\<}{\char`\<}}% 
            \def\PYGZgt{\discretionary{\char`\>}{\Wrappedafterbreak}{\char`\>}}% 
            \def\PYGZsh{\discretionary{}{\Wrappedafterbreak\char`\#}{\char`\#}}% 
            \def\PYGZpc{\discretionary{}{\Wrappedafterbreak\char`\%}{\char`\%}}% 
            \def\PYGZdl{\discretionary{}{\Wrappedafterbreak\char`\$}{\char`\$}}% 
            \def\PYGZhy{\discretionary{\char`\-}{\Wrappedafterbreak}{\char`\-}}% 
            \def\PYGZsq{\discretionary{}{\Wrappedafterbreak\textquotesingle}{\textquotesingle}}% 
            \def\PYGZdq{\discretionary{}{\Wrappedafterbreak\char`\"}{\char`\"}}% 
            \def\PYGZti{\discretionary{\char`\~}{\Wrappedafterbreak}{\char`\~}}% 
        } 
        % Some characters . , ; ? ! / are not pygmentized. 
        % This macro makes them "active" and they will insert potential linebreaks 
        \newcommand*\Wrappedbreaksatpunct {% 
            \lccode`\~`\.\lowercase{\def~}{\discretionary{\hbox{\char`\.}}{\Wrappedafterbreak}{\hbox{\char`\.}}}% 
            \lccode`\~`\,\lowercase{\def~}{\discretionary{\hbox{\char`\,}}{\Wrappedafterbreak}{\hbox{\char`\,}}}% 
            \lccode`\~`\;\lowercase{\def~}{\discretionary{\hbox{\char`\;}}{\Wrappedafterbreak}{\hbox{\char`\;}}}% 
            \lccode`\~`\:\lowercase{\def~}{\discretionary{\hbox{\char`\:}}{\Wrappedafterbreak}{\hbox{\char`\:}}}% 
            \lccode`\~`\?\lowercase{\def~}{\discretionary{\hbox{\char`\?}}{\Wrappedafterbreak}{\hbox{\char`\?}}}% 
            \lccode`\~`\!\lowercase{\def~}{\discretionary{\hbox{\char`\!}}{\Wrappedafterbreak}{\hbox{\char`\!}}}% 
            \lccode`\~`\/\lowercase{\def~}{\discretionary{\hbox{\char`\/}}{\Wrappedafterbreak}{\hbox{\char`\/}}}% 
            \catcode`\.\active
            \catcode`\,\active 
            \catcode`\;\active
            \catcode`\:\active
            \catcode`\?\active
            \catcode`\!\active
            \catcode`\/\active 
            \lccode`\~`\~ 	
        }
    \makeatother

    \let\OriginalVerbatim=\Verbatim
    \makeatletter
    \renewcommand{\Verbatim}[1][1]{%
        %\parskip\z@skip
        \sbox\Wrappedcontinuationbox {\Wrappedcontinuationsymbol}%
        \sbox\Wrappedvisiblespacebox {\FV@SetupFont\Wrappedvisiblespace}%
        \def\FancyVerbFormatLine ##1{\hsize\linewidth
            \vtop{\raggedright\hyphenpenalty\z@\exhyphenpenalty\z@
                \doublehyphendemerits\z@\finalhyphendemerits\z@
                \strut ##1\strut}%
        }%
        % If the linebreak is at a space, the latter will be displayed as visible
        % space at end of first line, and a continuation symbol starts next line.
        % Stretch/shrink are however usually zero for typewriter font.
        \def\FV@Space {%
            \nobreak\hskip\z@ plus\fontdimen3\font minus\fontdimen4\font
            \discretionary{\copy\Wrappedvisiblespacebox}{\Wrappedafterbreak}
            {\kern\fontdimen2\font}%
        }%
        
        % Allow breaks at special characters using \PYG... macros.
        \Wrappedbreaksatspecials
        % Breaks at punctuation characters . , ; ? ! and / need catcode=\active 	
        \OriginalVerbatim[#1,codes*=\Wrappedbreaksatpunct]%
    }
    \makeatother

    % Exact colors from NB
    \definecolor{incolor}{HTML}{303F9F}
    \definecolor{outcolor}{HTML}{D84315}
    \definecolor{cellborder}{HTML}{CFCFCF}
    \definecolor{cellbackground}{HTML}{F7F7F7}
    
    % prompt
    \makeatletter
    \newcommand{\boxspacing}{\kern\kvtcb@left@rule\kern\kvtcb@boxsep}
    \makeatother
    \newcommand{\prompt}[4]{
        {\ttfamily\llap{{\color{#2}[#3]:\hspace{3pt}#4}}\vspace{-\baselineskip}}
    }
    

    
    % Prevent overflowing lines due to hard-to-break entities
    \sloppy 
    % Setup hyperref package
    \hypersetup{
      breaklinks=true,  % so long urls are correctly broken across lines
      colorlinks=true,
      urlcolor=urlcolor,
      linkcolor=linkcolor,
      citecolor=citecolor,
      }
    % Slightly bigger margins than the latex defaults
    
    \geometry{verbose,tmargin=1in,bmargin=1in,lmargin=1in,rmargin=1in}
    
    

\begin{document}
    
    \maketitle
    
    

    
    \begin{tcolorbox}[breakable, size=fbox, boxrule=1pt, pad at break*=1mm,colback=cellbackground, colframe=cellborder]
\prompt{In}{incolor}{5}{\boxspacing}
\begin{Verbatim}[commandchars=\\\{\}]
\PY{n+nf}{library}\PY{p}{(}\PY{n}{pacman}\PY{p}{)}
\PY{n+nf}{p\PYZus{}load}\PY{p}{(}\PY{n}{dplyr}\PY{p}{,}\PY{n}{tidyr}\PY{p}{,}\PY{n}{splines}\PY{p}{,}\PY{n}{glarma}\PY{p}{)}
\end{Verbatim}
\end{tcolorbox}

    \begin{tcolorbox}[breakable, size=fbox, boxrule=1pt, pad at break*=1mm,colback=cellbackground, colframe=cellborder]
\prompt{In}{incolor}{6}{\boxspacing}
\begin{Verbatim}[commandchars=\\\{\}]
\PY{n+nf}{setwd}\PY{p}{(}\PY{l+s}{\PYZsq{}}\PY{l+s}{C:\PYZbs{}\PYZbs{}Users\PYZbs{}\PYZbs{}Marcelo\PYZbs{}\PYZbs{}OneDrive\PYZbs{}\PYZbs{}Área de Trabalho\PYZbs{}\PYZbs{}ts\PYZbs{}\PYZbs{}trabalho 2 \PYZhy{} TS\PYZbs{}\PYZbs{}Trabalho7\PYZsq{}}\PY{p}{)}
\PY{n+nf}{getwd}\PY{p}{(}\PY{p}{)}
\end{Verbatim}
\end{tcolorbox}

    'C:/Users/Marcelo/OneDrive/Área de Trabalho/ts/trabalho 2 - TS/Trabalho7'

    
    Função de Verossimilhança da Professora está no script7.R para não
poluir o notebook.

    \begin{tcolorbox}[breakable, size=fbox, boxrule=1pt, pad at break*=1mm,colback=cellbackground, colframe=cellborder]
\prompt{In}{incolor}{7}{\boxspacing}
\begin{Verbatim}[commandchars=\\\{\}]
\PY{n+nf}{source}\PY{p}{(}\PY{l+s}{\PYZsq{}}\PY{l+s}{script7.R\PYZsq{}}\PY{p}{)}
\end{Verbatim}
\end{tcolorbox}

    Função que adiciona faz a estimação, calcula o AIC dado o numero de
bases para cada covariável não linear:

    \begin{tcolorbox}[breakable, size=fbox, boxrule=1pt, pad at break*=1mm,colback=cellbackground, colframe=cellborder]
\prompt{In}{incolor}{8}{\boxspacing}
\begin{Verbatim}[commandchars=\\\{\}]
\PY{n}{MLE\PYZus{}com\PYZus{}splines} \PY{o}{\PYZlt{}\PYZhy{}} \PY{n+nf}{function}\PY{p}{(}\PY{n}{y}\PY{p}{,} \PY{n}{z\PYZus{}list}\PY{p}{,} \PY{n}{k\PYZus{}list}\PY{p}{,}
                          \PY{n}{other\PYZus{}covariates} \PY{o}{=} \PY{k+kc}{NULL}\PY{p}{,}
                          \PY{n}{np} \PY{o}{=} \PY{l+m}{0}\PY{p}{,} \PY{n}{nq} \PY{o}{=} \PY{l+m}{0}\PY{p}{,} \PY{n}{lamb} \PY{o}{=} \PY{l+m}{0.5}\PY{p}{,}
                          \PY{n}{Ind} \PY{o}{=} \PY{l+m}{0}\PY{p}{)} \PY{p}{\PYZob{}}
  
  \PY{n}{nz} \PY{o}{\PYZlt{}\PYZhy{}} \PY{n+nf}{length}\PY{p}{(}\PY{n}{z\PYZus{}list}\PY{p}{)}
  
  \PY{c+c1}{\PYZsh{} Construindo as bases splines com seus respectivos ks.}
  \PY{n}{spline\PYZus{}list} \PY{o}{\PYZlt{}\PYZhy{}} \PY{n+nf}{list}\PY{p}{(}\PY{p}{)}
  \PY{n}{spline\PYZus{}sizes} \PY{o}{\PYZlt{}\PYZhy{}} \PY{n+nf}{integer}\PY{p}{(}\PY{n}{nz}\PY{p}{)}
  
  \PY{n+nf}{for }\PY{p}{(}\PY{n}{i} \PY{n}{in} \PY{n+nf}{seq\PYZus{}along}\PY{p}{(}\PY{n}{z\PYZus{}list}\PY{p}{)}\PY{p}{)} \PY{p}{\PYZob{}}
    \PY{n}{k\PYZus{}i} \PY{o}{\PYZlt{}\PYZhy{}} \PY{n}{k\PYZus{}list}\PY{p}{[[}\PY{n}{i}\PY{p}{]]}
    \PY{n}{spline\PYZus{}i} \PY{o}{\PYZlt{}\PYZhy{}} \PY{n+nf}{ns}\PY{p}{(}\PY{n}{z\PYZus{}list}\PY{p}{[[}\PY{n}{i}\PY{p}{]]}\PY{p}{,} \PY{n}{df} \PY{o}{=} \PY{n}{k\PYZus{}i}\PY{p}{)} \PY{c+c1}{\PYZsh{}bases splines para a variável não linear numero i}
    \PY{n}{spline\PYZus{}list}\PY{p}{[[}\PY{n}{i}\PY{p}{]]} \PY{o}{\PYZlt{}\PYZhy{}} \PY{n}{spline\PYZus{}i} \PY{c+c1}{\PYZsh{}lista com as bases}
    \PY{n}{spline\PYZus{}sizes}\PY{p}{[}\PY{n}{i}\PY{p}{]} \PY{o}{\PYZlt{}\PYZhy{}} \PY{n+nf}{ncol}\PY{p}{(}\PY{n}{spline\PYZus{}i}\PY{p}{)} \PY{c+c1}{\PYZsh{}numero de bases a variavel não linear i}
  \PY{p}{\PYZcb{}}

  \PY{c+c1}{\PYZsh{} juntando as bases splines usando cbind:}
  \PY{n+nf}{names}\PY{p}{(}\PY{n}{spline\PYZus{}list}\PY{p}{)} \PY{o}{\PYZlt{}\PYZhy{}} \PY{n+nf}{names}\PY{p}{(}\PY{n}{z\PYZus{}list}\PY{p}{)}
  \PY{n}{spline\PYZus{}basis} \PY{o}{\PYZlt{}\PYZhy{}} \PY{n+nf}{do.call}\PY{p}{(}\PY{n}{cbind}\PY{p}{,} \PY{n}{spline\PYZus{}list}\PY{p}{)}
  \PY{n}{n\PYZus{}spline} \PY{o}{\PYZlt{}\PYZhy{}} \PY{n+nf}{sum}\PY{p}{(}\PY{n}{spline\PYZus{}sizes}\PY{p}{)}
  
  \PY{c+c1}{\PYZsh{} Juntando com as outras covariáveis:}
  \PY{n+nf}{if }\PY{p}{(}\PY{o}{!}\PY{n+nf}{is.null}\PY{p}{(}\PY{n}{other\PYZus{}covariates}\PY{p}{)}\PY{p}{)} \PY{p}{\PYZob{}}
    \PY{n}{x} \PY{o}{\PYZlt{}\PYZhy{}} \PY{n+nf}{cbind}\PY{p}{(}\PY{n}{spline\PYZus{}basis}\PY{p}{,} \PY{n}{other\PYZus{}covariates}\PY{p}{)}
    \PY{n}{n\PYZus{}other} \PY{o}{\PYZlt{}\PYZhy{}} \PY{n+nf}{ncol}\PY{p}{(}\PY{n}{other\PYZus{}covariates}\PY{p}{)}
  \PY{p}{\PYZcb{}} \PY{n}{else} \PY{p}{\PYZob{}}
    \PY{n}{x} \PY{o}{\PYZlt{}\PYZhy{}} \PY{n}{spline\PYZus{}basis}
    \PY{n}{n\PYZus{}other} \PY{o}{\PYZlt{}\PYZhy{}} \PY{l+m}{0}
  \PY{p}{\PYZcb{}}
  
  \PY{n}{nb} \PY{o}{\PYZlt{}\PYZhy{}} \PY{l+m}{1} \PY{o}{+} \PY{n}{n\PYZus{}spline} \PY{o}{+} \PY{n}{n\PYZus{}other}  \PY{c+c1}{\PYZsh{} intercepto + splines + numero betas das outras cov}
  \PY{n}{n\PYZus{}params} \PY{o}{\PYZlt{}\PYZhy{}} \PY{n}{nb} \PY{o}{+} \PY{n}{np} \PY{o}{+} \PY{n}{nq}
  \PY{n}{start\PYZus{}vals} \PY{o}{\PYZlt{}\PYZhy{}} \PY{n+nf}{rep}\PY{p}{(}\PY{l+m}{0}\PY{p}{,} \PY{n}{n\PYZus{}params}\PY{p}{)}

  \PY{c+c1}{\PYZsh{} ajustando o modelo dado um numero fixo de bases para cada covariável não linear:}
  \PY{n}{opt} \PY{o}{\PYZlt{}\PYZhy{}} \PY{n+nf}{optim}\PY{p}{(}\PY{n}{par} \PY{o}{=} \PY{n}{start\PYZus{}vals}\PY{p}{,}
               \PY{n}{fn} \PY{o}{=} \PY{n}{Like}\PY{p}{,}
               \PY{n}{y} \PY{o}{=} \PY{n}{y}\PY{p}{,}
               \PY{n}{nb} \PY{o}{=} \PY{n}{nb}\PY{p}{,}
               \PY{n}{np} \PY{o}{=} \PY{n}{np}\PY{p}{,}
               \PY{n}{nq} \PY{o}{=} \PY{n}{nq}\PY{p}{,}
               \PY{n}{lamb} \PY{o}{=} \PY{n}{lamb}\PY{p}{,}
               \PY{n}{x} \PY{o}{=} \PY{n}{x}\PY{p}{,}
               \PY{n}{Ind} \PY{o}{=} \PY{n}{Ind}\PY{p}{,}
               \PY{n}{method} \PY{o}{=} \PY{l+s}{\PYZdq{}}\PY{l+s}{BFGS\PYZdq{}}\PY{p}{,}
               \PY{n}{hessian} \PY{o}{=} \PY{k+kc}{FALSE}\PY{p}{)}

  \PY{n}{logLik} \PY{o}{\PYZlt{}\PYZhy{}} \PY{o}{\PYZhy{}}\PY{n}{opt}\PY{o}{\PYZdl{}}\PY{n}{value}
  \PY{n}{AIC} \PY{o}{\PYZlt{}\PYZhy{}} \PY{l+m}{2} \PY{o}{*} \PY{n}{n\PYZus{}params} \PY{o}{\PYZhy{}} \PY{l+m}{2} \PY{o}{*} \PY{n}{logLik}

  \PY{n}{best\PYZus{}coef} \PY{o}{\PYZlt{}\PYZhy{}} \PY{n}{opt}\PY{o}{\PYZdl{}}\PY{n}{par}
  \PY{n}{beta0} \PY{o}{\PYZlt{}\PYZhy{}} \PY{n}{best\PYZus{}coef}\PY{p}{[}\PY{l+m}{1}\PY{p}{]}
  \PY{n}{beta\PYZus{}spline\PYZus{}all} \PY{o}{\PYZlt{}\PYZhy{}} \PY{n}{best\PYZus{}coef}\PY{p}{[}\PY{l+m}{2}\PY{o}{:}\PY{p}{(}\PY{l+m}{1} \PY{o}{+} \PY{n}{n\PYZus{}spline}\PY{p}{)}\PY{p}{]}
  
  \PY{c+c1}{\PYZsh{} separando os coeficientes das bases splines para cada variável não linear:}
  \PY{n}{beta\PYZus{}spline\PYZus{}list} \PY{o}{\PYZlt{}\PYZhy{}} \PY{n+nf}{list}\PY{p}{(}\PY{p}{)}
  \PY{n}{idx} \PY{o}{\PYZlt{}\PYZhy{}} \PY{l+m}{1}
  \PY{n+nf}{for }\PY{p}{(}\PY{n}{i} \PY{n}{in} \PY{n+nf}{seq\PYZus{}along}\PY{p}{(}\PY{n}{z\PYZus{}list}\PY{p}{)}\PY{p}{)} \PY{p}{\PYZob{}}
    \PY{n}{varname} \PY{o}{\PYZlt{}\PYZhy{}} \PY{n+nf}{names}\PY{p}{(}\PY{n}{z\PYZus{}list}\PY{p}{)}\PY{p}{[}\PY{n}{i}\PY{p}{]}
    \PY{n}{len} \PY{o}{\PYZlt{}\PYZhy{}} \PY{n}{spline\PYZus{}sizes}\PY{p}{[}\PY{n}{i}\PY{p}{]}
    \PY{n}{beta\PYZus{}spline\PYZus{}list}\PY{p}{[[}\PY{n}{varname}\PY{p}{]]} \PY{o}{\PYZlt{}\PYZhy{}} \PY{n}{beta\PYZus{}spline\PYZus{}all}\PY{p}{[}\PY{n}{idx}\PY{o}{:}\PY{p}{(}\PY{n}{idx} \PY{o}{+} \PY{n}{len} \PY{o}{\PYZhy{}} \PY{l+m}{1}\PY{p}{)}\PY{p}{]}
    \PY{n}{idx} \PY{o}{\PYZlt{}\PYZhy{}} \PY{n}{idx} \PY{o}{+} \PY{n}{len}
  \PY{p}{\PYZcb{}}
    
  \PY{c+c1}{\PYZsh{} separando os outros coeficientes:}
  \PY{n}{beta\PYZus{}other} \PY{o}{\PYZlt{}\PYZhy{}} \PY{n+nf}{if }\PY{p}{(}\PY{n}{n\PYZus{}other} \PY{o}{\PYZgt{}} \PY{l+m}{0}\PY{p}{)}
      \PY{p}{\PYZob{}}\PY{n}{best\PYZus{}coef}\PY{p}{[}\PY{p}{(}\PY{l+m}{2} \PY{o}{+} \PY{n}{n\PYZus{}spline}\PY{p}{)}\PY{o}{:}\PY{p}{(}\PY{l+m}{1} \PY{o}{+} \PY{n}{n\PYZus{}spline} \PY{o}{+} \PY{n}{n\PYZus{}other}\PY{p}{)}\PY{p}{]}\PY{p}{\PYZcb{}}
                \PY{n}{else} \PY{p}{\PYZob{}}\PY{n+nf}{numeric}\PY{p}{(}\PY{l+m}{0}\PY{p}{)}\PY{p}{\PYZcb{}}
    
  \PY{n}{fi}   \PY{o}{\PYZlt{}\PYZhy{}} \PY{n+nf}{if }\PY{p}{(}\PY{n}{np} \PY{o}{\PYZgt{}} \PY{l+m}{0}\PY{p}{)} 
      \PY{p}{\PYZob{}}\PY{n}{best\PYZus{}coef}\PY{p}{[}\PY{p}{(}\PY{n}{nb} \PY{o}{+} \PY{l+m}{1}\PY{p}{)}\PY{o}{:}\PY{p}{(}\PY{n}{nb} \PY{o}{+} \PY{n}{np}\PY{p}{)}\PY{p}{]}\PY{p}{\PYZcb{}}
                \PY{n}{else} \PY{p}{\PYZob{}}\PY{n+nf}{numeric}\PY{p}{(}\PY{l+m}{0}\PY{p}{)}\PY{p}{\PYZcb{}}
  
  \PY{n}{theta} \PY{o}{\PYZlt{}\PYZhy{}} \PY{n+nf}{if }\PY{p}{(}\PY{n}{nq} \PY{o}{\PYZgt{}} \PY{l+m}{0}\PY{p}{)} 
      \PY{p}{\PYZob{}}\PY{n}{best\PYZus{}coef}\PY{p}{[}\PY{p}{(}\PY{n}{nb} \PY{o}{+} \PY{n}{np} \PY{o}{+} \PY{l+m}{1}\PY{p}{)}\PY{o}{:}\PY{p}{(}\PY{n}{nb} \PY{o}{+} \PY{n}{np} \PY{o}{+} \PY{n}{nq}\PY{p}{)}\PY{p}{]}\PY{p}{\PYZcb{}} 
       \PY{n}{else} \PY{p}{\PYZob{}}\PY{n+nf}{numeric}\PY{p}{(}\PY{l+m}{0}\PY{p}{)}\PY{p}{\PYZcb{}}

  \PY{n}{fit} \PY{o}{\PYZlt{}\PYZhy{}} \PY{n+nf}{list}\PY{p}{(}
    \PY{n}{AIC} \PY{o}{=} \PY{n}{AIC}\PY{p}{,}
    \PY{n}{logLik} \PY{o}{=} \PY{n}{logLik}\PY{p}{,}
    \PY{n}{beta0} \PY{o}{=} \PY{n}{beta0}\PY{p}{,}
    \PY{n}{beta\PYZus{}spline} \PY{o}{=} \PY{n}{beta\PYZus{}spline\PYZus{}list}\PY{p}{,}
    \PY{n}{beta\PYZus{}other} \PY{o}{=} \PY{n}{beta\PYZus{}other}\PY{p}{,}
    \PY{n}{fi} \PY{o}{=} \PY{n}{fi}\PY{p}{,}
    \PY{n}{theta} \PY{o}{=} \PY{n}{theta}\PY{p}{,}
    \PY{n}{k\PYZus{}list} \PY{o}{=} \PY{n}{k\PYZus{}list}
  \PY{p}{)}

  \PY{n+nf}{return}\PY{p}{(}\PY{n}{fit}\PY{p}{)}
\PY{p}{\PYZcb{}}
\end{Verbatim}
\end{tcolorbox}

    \begin{tcolorbox}[breakable, size=fbox, boxrule=1pt, pad at break*=1mm,colback=cellbackground, colframe=cellborder]
\prompt{In}{incolor}{9}{\boxspacing}
\begin{Verbatim}[commandchars=\\\{\}]
\PY{n}{select\PYZus{}best\PYZus{}k\PYZus{}multi} \PY{o}{\PYZlt{}\PYZhy{}} \PY{n+nf}{function}\PY{p}{(}\PY{n}{y}\PY{p}{,} \PY{n}{z\PYZus{}list}\PY{p}{,} \PY{n}{k\PYZus{}ranges}\PY{p}{,}
                                \PY{n}{other\PYZus{}covariates} \PY{o}{=} \PY{k+kc}{NULL}\PY{p}{,}
                                \PY{n}{np} \PY{o}{=} \PY{l+m}{0}\PY{p}{,} \PY{n}{nq} \PY{o}{=} \PY{l+m}{0}\PY{p}{,} \PY{n}{lamb} \PY{o}{=} \PY{l+m}{0.5}\PY{p}{,}
                                \PY{n}{Ind} \PY{o}{=} \PY{l+m}{0}\PY{p}{)} \PY{p}{\PYZob{}}
  
  \PY{c+c1}{\PYZsh{} Gerando um grid com os possíveis numeros de bases para cada variável spline:}
    
  \PY{n}{k\PYZus{}grid} \PY{o}{\PYZlt{}\PYZhy{}} \PY{n+nf}{expand.grid}\PY{p}{(}\PY{n}{k\PYZus{}ranges}\PY{p}{)}
  
  \PY{n}{best\PYZus{}AIC} \PY{o}{\PYZlt{}\PYZhy{}} \PY{k+kc}{Inf}
  \PY{n}{best\PYZus{}fit} \PY{o}{\PYZlt{}\PYZhy{}} \PY{k+kc}{NULL}
  \PY{n}{best\PYZus{}k\PYZus{}list} \PY{o}{\PYZlt{}\PYZhy{}} \PY{k+kc}{NULL}
  
 \PY{c+c1}{\PYZsh{}calculando o AIC para cada combinação de número de bases:}
  \PY{n+nf}{for }\PY{p}{(}\PY{n}{i} \PY{n}{in} \PY{n+nf}{seq\PYZus{}len}\PY{p}{(}\PY{n+nf}{nrow}\PY{p}{(}\PY{n}{k\PYZus{}grid}\PY{p}{)}\PY{p}{)}\PY{p}{)} 
  \PY{p}{\PYZob{}}
    \PY{n}{current\PYZus{}k} \PY{o}{\PYZlt{}\PYZhy{}} \PY{n+nf}{as.list}\PY{p}{(}\PY{n}{k\PYZus{}grid}\PY{p}{[}\PY{n}{i}\PY{p}{,} \PY{p}{]}\PY{p}{)}
    
    \PY{n}{result} \PY{o}{\PYZlt{}\PYZhy{}} \PY{n+nf}{tryCatch}\PY{p}{(}\PY{p}{\PYZob{}}
                          \PY{n+nf}{MLE\PYZus{}com\PYZus{}splines}\PY{p}{(}
                            \PY{n}{y} \PY{o}{=} \PY{n}{y}\PY{p}{,}
                            \PY{n}{z\PYZus{}list} \PY{o}{=} \PY{n}{z\PYZus{}list}\PY{p}{,}
                            \PY{n}{k\PYZus{}list} \PY{o}{=} \PY{n}{current\PYZus{}k}\PY{p}{,}
                            \PY{n}{other\PYZus{}covariates} \PY{o}{=} \PY{n}{other\PYZus{}covariates}\PY{p}{,}
                            \PY{n}{np} \PY{o}{=} \PY{n}{np}\PY{p}{,}
                            \PY{n}{nq} \PY{o}{=} \PY{n}{nq}\PY{p}{,}
                            \PY{n}{lamb} \PY{o}{=} \PY{n}{lamb}\PY{p}{,}
                            \PY{n}{Ind} \PY{o}{=} \PY{n}{Ind}
                          \PY{p}{)}
                        \PY{p}{\PYZcb{}}\PY{p}{,} 
                       \PY{n}{error} \PY{o}{=} \PY{n+nf}{function}\PY{p}{(}\PY{n}{e}\PY{p}{)} \PY{p}{\PYZob{}}\PY{n+nf}{return}\PY{p}{(}\PY{k+kc}{NULL}\PY{p}{)}\PY{p}{\PYZcb{}}
                      \PY{p}{)}
    
    \PY{n+nf}{if }\PY{p}{(}\PY{o}{!}\PY{n+nf}{is.null}\PY{p}{(}\PY{n}{result}\PY{p}{)} \PY{o}{\PYZam{}\PYZam{}} \PY{n}{result}\PY{o}{\PYZdl{}}\PY{n}{AIC} \PY{o}{\PYZlt{}} \PY{n}{best\PYZus{}AIC}\PY{p}{)} \PY{p}{\PYZob{}}
        
      \PY{n}{best\PYZus{}AIC} \PY{o}{\PYZlt{}\PYZhy{}} \PY{n}{result}\PY{o}{\PYZdl{}}\PY{n}{AIC}
      \PY{n}{best\PYZus{}fit} \PY{o}{\PYZlt{}\PYZhy{}} \PY{n}{result}
      \PY{n}{best\PYZus{}k\PYZus{}list} \PY{o}{\PYZlt{}\PYZhy{}} \PY{n}{current\PYZus{}k}
    \PY{p}{\PYZcb{}}
  \PY{p}{\PYZcb{}}
    
  \PY{n}{best\PYZus{}fit}\PY{o}{\PYZdl{}}\PY{n}{k\PYZus{}list} \PY{o}{\PYZlt{}\PYZhy{}} \PY{n}{best\PYZus{}k\PYZus{}list}
  \PY{n+nf}{return}\PY{p}{(}\PY{n}{best\PYZus{}fit}\PY{p}{)}
\PY{p}{\PYZcb{}}
\end{Verbatim}
\end{tcolorbox}

    Pegando o dataset de Doença respiratórias:

    \begin{tcolorbox}[breakable, size=fbox, boxrule=1pt, pad at break*=1mm,colback=cellbackground, colframe=cellborder]
\prompt{In}{incolor}{10}{\boxspacing}
\begin{Verbatim}[commandchars=\\\{\}]
\PY{n}{df} \PY{o}{\PYZlt{}\PYZhy{}} \PY{n+nf}{read.csv}\PY{p}{(}\PY{l+s}{\PYZdq{}}\PY{l+s}{DR.csv\PYZdq{}}\PY{p}{,}\PY{n}{header} \PY{o}{=} \PY{k+kc}{TRUE}\PY{p}{,}\PY{n}{sep}\PY{o}{=}\PY{l+s}{\PYZsq{}}\PY{l+s}{;\PYZsq{}}\PY{p}{)}
\end{Verbatim}
\end{tcolorbox}

    \begin{tcolorbox}[breakable, size=fbox, boxrule=1pt, pad at break*=1mm,colback=cellbackground, colframe=cellborder]
\prompt{In}{incolor}{11}{\boxspacing}
\begin{Verbatim}[commandchars=\\\{\}]
\PY{n+nf}{summary}\PY{p}{(}\PY{n}{df}\PY{p}{)}
\end{Verbatim}
\end{tcolorbox}

    
    \begin{Verbatim}[commandchars=\\\{\}]
     Data            Atendimentos    tempmed               O3           
 Length:72          Min.   : 1.0   Length:72          Length:72         
 Class :character   1st Qu.:12.0   Class :character   Class :character  
 Mode  :character   Median :21.5   Mode  :character   Mode  :character  
                    Mean   :24.1                                        
                    3rd Qu.:33.0                                        
                    Max.   :91.0                                        
      CO                NO2                SO2           
 Length:72          Length:72          Length:72         
 Class :character   Class :character   Class :character  
 Mode  :character   Mode  :character   Mode  :character  
                                                         
                                                         
                                                         
    \end{Verbatim}

    
    \begin{tcolorbox}[breakable, size=fbox, boxrule=1pt, pad at break*=1mm,colback=cellbackground, colframe=cellborder]
\prompt{In}{incolor}{12}{\boxspacing}
\begin{Verbatim}[commandchars=\\\{\}]
\PY{n}{df}\PY{o}{\PYZlt{}\PYZhy{}} \PY{n}{df} \PY{o}{\PYZpc{}\PYZgt{}\PYZpc{}} \PY{n+nf}{select}\PY{p}{(}\PY{n}{Atendimentos}\PY{p}{,}\PY{n}{tempmed}\PY{p}{,}\PY{n}{O3}\PY{p}{,}\PY{n}{CO}\PY{p}{,}\PY{n}{NO2}\PY{p}{,}\PY{n}{SO2}\PY{p}{)}
\end{Verbatim}
\end{tcolorbox}

    Limpando os dados:

    \begin{tcolorbox}[breakable, size=fbox, boxrule=1pt, pad at break*=1mm,colback=cellbackground, colframe=cellborder]
\prompt{In}{incolor}{13}{\boxspacing}
\begin{Verbatim}[commandchars=\\\{\}]
\PY{n}{df}\PY{p}{[}\PY{p}{]} \PY{o}{\PYZlt{}\PYZhy{}} \PY{n+nf}{lapply}\PY{p}{(}\PY{n}{df}\PY{p}{,} \PY{n+nf}{function}\PY{p}{(}\PY{n}{x}\PY{p}{)} \PY{p}{\PYZob{}}
  \PY{n+nf}{if }\PY{p}{(}\PY{n+nf}{is.character}\PY{p}{(}\PY{n}{x}\PY{p}{)}\PY{p}{)} \PY{n+nf}{as.numeric}\PY{p}{(}\PY{n+nf}{gsub}\PY{p}{(}\PY{l+s}{\PYZdq{}}\PY{l+s}{,\PYZdq{}}\PY{p}{,} \PY{l+s}{\PYZdq{}}\PY{l+s}{.\PYZdq{}}\PY{p}{,} \PY{n}{x}\PY{p}{)}\PY{p}{)} \PY{n}{else} \PY{n}{x}
\PY{p}{\PYZcb{}}\PY{p}{)}
\end{Verbatim}
\end{tcolorbox}

    \begin{tcolorbox}[breakable, size=fbox, boxrule=1pt, pad at break*=1mm,colback=cellbackground, colframe=cellborder]
\prompt{In}{incolor}{14}{\boxspacing}
\begin{Verbatim}[commandchars=\\\{\}]
\PY{n+nf}{summary}\PY{p}{(}\PY{n}{df}\PY{p}{)}
\end{Verbatim}
\end{tcolorbox}

    
    \begin{Verbatim}[commandchars=\\\{\}]
  Atendimentos     tempmed            O3              CO        
 Min.   : 1.0   Min.   :18.85   Min.   :16.76   Min.   : 295.2  
 1st Qu.:12.0   1st Qu.:22.61   1st Qu.:26.34   1st Qu.: 645.7  
 Median :21.5   Median :23.97   Median :32.22   Median : 829.8  
 Mean   :24.1   Mean   :24.17   Mean   :32.72   Mean   : 826.8  
 3rd Qu.:33.0   3rd Qu.:26.02   3rd Qu.:38.44   3rd Qu.:1018.4  
 Max.   :91.0   Max.   :29.14   Max.   :48.15   Max.   :1616.3  
      NO2             SO2        
 Min.   :11.19   Min.   : 5.959  
 1st Qu.:19.33   1st Qu.: 9.361  
 Median :24.45   Median :11.140  
 Mean   :24.42   Mean   :11.809  
 3rd Qu.:28.70   3rd Qu.:13.695  
 Max.   :38.72   Max.   :20.420  
    \end{Verbatim}

    
    \begin{tcolorbox}[breakable, size=fbox, boxrule=1pt, pad at break*=1mm,colback=cellbackground, colframe=cellborder]
\prompt{In}{incolor}{15}{\boxspacing}
\begin{Verbatim}[commandchars=\\\{\}]
\PY{n}{df}\PY{o}{\PYZlt{}\PYZhy{}} \PY{n}{df} \PY{o}{\PYZpc{}\PYZgt{}\PYZpc{}} \PY{n+nf}{drop\PYZus{}na}\PY{p}{(}\PY{p}{)}
\end{Verbatim}
\end{tcolorbox}

    Montando o X:

    \begin{tcolorbox}[breakable, size=fbox, boxrule=1pt, pad at break*=1mm,colback=cellbackground, colframe=cellborder]
\prompt{In}{incolor}{16}{\boxspacing}
\begin{Verbatim}[commandchars=\\\{\}]
\PY{n}{X} \PY{o}{\PYZlt{}\PYZhy{}} \PY{n+nf}{as.matrix}\PY{p}{(}\PY{n}{df}\PY{p}{[} \PY{p}{,} \PY{n+nf}{setdiff}\PY{p}{(}\PY{n+nf}{names}\PY{p}{(}\PY{n}{df}\PY{p}{)}\PY{p}{,} \PY{l+s}{\PYZdq{}}\PY{l+s}{Atendimentos\PYZdq{}}\PY{p}{)}\PY{p}{]}\PY{p}{)}
\end{Verbatim}
\end{tcolorbox}

    Ajustando o modelo sem os splines:

    \begin{tcolorbox}[breakable, size=fbox, boxrule=1pt, pad at break*=1mm,colback=cellbackground, colframe=cellborder]
\prompt{In}{incolor}{17}{\boxspacing}
\begin{Verbatim}[commandchars=\\\{\}]
\PY{n}{fit} \PY{o}{=}\PY{n+nf}{glarma}\PY{p}{(}\PY{n}{df}\PY{o}{\PYZdl{}}\PY{n}{Atendimentos}\PY{p}{,} \PY{n}{X}\PY{o}{=}\PY{n}{X} \PY{p}{,} \PY{n}{phiLags} \PY{o}{=} \PY{n+nf}{c}\PY{p}{(}\PY{l+m}{1}\PY{p}{)} \PY{p}{,}\PY{n}{thetaLags} \PY{o}{=} \PY{k+kc}{NULL}\PY{p}{,} \PY{n}{type} \PY{o}{=} \PY{l+s}{\PYZdq{}}\PY{l+s}{Poi\PYZdq{}}\PY{p}{)}
\PY{n+nf}{summary}\PY{p}{(}\PY{n}{fit}\PY{p}{)}
\end{Verbatim}
\end{tcolorbox}

    
    \begin{Verbatim}[commandchars=\\\{\}]

Call: glarma(y = df\$Atendimentos, X = X, type = "Poi", phiLags = c(1), 
    thetaLags = NULL)

Pearson Residuals:
    Min       1Q   Median       3Q      Max  
-5.1470  -1.5949  -0.2822   1.8875  11.4303  

GLARMA Coefficients:
      Estimate Std.Error z-ratio Pr(>|z|)    
phi\_1 0.064316  0.007945   8.095 6.66e-16 ***

Linear Model Coefficients:
          Estimate  Std.Error z-ratio Pr(>|z|)    
tempmed  0.0626788  0.0080746   7.763 8.44e-15 ***
O3       0.0417608  0.0038306  10.902  < 2e-16 ***
CO       0.0018250  0.0001737  10.508  < 2e-16 ***
NO2     -0.0078196  0.0055289  -1.414    0.157    
SO2     -0.0994414  0.0113291  -8.778  < 2e-16 ***

    Null deviance: 795.53  on 71  degrees of freedom
Residual deviance: 612.48  on 66  degrees of freedom
AIC: 943.4291 

Number of Fisher Scoring iterations: 30

LRT and Wald Test:
Alternative hypothesis: model is a GLARMA process
Null hypothesis: model is a GLM with the same regression structure
          Statistic  p-value    
LR Test       75.71  < 2e-16 ***
Wald Test     65.53 5.55e-16 ***
---
Signif. codes:  0 '***' 0.001 '**' 0.01 '*' 0.05 '.' 0.1 ' ' 1

    \end{Verbatim}

    
    \begin{tcolorbox}[breakable, size=fbox, boxrule=1pt, pad at break*=1mm,colback=cellbackground, colframe=cellborder]
\prompt{In}{incolor}{18}{\boxspacing}
\begin{Verbatim}[commandchars=\\\{\}]
\PY{n}{fit} \PY{o}{=}\PY{n+nf}{glarma}\PY{p}{(}\PY{n}{df}\PY{o}{\PYZdl{}}\PY{n}{Atendimentos}\PY{p}{,} \PY{n}{X}\PY{o}{=}\PY{n}{X}\PY{p}{[}\PY{p}{,}\PY{n+nf}{c}\PY{p}{(}\PY{l+s}{\PYZsq{}}\PY{l+s}{tempmed\PYZsq{}}\PY{p}{,}\PY{l+s}{\PYZsq{}}\PY{l+s}{O3\PYZsq{}}\PY{p}{,}\PY{l+s}{\PYZsq{}}\PY{l+s}{CO\PYZsq{}}\PY{p}{,}\PY{l+s}{\PYZsq{}}\PY{l+s}{SO2\PYZsq{}}\PY{p}{)}\PY{p}{]} \PY{p}{,} \PY{n}{phiLags} \PY{o}{=} \PY{n+nf}{c}\PY{p}{(}\PY{l+m}{1}\PY{p}{,}\PY{l+m}{2}\PY{p}{)} \PY{p}{,}\PY{n}{thetaLags} \PY{o}{=} \PY{k+kc}{NULL}\PY{p}{,} \PY{n}{type} \PY{o}{=} \PY{l+s}{\PYZdq{}}\PY{l+s}{Poi\PYZdq{}}\PY{p}{)}
\PY{n+nf}{summary}\PY{p}{(}\PY{n}{fit}\PY{p}{)}
\end{Verbatim}
\end{tcolorbox}

    
    \begin{Verbatim}[commandchars=\\\{\}]

Call: glarma(y = df\$Atendimentos, X = X[, c("tempmed", "O3", "CO", 
    "SO2")], type = "Poi", phiLags = c(1, 2), thetaLags = NULL)

Pearson Residuals:
    Min       1Q   Median       3Q      Max  
-5.2529  -1.7941  -0.1555   1.4626   8.5222  

GLARMA Coefficients:
      Estimate Std.Error z-ratio Pr(>|z|)    
phi\_1 0.082932  0.008198  10.116  < 2e-16 ***
phi\_2 0.046149  0.008895   5.188 2.13e-07 ***

Linear Model Coefficients:
          Estimate  Std.Error z-ratio Pr(>|z|)    
tempmed  0.0639349  0.0083392   7.667 1.75e-14 ***
O3       0.0364563  0.0032981  11.054  < 2e-16 ***
CO       0.0013487  0.0001142  11.814  < 2e-16 ***
SO2     -0.0720963  0.0118371  -6.091 1.12e-09 ***

    Null deviance: 795.53  on 71  degrees of freedom
Residual deviance: 578.17  on 66  degrees of freedom
AIC: 917.5981 

Number of Fisher Scoring iterations: 30

LRT and Wald Test:
Alternative hypothesis: model is a GLARMA process
Null hypothesis: model is a GLM with the same regression structure
          Statistic p-value    
LR Test       106.0  <2e-16 ***
Wald Test     112.5  <2e-16 ***
---
Signif. codes:  0 '***' 0.001 '**' 0.01 '*' 0.05 '.' 0.1 ' ' 1

    \end{Verbatim}

    
    A partir do AR(2) os coeficientes do phi não são significativos em todos
os níveis de confiança.

    \begin{tcolorbox}[breakable, size=fbox, boxrule=1pt, pad at break*=1mm,colback=cellbackground, colframe=cellborder]
\prompt{In}{incolor}{19}{\boxspacing}
\begin{Verbatim}[commandchars=\\\{\}]
\PY{n+nf}{plot}\PY{p}{(}\PY{n}{X}\PY{p}{[}\PY{p}{,} \PY{l+s}{\PYZdq{}}\PY{l+s}{O3\PYZdq{}}\PY{p}{]}\PY{p}{,} \PY{n}{df}\PY{o}{\PYZdl{}}\PY{n}{Atendimento}\PY{p}{,} \PY{n}{main} \PY{o}{=} \PY{l+s}{\PYZdq{}}\PY{l+s}{Atendimento X O3\PYZdq{}}\PY{p}{)}
\end{Verbatim}
\end{tcolorbox}

    \begin{center}
    \adjustimage{max size={0.9\linewidth}{0.9\paperheight}}{Trabalho7_files/Trabalho7_21_0.png}
    \end{center}
    { \hspace*{\fill} \\}
    
    \begin{tcolorbox}[breakable, size=fbox, boxrule=1pt, pad at break*=1mm,colback=cellbackground, colframe=cellborder]
\prompt{In}{incolor}{20}{\boxspacing}
\begin{Verbatim}[commandchars=\\\{\}]
\PY{n+nf}{plot}\PY{p}{(}\PY{n}{X}\PY{p}{[}\PY{p}{,} \PY{l+s}{\PYZdq{}}\PY{l+s}{tempmed\PYZdq{}}\PY{p}{]}\PY{p}{,} \PY{n}{df}\PY{o}{\PYZdl{}}\PY{n}{Atendimento}\PY{p}{,} \PY{n}{main} \PY{o}{=} \PY{l+s}{\PYZdq{}}\PY{l+s}{Atendimento X tempmed\PYZdq{}}\PY{p}{)}
\end{Verbatim}
\end{tcolorbox}

    \begin{center}
    \adjustimage{max size={0.9\linewidth}{0.9\paperheight}}{Trabalho7_files/Trabalho7_22_0.png}
    \end{center}
    { \hspace*{\fill} \\}
    
    \begin{tcolorbox}[breakable, size=fbox, boxrule=1pt, pad at break*=1mm,colback=cellbackground, colframe=cellborder]
\prompt{In}{incolor}{21}{\boxspacing}
\begin{Verbatim}[commandchars=\\\{\}]
\PY{n+nf}{plot}\PY{p}{(}\PY{n}{X}\PY{p}{[}\PY{p}{,} \PY{l+s}{\PYZdq{}}\PY{l+s}{CO\PYZdq{}}\PY{p}{]}\PY{p}{,} \PY{n}{df}\PY{o}{\PYZdl{}}\PY{n}{Atendimento}\PY{p}{,} \PY{n}{main} \PY{o}{=} \PY{l+s}{\PYZdq{}}\PY{l+s}{Atendimento X CO\PYZdq{}}\PY{p}{)}
\end{Verbatim}
\end{tcolorbox}

    \begin{center}
    \adjustimage{max size={0.9\linewidth}{0.9\paperheight}}{Trabalho7_files/Trabalho7_23_0.png}
    \end{center}
    { \hspace*{\fill} \\}
    
    \begin{tcolorbox}[breakable, size=fbox, boxrule=1pt, pad at break*=1mm,colback=cellbackground, colframe=cellborder]
\prompt{In}{incolor}{22}{\boxspacing}
\begin{Verbatim}[commandchars=\\\{\}]
\PY{n+nf}{plot}\PY{p}{(}\PY{n}{X}\PY{p}{[}\PY{p}{,} \PY{l+s}{\PYZdq{}}\PY{l+s}{SO2\PYZdq{}}\PY{p}{]}\PY{p}{,} \PY{n}{df}\PY{o}{\PYZdl{}}\PY{n}{Atendimento}\PY{p}{,} \PY{n}{main} \PY{o}{=} \PY{l+s}{\PYZdq{}}\PY{l+s}{Atendimento X SO2\PYZdq{}}\PY{p}{)}
\end{Verbatim}
\end{tcolorbox}

    \begin{center}
    \adjustimage{max size={0.9\linewidth}{0.9\paperheight}}{Trabalho7_files/Trabalho7_24_0.png}
    \end{center}
    { \hspace*{\fill} \\}
    
    O3 aparentemente tem relação não linear com o número de atendimentos

    Rodando nossa regressão com os splines:

    \begin{tcolorbox}[breakable, size=fbox, boxrule=1pt, pad at break*=1mm,colback=cellbackground, colframe=cellborder]
\prompt{In}{incolor}{23}{\boxspacing}
\begin{Verbatim}[commandchars=\\\{\}]
\PY{n}{other\PYZus{}covs} \PY{o}{\PYZlt{}\PYZhy{}} \PY{n}{X}\PY{p}{[}\PY{p}{,}\PY{n+nf}{c}\PY{p}{(}\PY{l+s}{\PYZsq{}}\PY{l+s}{tempmed\PYZsq{}}\PY{p}{,}\PY{l+s}{\PYZsq{}}\PY{l+s}{CO\PYZsq{}}\PY{p}{,}\PY{l+s}{\PYZsq{}}\PY{l+s}{SO2\PYZsq{}}\PY{p}{)}\PY{p}{]}
\end{Verbatim}
\end{tcolorbox}

    \begin{tcolorbox}[breakable, size=fbox, boxrule=1pt, pad at break*=1mm,colback=cellbackground, colframe=cellborder]
\prompt{In}{incolor}{24}{\boxspacing}
\begin{Verbatim}[commandchars=\\\{\}]
\PY{n}{resposta}\PY{o}{\PYZlt{}\PYZhy{}}\PY{n+nf}{select\PYZus{}best\PYZus{}k\PYZus{}multi}\PY{p}{(}\PY{n}{y}\PY{o}{=}\PY{n}{df}\PY{o}{\PYZdl{}}\PY{n}{Atendimentos}\PY{p}{,}\PY{n}{z}\PY{o}{=}\PY{n+nf}{list}\PY{p}{(}\PY{n}{O3}\PY{o}{=}\PY{n}{df}\PY{o}{\PYZdl{}}\PY{n}{O3}\PY{p}{,}\PY{n}{tempmed}\PY{o}{=}\PY{n}{df}\PY{o}{\PYZdl{}}\PY{n}{tempmed}\PY{p}{)}\PY{p}{,}\PY{n}{other\PYZus{}covariates} \PY{o}{=} \PY{n}{other\PYZus{}covs}\PY{p}{,}
                        \PY{n}{k\PYZus{}ranges}\PY{o}{=}\PY{n+nf}{list}\PY{p}{(}\PY{n}{O3}\PY{o}{=}\PY{l+m}{1}\PY{o}{:}\PY{l+m}{8}\PY{p}{,}\PY{n}{temp\PYZus{}med}\PY{o}{=}\PY{l+m}{1}\PY{o}{:}\PY{l+m}{8}\PY{p}{)}\PY{p}{,}\PY{n}{np}\PY{o}{=}\PY{l+m}{3}\PY{p}{,}\PY{n}{nq}\PY{o}{=}\PY{l+m}{0}\PY{p}{,}\PY{n}{lamb} \PY{o}{=}\PY{l+m}{0.5}\PY{p}{,}\PY{n}{Ind}\PY{o}{=}\PY{l+m}{0}\PY{p}{)}
\end{Verbatim}
\end{tcolorbox}

    \begin{tcolorbox}[breakable, size=fbox, boxrule=1pt, pad at break*=1mm,colback=cellbackground, colframe=cellborder]
\prompt{In}{incolor}{25}{\boxspacing}
\begin{Verbatim}[commandchars=\\\{\}]
\PY{n}{resposta}
\end{Verbatim}
\end{tcolorbox}

    \begin{description}
\item[\$AIC] -7840.30738737551
\item[\$logLik] 3942.15369368775
\item[\$beta0] 2.76306610689278
\item[\$beta\_spline] \begin{description}
\item[\$O3] \begin{enumerate*}
\item 0.351727607570428
\item 0.748339770830553
\item 0.0514625420616618
\item 1.07794204788271
\item -0.521010265442694
\item 0.572686123411616
\item 0.636101002121954
\end{enumerate*}

\item[\$tempmed] \begin{enumerate*}
\item -0.10411575379499
\item -0.342784661481924
\item -0.0206866784576695
\item -0.127113092415123
\item -0.622252363904484
\item 0.0563834096290639
\item -1.64067137163042
\item -1.7303685099724
\end{enumerate*}

\end{description}

\item[\$beta\_other] \begin{enumerate*}
\item 0.0284734533890816
\item 0.000740302083696139
\item -0.0866522606064693
\end{enumerate*}

\item[\$fi] \begin{enumerate*}
\item 0.0433467203032375
\item 0.00227101251663183
\item -0.0660772354515024
\end{enumerate*}

\item[\$theta] 
\item[\$k\_list] \begin{description}
\item[\$O3] 7
\item[\$temp\_med] 8
\end{description}

\end{description}


    

    % Add a bibliography block to the postdoc
    
    
    
\end{document}
