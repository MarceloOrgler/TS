\documentclass[11pt]{article}

    \usepackage[breakable]{tcolorbox}
    \usepackage{parskip} % Stop auto-indenting (to mimic markdown behaviour)
    
    \usepackage{iftex}
    \ifPDFTeX
    	\usepackage[T1]{fontenc}
    	\usepackage{mathpazo}
    \else
    	\usepackage{fontspec}
    \fi

    % Basic figure setup, for now with no caption control since it's done
    % automatically by Pandoc (which extracts ![](path) syntax from Markdown).
    \usepackage{graphicx}
    % Maintain compatibility with old templates. Remove in nbconvert 6.0
    \let\Oldincludegraphics\includegraphics
    % Ensure that by default, figures have no caption (until we provide a
    % proper Figure object with a Caption API and a way to capture that
    % in the conversion process - todo).
    \usepackage{caption}
    \DeclareCaptionFormat{nocaption}{}
    \captionsetup{format=nocaption,aboveskip=0pt,belowskip=0pt}

    \usepackage{float}
    \floatplacement{figure}{H} % forces figures to be placed at the correct location
    \usepackage{xcolor} % Allow colors to be defined
    \usepackage{enumerate} % Needed for markdown enumerations to work
    \usepackage{geometry} % Used to adjust the document margins
    \usepackage{amsmath} % Equations
    \usepackage{amssymb} % Equations
    \usepackage{textcomp} % defines textquotesingle
    % Hack from http://tex.stackexchange.com/a/47451/13684:
    \AtBeginDocument{%
        \def\PYZsq{\textquotesingle}% Upright quotes in Pygmentized code
    }
    \usepackage{upquote} % Upright quotes for verbatim code
    \usepackage{eurosym} % defines \euro
    \usepackage[mathletters]{ucs} % Extended unicode (utf-8) support
    \usepackage{fancyvrb} % verbatim replacement that allows latex
    \usepackage{grffile} % extends the file name processing of package graphics 
                         % to support a larger range
    \makeatletter % fix for old versions of grffile with XeLaTeX
    \@ifpackagelater{grffile}{2019/11/01}
    {
      % Do nothing on new versions
    }
    {
      \def\Gread@@xetex#1{%
        \IfFileExists{"\Gin@base".bb}%
        {\Gread@eps{\Gin@base.bb}}%
        {\Gread@@xetex@aux#1}%
      }
    }
    \makeatother
    \usepackage[Export]{adjustbox} % Used to constrain images to a maximum size
    \adjustboxset{max size={0.9\linewidth}{0.9\paperheight}}

    % The hyperref package gives us a pdf with properly built
    % internal navigation ('pdf bookmarks' for the table of contents,
    % internal cross-reference links, web links for URLs, etc.)
    \usepackage{hyperref}
    % The default LaTeX title has an obnoxious amount of whitespace. By default,
    % titling removes some of it. It also provides customization options.
    \usepackage{titling}
    \usepackage{longtable} % longtable support required by pandoc >1.10
    \usepackage{booktabs}  % table support for pandoc > 1.12.2
    \usepackage[inline]{enumitem} % IRkernel/repr support (it uses the enumerate* environment)
    \usepackage[normalem]{ulem} % ulem is needed to support strikethroughs (\sout)
                                % normalem makes italics be italics, not underlines
    \usepackage{mathrsfs}
    

    
    % Colors for the hyperref package
    \definecolor{urlcolor}{rgb}{0,.145,.698}
    \definecolor{linkcolor}{rgb}{.71,0.21,0.01}
    \definecolor{citecolor}{rgb}{.12,.54,.11}

    % ANSI colors
    \definecolor{ansi-black}{HTML}{3E424D}
    \definecolor{ansi-black-intense}{HTML}{282C36}
    \definecolor{ansi-red}{HTML}{E75C58}
    \definecolor{ansi-red-intense}{HTML}{B22B31}
    \definecolor{ansi-green}{HTML}{00A250}
    \definecolor{ansi-green-intense}{HTML}{007427}
    \definecolor{ansi-yellow}{HTML}{DDB62B}
    \definecolor{ansi-yellow-intense}{HTML}{B27D12}
    \definecolor{ansi-blue}{HTML}{208FFB}
    \definecolor{ansi-blue-intense}{HTML}{0065CA}
    \definecolor{ansi-magenta}{HTML}{D160C4}
    \definecolor{ansi-magenta-intense}{HTML}{A03196}
    \definecolor{ansi-cyan}{HTML}{60C6C8}
    \definecolor{ansi-cyan-intense}{HTML}{258F8F}
    \definecolor{ansi-white}{HTML}{C5C1B4}
    \definecolor{ansi-white-intense}{HTML}{A1A6B2}
    \definecolor{ansi-default-inverse-fg}{HTML}{FFFFFF}
    \definecolor{ansi-default-inverse-bg}{HTML}{000000}

    % common color for the border for error outputs.
    \definecolor{outerrorbackground}{HTML}{FFDFDF}

    % commands and environments needed by pandoc snippets
    % extracted from the output of `pandoc -s`
    \providecommand{\tightlist}{%
      \setlength{\itemsep}{0pt}\setlength{\parskip}{0pt}}
    \DefineVerbatimEnvironment{Highlighting}{Verbatim}{commandchars=\\\{\}}
    % Add ',fontsize=\small' for more characters per line
    \newenvironment{Shaded}{}{}
    \newcommand{\KeywordTok}[1]{\textcolor[rgb]{0.00,0.44,0.13}{\textbf{{#1}}}}
    \newcommand{\DataTypeTok}[1]{\textcolor[rgb]{0.56,0.13,0.00}{{#1}}}
    \newcommand{\DecValTok}[1]{\textcolor[rgb]{0.25,0.63,0.44}{{#1}}}
    \newcommand{\BaseNTok}[1]{\textcolor[rgb]{0.25,0.63,0.44}{{#1}}}
    \newcommand{\FloatTok}[1]{\textcolor[rgb]{0.25,0.63,0.44}{{#1}}}
    \newcommand{\CharTok}[1]{\textcolor[rgb]{0.25,0.44,0.63}{{#1}}}
    \newcommand{\StringTok}[1]{\textcolor[rgb]{0.25,0.44,0.63}{{#1}}}
    \newcommand{\CommentTok}[1]{\textcolor[rgb]{0.38,0.63,0.69}{\textit{{#1}}}}
    \newcommand{\OtherTok}[1]{\textcolor[rgb]{0.00,0.44,0.13}{{#1}}}
    \newcommand{\AlertTok}[1]{\textcolor[rgb]{1.00,0.00,0.00}{\textbf{{#1}}}}
    \newcommand{\FunctionTok}[1]{\textcolor[rgb]{0.02,0.16,0.49}{{#1}}}
    \newcommand{\RegionMarkerTok}[1]{{#1}}
    \newcommand{\ErrorTok}[1]{\textcolor[rgb]{1.00,0.00,0.00}{\textbf{{#1}}}}
    \newcommand{\NormalTok}[1]{{#1}}
    
    % Additional commands for more recent versions of Pandoc
    \newcommand{\ConstantTok}[1]{\textcolor[rgb]{0.53,0.00,0.00}{{#1}}}
    \newcommand{\SpecialCharTok}[1]{\textcolor[rgb]{0.25,0.44,0.63}{{#1}}}
    \newcommand{\VerbatimStringTok}[1]{\textcolor[rgb]{0.25,0.44,0.63}{{#1}}}
    \newcommand{\SpecialStringTok}[1]{\textcolor[rgb]{0.73,0.40,0.53}{{#1}}}
    \newcommand{\ImportTok}[1]{{#1}}
    \newcommand{\DocumentationTok}[1]{\textcolor[rgb]{0.73,0.13,0.13}{\textit{{#1}}}}
    \newcommand{\AnnotationTok}[1]{\textcolor[rgb]{0.38,0.63,0.69}{\textbf{\textit{{#1}}}}}
    \newcommand{\CommentVarTok}[1]{\textcolor[rgb]{0.38,0.63,0.69}{\textbf{\textit{{#1}}}}}
    \newcommand{\VariableTok}[1]{\textcolor[rgb]{0.10,0.09,0.49}{{#1}}}
    \newcommand{\ControlFlowTok}[1]{\textcolor[rgb]{0.00,0.44,0.13}{\textbf{{#1}}}}
    \newcommand{\OperatorTok}[1]{\textcolor[rgb]{0.40,0.40,0.40}{{#1}}}
    \newcommand{\BuiltInTok}[1]{{#1}}
    \newcommand{\ExtensionTok}[1]{{#1}}
    \newcommand{\PreprocessorTok}[1]{\textcolor[rgb]{0.74,0.48,0.00}{{#1}}}
    \newcommand{\AttributeTok}[1]{\textcolor[rgb]{0.49,0.56,0.16}{{#1}}}
    \newcommand{\InformationTok}[1]{\textcolor[rgb]{0.38,0.63,0.69}{\textbf{\textit{{#1}}}}}
    \newcommand{\WarningTok}[1]{\textcolor[rgb]{0.38,0.63,0.69}{\textbf{\textit{{#1}}}}}
    
    
    % Define a nice break command that doesn't care if a line doesn't already
    % exist.
    \def\br{\hspace*{\fill} \\* }
    % Math Jax compatibility definitions
    \def\gt{>}
    \def\lt{<}
    \let\Oldtex\TeX
    \let\Oldlatex\LaTeX
    \renewcommand{\TeX}{\textrm{\Oldtex}}
    \renewcommand{\LaTeX}{\textrm{\Oldlatex}}
    % Document parameters
    % Document title
    \title{T2\_FK}
    
    
    
    
    
% Pygments definitions
\makeatletter
\def\PY@reset{\let\PY@it=\relax \let\PY@bf=\relax%
    \let\PY@ul=\relax \let\PY@tc=\relax%
    \let\PY@bc=\relax \let\PY@ff=\relax}
\def\PY@tok#1{\csname PY@tok@#1\endcsname}
\def\PY@toks#1+{\ifx\relax#1\empty\else%
    \PY@tok{#1}\expandafter\PY@toks\fi}
\def\PY@do#1{\PY@bc{\PY@tc{\PY@ul{%
    \PY@it{\PY@bf{\PY@ff{#1}}}}}}}
\def\PY#1#2{\PY@reset\PY@toks#1+\relax+\PY@do{#2}}

\@namedef{PY@tok@w}{\def\PY@tc##1{\textcolor[rgb]{0.73,0.73,0.73}{##1}}}
\@namedef{PY@tok@c}{\let\PY@it=\textit\def\PY@tc##1{\textcolor[rgb]{0.24,0.48,0.48}{##1}}}
\@namedef{PY@tok@cp}{\def\PY@tc##1{\textcolor[rgb]{0.61,0.40,0.00}{##1}}}
\@namedef{PY@tok@k}{\let\PY@bf=\textbf\def\PY@tc##1{\textcolor[rgb]{0.00,0.50,0.00}{##1}}}
\@namedef{PY@tok@kp}{\def\PY@tc##1{\textcolor[rgb]{0.00,0.50,0.00}{##1}}}
\@namedef{PY@tok@kt}{\def\PY@tc##1{\textcolor[rgb]{0.69,0.00,0.25}{##1}}}
\@namedef{PY@tok@o}{\def\PY@tc##1{\textcolor[rgb]{0.40,0.40,0.40}{##1}}}
\@namedef{PY@tok@ow}{\let\PY@bf=\textbf\def\PY@tc##1{\textcolor[rgb]{0.67,0.13,1.00}{##1}}}
\@namedef{PY@tok@nb}{\def\PY@tc##1{\textcolor[rgb]{0.00,0.50,0.00}{##1}}}
\@namedef{PY@tok@nf}{\def\PY@tc##1{\textcolor[rgb]{0.00,0.00,1.00}{##1}}}
\@namedef{PY@tok@nc}{\let\PY@bf=\textbf\def\PY@tc##1{\textcolor[rgb]{0.00,0.00,1.00}{##1}}}
\@namedef{PY@tok@nn}{\let\PY@bf=\textbf\def\PY@tc##1{\textcolor[rgb]{0.00,0.00,1.00}{##1}}}
\@namedef{PY@tok@ne}{\let\PY@bf=\textbf\def\PY@tc##1{\textcolor[rgb]{0.80,0.25,0.22}{##1}}}
\@namedef{PY@tok@nv}{\def\PY@tc##1{\textcolor[rgb]{0.10,0.09,0.49}{##1}}}
\@namedef{PY@tok@no}{\def\PY@tc##1{\textcolor[rgb]{0.53,0.00,0.00}{##1}}}
\@namedef{PY@tok@nl}{\def\PY@tc##1{\textcolor[rgb]{0.46,0.46,0.00}{##1}}}
\@namedef{PY@tok@ni}{\let\PY@bf=\textbf\def\PY@tc##1{\textcolor[rgb]{0.44,0.44,0.44}{##1}}}
\@namedef{PY@tok@na}{\def\PY@tc##1{\textcolor[rgb]{0.41,0.47,0.13}{##1}}}
\@namedef{PY@tok@nt}{\let\PY@bf=\textbf\def\PY@tc##1{\textcolor[rgb]{0.00,0.50,0.00}{##1}}}
\@namedef{PY@tok@nd}{\def\PY@tc##1{\textcolor[rgb]{0.67,0.13,1.00}{##1}}}
\@namedef{PY@tok@s}{\def\PY@tc##1{\textcolor[rgb]{0.73,0.13,0.13}{##1}}}
\@namedef{PY@tok@sd}{\let\PY@it=\textit\def\PY@tc##1{\textcolor[rgb]{0.73,0.13,0.13}{##1}}}
\@namedef{PY@tok@si}{\let\PY@bf=\textbf\def\PY@tc##1{\textcolor[rgb]{0.64,0.35,0.47}{##1}}}
\@namedef{PY@tok@se}{\let\PY@bf=\textbf\def\PY@tc##1{\textcolor[rgb]{0.67,0.36,0.12}{##1}}}
\@namedef{PY@tok@sr}{\def\PY@tc##1{\textcolor[rgb]{0.64,0.35,0.47}{##1}}}
\@namedef{PY@tok@ss}{\def\PY@tc##1{\textcolor[rgb]{0.10,0.09,0.49}{##1}}}
\@namedef{PY@tok@sx}{\def\PY@tc##1{\textcolor[rgb]{0.00,0.50,0.00}{##1}}}
\@namedef{PY@tok@m}{\def\PY@tc##1{\textcolor[rgb]{0.40,0.40,0.40}{##1}}}
\@namedef{PY@tok@gh}{\let\PY@bf=\textbf\def\PY@tc##1{\textcolor[rgb]{0.00,0.00,0.50}{##1}}}
\@namedef{PY@tok@gu}{\let\PY@bf=\textbf\def\PY@tc##1{\textcolor[rgb]{0.50,0.00,0.50}{##1}}}
\@namedef{PY@tok@gd}{\def\PY@tc##1{\textcolor[rgb]{0.63,0.00,0.00}{##1}}}
\@namedef{PY@tok@gi}{\def\PY@tc##1{\textcolor[rgb]{0.00,0.52,0.00}{##1}}}
\@namedef{PY@tok@gr}{\def\PY@tc##1{\textcolor[rgb]{0.89,0.00,0.00}{##1}}}
\@namedef{PY@tok@ge}{\let\PY@it=\textit}
\@namedef{PY@tok@gs}{\let\PY@bf=\textbf}
\@namedef{PY@tok@gp}{\let\PY@bf=\textbf\def\PY@tc##1{\textcolor[rgb]{0.00,0.00,0.50}{##1}}}
\@namedef{PY@tok@go}{\def\PY@tc##1{\textcolor[rgb]{0.44,0.44,0.44}{##1}}}
\@namedef{PY@tok@gt}{\def\PY@tc##1{\textcolor[rgb]{0.00,0.27,0.87}{##1}}}
\@namedef{PY@tok@err}{\def\PY@bc##1{{\setlength{\fboxsep}{\string -\fboxrule}\fcolorbox[rgb]{1.00,0.00,0.00}{1,1,1}{\strut ##1}}}}
\@namedef{PY@tok@kc}{\let\PY@bf=\textbf\def\PY@tc##1{\textcolor[rgb]{0.00,0.50,0.00}{##1}}}
\@namedef{PY@tok@kd}{\let\PY@bf=\textbf\def\PY@tc##1{\textcolor[rgb]{0.00,0.50,0.00}{##1}}}
\@namedef{PY@tok@kn}{\let\PY@bf=\textbf\def\PY@tc##1{\textcolor[rgb]{0.00,0.50,0.00}{##1}}}
\@namedef{PY@tok@kr}{\let\PY@bf=\textbf\def\PY@tc##1{\textcolor[rgb]{0.00,0.50,0.00}{##1}}}
\@namedef{PY@tok@bp}{\def\PY@tc##1{\textcolor[rgb]{0.00,0.50,0.00}{##1}}}
\@namedef{PY@tok@fm}{\def\PY@tc##1{\textcolor[rgb]{0.00,0.00,1.00}{##1}}}
\@namedef{PY@tok@vc}{\def\PY@tc##1{\textcolor[rgb]{0.10,0.09,0.49}{##1}}}
\@namedef{PY@tok@vg}{\def\PY@tc##1{\textcolor[rgb]{0.10,0.09,0.49}{##1}}}
\@namedef{PY@tok@vi}{\def\PY@tc##1{\textcolor[rgb]{0.10,0.09,0.49}{##1}}}
\@namedef{PY@tok@vm}{\def\PY@tc##1{\textcolor[rgb]{0.10,0.09,0.49}{##1}}}
\@namedef{PY@tok@sa}{\def\PY@tc##1{\textcolor[rgb]{0.73,0.13,0.13}{##1}}}
\@namedef{PY@tok@sb}{\def\PY@tc##1{\textcolor[rgb]{0.73,0.13,0.13}{##1}}}
\@namedef{PY@tok@sc}{\def\PY@tc##1{\textcolor[rgb]{0.73,0.13,0.13}{##1}}}
\@namedef{PY@tok@dl}{\def\PY@tc##1{\textcolor[rgb]{0.73,0.13,0.13}{##1}}}
\@namedef{PY@tok@s2}{\def\PY@tc##1{\textcolor[rgb]{0.73,0.13,0.13}{##1}}}
\@namedef{PY@tok@sh}{\def\PY@tc##1{\textcolor[rgb]{0.73,0.13,0.13}{##1}}}
\@namedef{PY@tok@s1}{\def\PY@tc##1{\textcolor[rgb]{0.73,0.13,0.13}{##1}}}
\@namedef{PY@tok@mb}{\def\PY@tc##1{\textcolor[rgb]{0.40,0.40,0.40}{##1}}}
\@namedef{PY@tok@mf}{\def\PY@tc##1{\textcolor[rgb]{0.40,0.40,0.40}{##1}}}
\@namedef{PY@tok@mh}{\def\PY@tc##1{\textcolor[rgb]{0.40,0.40,0.40}{##1}}}
\@namedef{PY@tok@mi}{\def\PY@tc##1{\textcolor[rgb]{0.40,0.40,0.40}{##1}}}
\@namedef{PY@tok@il}{\def\PY@tc##1{\textcolor[rgb]{0.40,0.40,0.40}{##1}}}
\@namedef{PY@tok@mo}{\def\PY@tc##1{\textcolor[rgb]{0.40,0.40,0.40}{##1}}}
\@namedef{PY@tok@ch}{\let\PY@it=\textit\def\PY@tc##1{\textcolor[rgb]{0.24,0.48,0.48}{##1}}}
\@namedef{PY@tok@cm}{\let\PY@it=\textit\def\PY@tc##1{\textcolor[rgb]{0.24,0.48,0.48}{##1}}}
\@namedef{PY@tok@cpf}{\let\PY@it=\textit\def\PY@tc##1{\textcolor[rgb]{0.24,0.48,0.48}{##1}}}
\@namedef{PY@tok@c1}{\let\PY@it=\textit\def\PY@tc##1{\textcolor[rgb]{0.24,0.48,0.48}{##1}}}
\@namedef{PY@tok@cs}{\let\PY@it=\textit\def\PY@tc##1{\textcolor[rgb]{0.24,0.48,0.48}{##1}}}

\def\PYZbs{\char`\\}
\def\PYZus{\char`\_}
\def\PYZob{\char`\{}
\def\PYZcb{\char`\}}
\def\PYZca{\char`\^}
\def\PYZam{\char`\&}
\def\PYZlt{\char`\<}
\def\PYZgt{\char`\>}
\def\PYZsh{\char`\#}
\def\PYZpc{\char`\%}
\def\PYZdl{\char`\$}
\def\PYZhy{\char`\-}
\def\PYZsq{\char`\'}
\def\PYZdq{\char`\"}
\def\PYZti{\char`\~}
% for compatibility with earlier versions
\def\PYZat{@}
\def\PYZlb{[}
\def\PYZrb{]}
\makeatother


    % For linebreaks inside Verbatim environment from package fancyvrb. 
    \makeatletter
        \newbox\Wrappedcontinuationbox 
        \newbox\Wrappedvisiblespacebox 
        \newcommand*\Wrappedvisiblespace {\textcolor{red}{\textvisiblespace}} 
        \newcommand*\Wrappedcontinuationsymbol {\textcolor{red}{\llap{\tiny$\m@th\hookrightarrow$}}} 
        \newcommand*\Wrappedcontinuationindent {3ex } 
        \newcommand*\Wrappedafterbreak {\kern\Wrappedcontinuationindent\copy\Wrappedcontinuationbox} 
        % Take advantage of the already applied Pygments mark-up to insert 
        % potential linebreaks for TeX processing. 
        %        {, <, #, %, $, ' and ": go to next line. 
        %        _, }, ^, &, >, - and ~: stay at end of broken line. 
        % Use of \textquotesingle for straight quote. 
        \newcommand*\Wrappedbreaksatspecials {% 
            \def\PYGZus{\discretionary{\char`\_}{\Wrappedafterbreak}{\char`\_}}% 
            \def\PYGZob{\discretionary{}{\Wrappedafterbreak\char`\{}{\char`\{}}% 
            \def\PYGZcb{\discretionary{\char`\}}{\Wrappedafterbreak}{\char`\}}}% 
            \def\PYGZca{\discretionary{\char`\^}{\Wrappedafterbreak}{\char`\^}}% 
            \def\PYGZam{\discretionary{\char`\&}{\Wrappedafterbreak}{\char`\&}}% 
            \def\PYGZlt{\discretionary{}{\Wrappedafterbreak\char`\<}{\char`\<}}% 
            \def\PYGZgt{\discretionary{\char`\>}{\Wrappedafterbreak}{\char`\>}}% 
            \def\PYGZsh{\discretionary{}{\Wrappedafterbreak\char`\#}{\char`\#}}% 
            \def\PYGZpc{\discretionary{}{\Wrappedafterbreak\char`\%}{\char`\%}}% 
            \def\PYGZdl{\discretionary{}{\Wrappedafterbreak\char`\$}{\char`\$}}% 
            \def\PYGZhy{\discretionary{\char`\-}{\Wrappedafterbreak}{\char`\-}}% 
            \def\PYGZsq{\discretionary{}{\Wrappedafterbreak\textquotesingle}{\textquotesingle}}% 
            \def\PYGZdq{\discretionary{}{\Wrappedafterbreak\char`\"}{\char`\"}}% 
            \def\PYGZti{\discretionary{\char`\~}{\Wrappedafterbreak}{\char`\~}}% 
        } 
        % Some characters . , ; ? ! / are not pygmentized. 
        % This macro makes them "active" and they will insert potential linebreaks 
        \newcommand*\Wrappedbreaksatpunct {% 
            \lccode`\~`\.\lowercase{\def~}{\discretionary{\hbox{\char`\.}}{\Wrappedafterbreak}{\hbox{\char`\.}}}% 
            \lccode`\~`\,\lowercase{\def~}{\discretionary{\hbox{\char`\,}}{\Wrappedafterbreak}{\hbox{\char`\,}}}% 
            \lccode`\~`\;\lowercase{\def~}{\discretionary{\hbox{\char`\;}}{\Wrappedafterbreak}{\hbox{\char`\;}}}% 
            \lccode`\~`\:\lowercase{\def~}{\discretionary{\hbox{\char`\:}}{\Wrappedafterbreak}{\hbox{\char`\:}}}% 
            \lccode`\~`\?\lowercase{\def~}{\discretionary{\hbox{\char`\?}}{\Wrappedafterbreak}{\hbox{\char`\?}}}% 
            \lccode`\~`\!\lowercase{\def~}{\discretionary{\hbox{\char`\!}}{\Wrappedafterbreak}{\hbox{\char`\!}}}% 
            \lccode`\~`\/\lowercase{\def~}{\discretionary{\hbox{\char`\/}}{\Wrappedafterbreak}{\hbox{\char`\/}}}% 
            \catcode`\.\active
            \catcode`\,\active 
            \catcode`\;\active
            \catcode`\:\active
            \catcode`\?\active
            \catcode`\!\active
            \catcode`\/\active 
            \lccode`\~`\~ 	
        }
    \makeatother

    \let\OriginalVerbatim=\Verbatim
    \makeatletter
    \renewcommand{\Verbatim}[1][1]{%
        %\parskip\z@skip
        \sbox\Wrappedcontinuationbox {\Wrappedcontinuationsymbol}%
        \sbox\Wrappedvisiblespacebox {\FV@SetupFont\Wrappedvisiblespace}%
        \def\FancyVerbFormatLine ##1{\hsize\linewidth
            \vtop{\raggedright\hyphenpenalty\z@\exhyphenpenalty\z@
                \doublehyphendemerits\z@\finalhyphendemerits\z@
                \strut ##1\strut}%
        }%
        % If the linebreak is at a space, the latter will be displayed as visible
        % space at end of first line, and a continuation symbol starts next line.
        % Stretch/shrink are however usually zero for typewriter font.
        \def\FV@Space {%
            \nobreak\hskip\z@ plus\fontdimen3\font minus\fontdimen4\font
            \discretionary{\copy\Wrappedvisiblespacebox}{\Wrappedafterbreak}
            {\kern\fontdimen2\font}%
        }%
        
        % Allow breaks at special characters using \PYG... macros.
        \Wrappedbreaksatspecials
        % Breaks at punctuation characters . , ; ? ! and / need catcode=\active 	
        \OriginalVerbatim[#1,codes*=\Wrappedbreaksatpunct]%
    }
    \makeatother

    % Exact colors from NB
    \definecolor{incolor}{HTML}{303F9F}
    \definecolor{outcolor}{HTML}{D84315}
    \definecolor{cellborder}{HTML}{CFCFCF}
    \definecolor{cellbackground}{HTML}{F7F7F7}
    
    % prompt
    \makeatletter
    \newcommand{\boxspacing}{\kern\kvtcb@left@rule\kern\kvtcb@boxsep}
    \makeatother
    \newcommand{\prompt}[4]{
        {\ttfamily\llap{{\color{#2}[#3]:\hspace{3pt}#4}}\vspace{-\baselineskip}}
    }
    

    
    % Prevent overflowing lines due to hard-to-break entities
    \sloppy 
    % Setup hyperref package
    \hypersetup{
      breaklinks=true,  % so long urls are correctly broken across lines
      colorlinks=true,
      urlcolor=urlcolor,
      linkcolor=linkcolor,
      citecolor=citecolor,
      }
    % Slightly bigger margins than the latex defaults
    
    \geometry{verbose,tmargin=1in,bmargin=1in,lmargin=1in,rmargin=1in}
    
    

\begin{document}
    
    \maketitle
    
    

    
    \begin{tcolorbox}[breakable, size=fbox, boxrule=1pt, pad at break*=1mm,colback=cellbackground, colframe=cellborder]
\prompt{In}{incolor}{165}{\boxspacing}
\begin{Verbatim}[commandchars=\\\{\}]
\PY{n+nf}{options}\PY{p}{(}\PY{n}{repr.plot.width}\PY{o}{=}\PY{l+m}{20}\PY{p}{,} \PY{n}{repr.plot.height}\PY{o}{=}\PY{l+m}{10}\PY{p}{)}  \PY{c+c1}{\PYZsh{} Ajuste dos gráficos}
\end{Verbatim}
\end{tcolorbox}

    Instalando e carregando as bibliotecas utilizadas

    \begin{tcolorbox}[breakable, size=fbox, boxrule=1pt, pad at break*=1mm,colback=cellbackground, colframe=cellborder]
\prompt{In}{incolor}{166}{\boxspacing}
\begin{Verbatim}[commandchars=\\\{\}]
\PY{n+nf}{library}\PY{p}{(}\PY{n}{pacman}\PY{p}{)}
\PY{n+nf}{p\PYZus{}load}\PY{p}{(}\PY{n}{ggplot2}\PY{p}{,}\PY{n}{forecast}\PY{p}{,}\PY{n}{dlm}\PY{p}{,}\PY{n}{numDeriv}\PY{p}{,}\PY{n}{plotly}\PY{p}{)}
\end{Verbatim}
\end{tcolorbox}

    Carregando as funções auxiliares, que não podem ser mostras aqui ainda

    \begin{tcolorbox}[breakable, size=fbox, boxrule=1pt, pad at break*=1mm,colback=cellbackground, colframe=cellborder]
\prompt{In}{incolor}{176}{\boxspacing}
\begin{Verbatim}[commandchars=\\\{\}]
\PY{n+nf}{source}\PY{p}{(}\PY{l+s}{\PYZsq{}}\PY{l+s}{funcoes\PYZus{}auxiliares.R\PYZsq{}}\PY{p}{)}
\end{Verbatim}
\end{tcolorbox}

    \subsection{Modelo MNL}\label{modelo-mnl}

    Simulando o y conforme a minha função do Trabalho 1:

    \begin{tcolorbox}[breakable, size=fbox, boxrule=1pt, pad at break*=1mm,colback=cellbackground, colframe=cellborder]
\prompt{In}{incolor}{152}{\boxspacing}
\begin{Verbatim}[commandchars=\\\{\}]
\PY{n}{y\PYZus{}mnl} \PY{o}{\PYZlt{}\PYZhy{}} \PY{n+nf}{simul\PYZus{}y\PYZus{}mnl}\PY{p}{(}\PY{n+nb+bp}{T}\PY{o}{=}\PY{l+m}{100}\PY{p}{,}\PY{l+m}{10}\PY{p}{,}\PY{l+m}{0.5}\PY{p}{,}\PY{n}{n\PYZus{}seed}\PY{o}{=}\PY{l+m}{1}\PY{p}{)}
\end{Verbatim}
\end{tcolorbox}

    Utilizando o StructTs para estimar os hyperparâmetros via MLE:

    \begin{tcolorbox}[breakable, size=fbox, boxrule=1pt, pad at break*=1mm,colback=cellbackground, colframe=cellborder]
\prompt{In}{incolor}{153}{\boxspacing}
\begin{Verbatim}[commandchars=\\\{\}]
\PY{n}{fit} \PY{o}{\PYZlt{}\PYZhy{}} \PY{n+nf}{StructTS}\PY{p}{(}\PY{n}{y\PYZus{}mnl}\PY{p}{,} \PY{l+s}{\PYZdq{}}\PY{l+s}{level\PYZdq{}}\PY{p}{)}
\end{Verbatim}
\end{tcolorbox}

    Criando um dataframe com o tempo (1:T), a série simulada e o nível
ajustado:

    \begin{tcolorbox}[breakable, size=fbox, boxrule=1pt, pad at break*=1mm,colback=cellbackground, colframe=cellborder]
\prompt{In}{incolor}{154}{\boxspacing}
\begin{Verbatim}[commandchars=\\\{\}]
\PY{n}{df} \PY{o}{\PYZlt{}\PYZhy{}} \PY{n+nf}{data.frame}\PY{p}{(}
  \PY{n}{time} \PY{o}{=} \PY{l+m}{1}\PY{o}{:}\PY{n+nf}{length}\PY{p}{(}\PY{n}{y\PYZus{}mnl}\PY{p}{)}\PY{p}{,}
  \PY{n}{observed} \PY{o}{=} \PY{n}{y\PYZus{}mnl}\PY{p}{,}
  \PY{n}{estimated\PYZus{}level} \PY{o}{=} \PY{n+nf}{fitted}\PY{p}{(}\PY{n}{fit}\PY{p}{)}\PY{p}{[}\PY{p}{,} \PY{l+s}{\PYZdq{}}\PY{l+s}{level\PYZdq{}}\PY{p}{]}
\PY{p}{)}
\end{Verbatim}
\end{tcolorbox}

    Utilizando o objeto `fit' , podemos extrair as estimativas para os
hyperparâmetros:

    \begin{tcolorbox}[breakable, size=fbox, boxrule=1pt, pad at break*=1mm,colback=cellbackground, colframe=cellborder]
\prompt{In}{incolor}{155}{\boxspacing}
\begin{Verbatim}[commandchars=\\\{\}]
\PY{c+c1}{\PYZsh{} Extract estimated variances}
\PY{n}{sigma2\PYZus{}epsilon} \PY{o}{\PYZlt{}\PYZhy{}} \PY{n}{fit}\PY{o}{\PYZdl{}}\PY{n}{coef}\PY{p}{[}\PY{l+s}{\PYZdq{}}\PY{l+s}{epsilon\PYZdq{}}\PY{p}{]}
\PY{n}{sigma2\PYZus{}eta} \PY{o}{\PYZlt{}\PYZhy{}} \PY{n}{fit}\PY{o}{\PYZdl{}}\PY{n}{coef}\PY{p}{[}\PY{l+s}{\PYZdq{}}\PY{l+s}{level\PYZdq{}}\PY{p}{]}
\end{Verbatim}
\end{tcolorbox}

    Podemos fazer o gráfico do nível estimado contra a série simulada
utilizando o ggplot2

    \begin{tcolorbox}[breakable, size=fbox, boxrule=1pt, pad at break*=1mm,colback=cellbackground, colframe=cellborder]
\prompt{In}{incolor}{156}{\boxspacing}
\begin{Verbatim}[commandchars=\\\{\}]
\PY{n+nf}{ggplot}\PY{p}{(}\PY{p}{)} \PY{o}{+} 
\PY{n+nf}{geom\PYZus{}line}\PY{p}{(}\PY{n}{data}\PY{o}{=}\PY{n}{df}\PY{p}{,}\PY{n+nf}{aes}\PY{p}{(}\PY{n}{y} \PY{o}{=} \PY{n}{observed}\PY{p}{,}\PY{n}{x} \PY{o}{=} \PY{n}{time}\PY{p}{,} \PY{n}{color} \PY{o}{=} \PY{l+s}{\PYZdq{}}\PY{l+s}{Observed y\PYZus{}mnl\PYZdq{}}\PY{p}{)}\PY{p}{,} \PY{n}{linewidth} \PY{o}{=} \PY{l+m}{1}\PY{p}{)} \PY{o}{+}
\PY{n+nf}{geom\PYZus{}line}\PY{p}{(}\PY{n}{data}\PY{o}{=}\PY{n}{df}\PY{p}{,}\PY{n+nf}{aes}\PY{p}{(}\PY{n}{y} \PY{o}{=} \PY{n}{estimated\PYZus{}level}\PY{p}{,}\PY{n}{x} \PY{o}{=} \PY{n}{time}\PY{p}{,} \PY{n}{color} \PY{o}{=} \PY{l+s}{\PYZdq{}}\PY{l+s}{Estimated Level\PYZdq{}}\PY{p}{)}\PY{p}{,} \PY{n}{linewidth} \PY{o}{=} \PY{l+m}{1}\PY{p}{)} \PY{o}{+}
\PY{n+nf}{scale\PYZus{}color\PYZus{}manual}\PY{p}{(}\PY{n}{values} \PY{o}{=} \PY{n+nf}{c}\PY{p}{(}\PY{l+s}{\PYZdq{}}\PY{l+s}{Observed y\PYZus{}mnl\PYZdq{}} \PY{o}{=} \PY{l+s}{\PYZdq{}}\PY{l+s}{black\PYZdq{}}\PY{p}{,} \PY{l+s}{\PYZdq{}}\PY{l+s}{Estimated Level\PYZdq{}} \PY{o}{=} \PY{l+s}{\PYZdq{}}\PY{l+s}{red\PYZdq{}}\PY{p}{)}\PY{p}{)} \PY{o}{+}
\PY{n+nf}{labs}\PY{p}{(}
\PY{n}{title} \PY{o}{=} \PY{l+s}{\PYZdq{}}\PY{l+s}{Observed Data vs. Estimated Level\PYZdq{}}\PY{p}{,}
\PY{n}{x} \PY{o}{=} \PY{l+s}{\PYZdq{}}\PY{l+s}{Time\PYZdq{}}\PY{p}{,}
\PY{n}{y} \PY{o}{=} \PY{l+s}{\PYZdq{}}\PY{l+s}{Value\PYZdq{}}\PY{p}{,}
\PY{n}{color} \PY{o}{=} \PY{l+s}{\PYZdq{}}\PY{l+s}{Legend\PYZdq{}}
\PY{p}{)} \PY{o}{+}
  
\PY{n+nf}{theme\PYZus{}minimal}\PY{p}{(}\PY{p}{)}\PY{o}{+}

\PY{n+nf}{theme}\PY{p}{(}
    \PY{n}{panel.grid.major} \PY{o}{=} \PY{n+nf}{element\PYZus{}line}\PY{p}{(}\PY{n}{color} \PY{o}{=} \PY{l+s}{\PYZdq{}}\PY{l+s}{gray\PYZdq{}}\PY{p}{,} \PY{n}{linewidth} \PY{o}{=} \PY{l+m}{1}\PY{p}{)}\PY{p}{,}
    \PY{n}{panel.grid.minor} \PY{o}{=} \PY{n+nf}{element\PYZus{}line}\PY{p}{(}\PY{n}{color} \PY{o}{=} \PY{l+s}{\PYZdq{}}\PY{l+s}{lightgray\PYZdq{}}\PY{p}{,} \PY{n}{linewidth} \PY{o}{=} \PY{l+m}{0.5}\PY{p}{)}\PY{p}{,}
    \PY{n}{axis.line} \PY{o}{=} \PY{n+nf}{element\PYZus{}line}\PY{p}{(}\PY{n}{color} \PY{o}{=} \PY{l+s}{\PYZdq{}}\PY{l+s}{black\PYZdq{}}\PY{p}{,} \PY{n}{linewidth} \PY{o}{=} \PY{l+m}{1}\PY{p}{)}\PY{p}{,}  \PY{c+c1}{\PYZsh{} Make axis lines thicker and black}
    \PY{n}{axis.ticks} \PY{o}{=} \PY{n+nf}{element\PYZus{}line}\PY{p}{(}\PY{n}{color} \PY{o}{=} \PY{l+s}{\PYZdq{}}\PY{l+s}{black\PYZdq{}}\PY{p}{,} \PY{n}{linewidth} \PY{o}{=} \PY{l+m}{0.8}\PY{p}{)}\PY{p}{,}  \PY{c+c1}{\PYZsh{} Make ticks more visible}
    \PY{n}{axis.text} \PY{o}{=} \PY{n+nf}{element\PYZus{}text}\PY{p}{(}\PY{n}{size} \PY{o}{=} \PY{l+m}{12}\PY{p}{,} \PY{n}{color} \PY{o}{=} \PY{l+s}{\PYZdq{}}\PY{l+s}{black\PYZdq{}}\PY{p}{)}\PY{p}{,}  \PY{c+c1}{\PYZsh{} Adjust axis text size and color}
    \PY{n}{axis.title} \PY{o}{=} \PY{n+nf}{element\PYZus{}text}\PY{p}{(}\PY{n}{size} \PY{o}{=} \PY{l+m}{14}\PY{p}{,} \PY{n}{color} \PY{o}{=} \PY{l+s}{\PYZdq{}}\PY{l+s}{black\PYZdq{}}\PY{p}{)}  \PY{c+c1}{\PYZsh{} Adjust axis title size and color}
    \PY{p}{)}
\end{Verbatim}
\end{tcolorbox}

    \begin{center}
    \adjustimage{max size={0.9\linewidth}{0.9\paperheight}}{T2_FK_files/T2_FK_15_0.png}
    \end{center}
    { \hspace*{\fill} \\}
    
    Recuperando o valor dos hyperparâmetros:

    \begin{tcolorbox}[breakable, size=fbox, boxrule=1pt, pad at break*=1mm,colback=cellbackground, colframe=cellborder]
\prompt{In}{incolor}{157}{\boxspacing}
\begin{Verbatim}[commandchars=\\\{\}]
\PY{c+c1}{\PYZsh{} Print estimated variances}
\PY{n+nf}{cat}\PY{p}{(}\PY{l+s}{\PYZdq{}}\PY{l+s}{Estimated Observation Noise Variance (sigma\PYZca{}2\PYZus{}epsilon):\PYZdq{}}\PY{p}{,} \PY{n}{sigma2\PYZus{}epsilon}\PY{p}{,} \PY{l+s}{\PYZdq{}}\PY{l+s}{\PYZbs{}n\PYZdq{}}\PY{p}{)}
\PY{n+nf}{cat}\PY{p}{(}\PY{l+s}{\PYZdq{}}\PY{l+s}{Estimated State Noise Variance (sigma\PYZca{}2\PYZus{}eta):\PYZdq{}}\PY{p}{,} \PY{n}{sigma2\PYZus{}eta}\PY{p}{,} \PY{l+s}{\PYZdq{}}\PY{l+s}{\PYZbs{}n\PYZdq{}}\PY{p}{)}
\end{Verbatim}
\end{tcolorbox}

    \begin{Verbatim}[commandchars=\\\{\}]
Estimated Observation Noise Variance (sigma\^{}2\_epsilon): 7.7368
Estimated State Noise Variance (sigma\^{}2\_eta): 0.5600975
    \end{Verbatim}

    Será que o estimador MLE é consistente? Vamos aumentar o T e ver se as
estimativas melhoram.

    \begin{tcolorbox}[breakable, size=fbox, boxrule=1pt, pad at break*=1mm,colback=cellbackground, colframe=cellborder]
\prompt{In}{incolor}{158}{\boxspacing}
\begin{Verbatim}[commandchars=\\\{\}]
\PY{n}{y\PYZus{}mnl} \PY{o}{\PYZlt{}\PYZhy{}} \PY{n+nf}{simul\PYZus{}y\PYZus{}mnl}\PY{p}{(}\PY{n+nb+bp}{T}\PY{o}{=}\PY{l+m}{1000}\PY{p}{,}\PY{l+m}{10}\PY{p}{,}\PY{l+m}{0.5}\PY{p}{,}\PY{l+m}{1}\PY{p}{)}
\end{Verbatim}
\end{tcolorbox}

    \begin{tcolorbox}[breakable, size=fbox, boxrule=1pt, pad at break*=1mm,colback=cellbackground, colframe=cellborder]
\prompt{In}{incolor}{159}{\boxspacing}
\begin{Verbatim}[commandchars=\\\{\}]
\PY{n}{fit} \PY{o}{\PYZlt{}\PYZhy{}} \PY{n+nf}{StructTS}\PY{p}{(}\PY{n}{y\PYZus{}mnl}\PY{p}{,} \PY{l+s}{\PYZdq{}}\PY{l+s}{level\PYZdq{}}\PY{p}{)}
\end{Verbatim}
\end{tcolorbox}

    \begin{tcolorbox}[breakable, size=fbox, boxrule=1pt, pad at break*=1mm,colback=cellbackground, colframe=cellborder]
\prompt{In}{incolor}{160}{\boxspacing}
\begin{Verbatim}[commandchars=\\\{\}]
\PY{n}{df} \PY{o}{\PYZlt{}\PYZhy{}} \PY{n+nf}{data.frame}\PY{p}{(}
  \PY{n}{time} \PY{o}{=} \PY{l+m}{1}\PY{o}{:}\PY{n+nf}{length}\PY{p}{(}\PY{n}{y\PYZus{}mnl}\PY{p}{)}\PY{p}{,}
  \PY{n}{observed} \PY{o}{=} \PY{n}{y\PYZus{}mnl}\PY{p}{,}
  \PY{n}{estimated\PYZus{}level} \PY{o}{=} \PY{n+nf}{fitted}\PY{p}{(}\PY{n}{fit}\PY{p}{)}\PY{p}{[}\PY{p}{,} \PY{l+s}{\PYZdq{}}\PY{l+s}{level\PYZdq{}}\PY{p}{]}
\PY{p}{)}
\end{Verbatim}
\end{tcolorbox}

    \begin{tcolorbox}[breakable, size=fbox, boxrule=1pt, pad at break*=1mm,colback=cellbackground, colframe=cellborder]
\prompt{In}{incolor}{161}{\boxspacing}
\begin{Verbatim}[commandchars=\\\{\}]
\PY{n}{sigma2\PYZus{}epsilon} \PY{o}{\PYZlt{}\PYZhy{}} \PY{n}{fit}\PY{o}{\PYZdl{}}\PY{n}{coef}\PY{p}{[}\PY{l+s}{\PYZdq{}}\PY{l+s}{epsilon\PYZdq{}}\PY{p}{]}
\PY{n}{sigma2\PYZus{}eta} \PY{o}{\PYZlt{}\PYZhy{}} \PY{n}{fit}\PY{o}{\PYZdl{}}\PY{n}{coef}\PY{p}{[}\PY{l+s}{\PYZdq{}}\PY{l+s}{level\PYZdq{}}\PY{p}{]}
\end{Verbatim}
\end{tcolorbox}

    \begin{tcolorbox}[breakable, size=fbox, boxrule=1pt, pad at break*=1mm,colback=cellbackground, colframe=cellborder]
\prompt{In}{incolor}{162}{\boxspacing}
\begin{Verbatim}[commandchars=\\\{\}]
\PY{n+nf}{cat}\PY{p}{(}\PY{l+s}{\PYZdq{}}\PY{l+s}{Estimated Observation Noise Variance (sigma\PYZca{}2\PYZus{}epsilon):\PYZdq{}}\PY{p}{,} \PY{n}{sigma2\PYZus{}epsilon}\PY{p}{,} \PY{l+s}{\PYZdq{}}\PY{l+s}{\PYZbs{}n\PYZdq{}}\PY{p}{)}
\PY{n+nf}{cat}\PY{p}{(}\PY{l+s}{\PYZdq{}}\PY{l+s}{Estimated State Noise Variance (sigma\PYZca{}2\PYZus{}eta):\PYZdq{}}\PY{p}{,} \PY{n}{sigma2\PYZus{}eta}\PY{p}{,} \PY{l+s}{\PYZdq{}}\PY{l+s}{\PYZbs{}n\PYZdq{}}\PY{p}{)}
\end{Verbatim}
\end{tcolorbox}

    \begin{Verbatim}[commandchars=\\\{\}]
Estimated Observation Noise Variance (sigma\^{}2\_epsilon): 10.83406
Estimated State Noise Variance (sigma\^{}2\_eta): 0.6076557
    \end{Verbatim}

    Parece que com mais observações, as estimativas ficam mais próximas do
valor verdadeiro.

    No trabalho 1, conseguimos achar a sequencia a\_t(1:T) e F\_t{[}1:T{]} a
partir de um chute incial para a\_0 e p\_0.

    Pergunta: será que se usarmos essa função do trabalho 2 utilizando as
variâncias estimadas, conseguimos reproduzir a série do nível estimado?

    Novamente:

    \begin{tcolorbox}[breakable, size=fbox, boxrule=1pt, pad at break*=1mm,colback=cellbackground, colframe=cellborder]
\prompt{In}{incolor}{163}{\boxspacing}
\begin{Verbatim}[commandchars=\\\{\}]
\PY{n}{y\PYZus{}mnl} \PY{o}{\PYZlt{}\PYZhy{}} \PY{n+nf}{simul\PYZus{}y\PYZus{}mnl}\PY{p}{(}\PY{n+nb+bp}{T}\PY{o}{=}\PY{l+m}{1000}\PY{p}{,}\PY{l+m}{1}\PY{p}{,}\PY{l+m}{0.5}\PY{p}{,}\PY{l+m}{1}\PY{p}{)}

\PY{n}{fit} \PY{o}{\PYZlt{}\PYZhy{}} \PY{n+nf}{StructTS}\PY{p}{(}\PY{n}{y\PYZus{}mnl}\PY{p}{,} \PY{l+s}{\PYZdq{}}\PY{l+s}{level\PYZdq{}}\PY{p}{)}
\end{Verbatim}
\end{tcolorbox}

    Utilizando a função da questão 2 do trabalho 1:

    \begin{tcolorbox}[breakable, size=fbox, boxrule=1pt, pad at break*=1mm,colback=cellbackground, colframe=cellborder]
\prompt{In}{incolor}{177}{\boxspacing}
\begin{Verbatim}[commandchars=\\\{\}]
\PY{n}{fitted}\PY{o}{\PYZlt{}\PYZhy{}}\PY{n+nf}{mnl\PYZus{}fk}\PY{p}{(}\PY{n+nb+bp}{T}\PY{o}{=}\PY{l+m}{1000}\PY{p}{,}\PY{n}{y\PYZus{}mnl}\PY{p}{,}\PY{n}{fit}\PY{o}{\PYZdl{}}\PY{n}{coef}\PY{p}{[}\PY{l+s}{\PYZdq{}}\PY{l+s}{epsilon\PYZdq{}}\PY{p}{]}\PY{p}{,}\PY{n}{fit}\PY{o}{\PYZdl{}}\PY{n}{coef}\PY{p}{[}\PY{l+s}{\PYZdq{}}\PY{l+s}{level\PYZdq{}}\PY{p}{]}\PY{p}{,}\PY{n}{a0}\PY{o}{=}\PY{n}{fit}\PY{o}{\PYZdl{}}\PY{n}{model0}\PY{o}{\PYZdl{}}\PY{n}{a}\PY{p}{,}\PY{n}{p0}\PY{o}{=}\PY{n}{fit}\PY{o}{\PYZdl{}}\PY{n}{model0}\PY{o}{\PYZdl{}}\PY{n}{P}\PY{p}{)}
\end{Verbatim}
\end{tcolorbox}

    \begin{tcolorbox}[breakable, size=fbox, boxrule=1pt, pad at break*=1mm,colback=cellbackground, colframe=cellborder]
\prompt{In}{incolor}{178}{\boxspacing}
\begin{Verbatim}[commandchars=\\\{\}]
\PY{n}{df\PYZus{}fitted\PYZus{}mnl} \PY{o}{\PYZlt{}\PYZhy{}} \PY{n+nf}{data.frame}\PY{p}{(}\PY{n}{manual} \PY{o}{=} \PY{n}{fitted}\PY{o}{\PYZdl{}}\PY{n}{a}\PY{p}{,}\PY{n}{structts}\PY{o}{=}\PY{n}{fit}\PY{o}{\PYZdl{}}\PY{n}{fitted}\PY{p}{[}\PY{p}{,}\PY{l+s}{\PYZsq{}}\PY{l+s}{level\PYZsq{}}\PY{p}{]}\PY{p}{)}
\end{Verbatim}
\end{tcolorbox}

    \begin{tcolorbox}[breakable, size=fbox, boxrule=1pt, pad at break*=1mm,colback=cellbackground, colframe=cellborder]
\prompt{In}{incolor}{179}{\boxspacing}
\begin{Verbatim}[commandchars=\\\{\}]
\PY{n+nf}{tail}\PY{p}{(}\PY{n}{df\PYZus{}fitted\PYZus{}mnl}\PY{p}{)}
\end{Verbatim}
\end{tcolorbox}

    A data.frame: 6 × 2
\begin{tabular}{r|ll}
  & manual & structts\\
  & <dbl> & <dbl>\\
\hline
	995 & -12.96527 & -12.96527\\
	996 & -13.12028 & -13.12028\\
	997 & -13.11330 & -13.11330\\
	998 & -13.16258 & -13.16258\\
	999 & -13.36646 & -13.36646\\
	1000 & -13.17953 & -13.17953\\
\end{tabular}


    
    \begin{tcolorbox}[breakable, size=fbox, boxrule=1pt, pad at break*=1mm,colback=cellbackground, colframe=cellborder]
\prompt{In}{incolor}{181}{\boxspacing}
\begin{Verbatim}[commandchars=\\\{\}]
\PY{n+nf}{head}\PY{p}{(}\PY{n}{df\PYZus{}fitted\PYZus{}mnl}\PY{p}{)}
\end{Verbatim}
\end{tcolorbox}

    A data.frame: 6 × 2
\begin{tabular}{r|ll}
  & manual & structts\\
  & <dbl> & <dbl>\\
\hline
	1 & -0.6264538 & -0.6264538\\
	2 &  0.3378710 &  0.3378710\\
	3 & -0.1941920 & -0.1941920\\
	4 &  0.8891581 &  0.8891581\\
	5 &  0.7917861 &  0.7917861\\
	6 & -0.4389378 & -0.4389378\\
\end{tabular}


    
    \subsection{MTL}\label{mtl}

    Definindo os parâmetros:

    \begin{tcolorbox}[breakable, size=fbox, boxrule=1pt, pad at break*=1mm,colback=cellbackground, colframe=cellborder]
\prompt{In}{incolor}{200}{\boxspacing}
\begin{Verbatim}[commandchars=\\\{\}]
\PY{n+nb+bp}{T} \PY{o}{\PYZlt{}\PYZhy{}} \PY{l+m}{100}
\PY{n}{sigma2\PYZus{}epsilon} \PY{o}{\PYZlt{}\PYZhy{}} \PY{l+m}{1}
\PY{n}{sigma2\PYZus{}eta} \PY{o}{\PYZlt{}\PYZhy{}} \PY{l+m}{1}   
\PY{n}{sigma2\PYZus{}qsi} \PY{o}{\PYZlt{}\PYZhy{}} \PY{l+m}{0.1}
\PY{n}{n\PYZus{}seed} \PY{o}{\PYZlt{}\PYZhy{}} \PY{l+m}{42}
\end{Verbatim}
\end{tcolorbox}

    Simulando a série:

    \begin{tcolorbox}[breakable, size=fbox, boxrule=1pt, pad at break*=1mm,colback=cellbackground, colframe=cellborder]
\prompt{In}{incolor}{201}{\boxspacing}
\begin{Verbatim}[commandchars=\\\{\}]
\PY{n}{y\PYZus{}mtl} \PY{o}{\PYZlt{}\PYZhy{}} \PY{n+nf}{simul\PYZus{}y\PYZus{}mtl}\PY{p}{(}\PY{n+nb+bp}{T}\PY{p}{,} \PY{n}{sigma2\PYZus{}eta}\PY{o}{=}\PY{n}{sigma2\PYZus{}eta}\PY{p}{,} \PY{n}{sigma2\PYZus{}qsi}\PY{o}{=}\PY{n}{sigma2\PYZus{}qsi}\PY{p}{,} \PY{n}{sigma2\PYZus{}epsilon}\PY{o}{=}\PY{n}{sigma2\PYZus{}epsilon}\PY{p}{,} \PY{n}{n\PYZus{}seed}\PY{o}{=}\PY{n}{n\PYZus{}seed}\PY{p}{)}
\end{Verbatim}
\end{tcolorbox}

    estimando os hyperparâmtros usando MLE. Novo argumento: `Trend'

    \begin{tcolorbox}[breakable, size=fbox, boxrule=1pt, pad at break*=1mm,colback=cellbackground, colframe=cellborder]
\prompt{In}{incolor}{202}{\boxspacing}
\begin{Verbatim}[commandchars=\\\{\}]
\PY{n}{fit\PYZus{}mtl} \PY{o}{\PYZlt{}\PYZhy{}} \PY{n+nf}{StructTS}\PY{p}{(}\PY{n}{y\PYZus{}mtl}\PY{p}{,} \PY{l+s}{\PYZdq{}}\PY{l+s}{trend\PYZdq{}}\PY{p}{)}
\end{Verbatim}
\end{tcolorbox}

    Montando um dataframe com a séries:

    \begin{tcolorbox}[breakable, size=fbox, boxrule=1pt, pad at break*=1mm,colback=cellbackground, colframe=cellborder]
\prompt{In}{incolor}{203}{\boxspacing}
\begin{Verbatim}[commandchars=\\\{\}]
\PY{n}{df\PYZus{}mtl} \PY{o}{\PYZlt{}\PYZhy{}} \PY{n+nf}{data.frame}\PY{p}{(}
  \PY{n}{time} \PY{o}{=} \PY{l+m}{1}\PY{o}{:}\PY{n+nb+bp}{T}\PY{p}{,}
  \PY{n}{observed} \PY{o}{=} \PY{n}{y\PYZus{}mtl}\PY{p}{,}
  \PY{n}{adjusted\PYZus{}level} \PY{o}{=} \PY{n}{fit\PYZus{}mtl}\PY{o}{\PYZdl{}}\PY{n}{fitted}\PY{p}{[}\PY{p}{,} \PY{l+s}{\PYZdq{}}\PY{l+s}{level\PYZdq{}}\PY{p}{]}\PY{p}{,}
  \PY{n}{adjusted\PYZus{}trend} \PY{o}{=} \PY{n}{fit\PYZus{}mtl}\PY{o}{\PYZdl{}}\PY{n}{fitted}\PY{p}{[}\PY{p}{,} \PY{l+s}{\PYZdq{}}\PY{l+s}{slope\PYZdq{}}\PY{p}{]}
\PY{p}{)}
\end{Verbatim}
\end{tcolorbox}

    Fazendo os gráficos usando ggplot2:

    \begin{tcolorbox}[breakable, size=fbox, boxrule=1pt, pad at break*=1mm,colback=cellbackground, colframe=cellborder]
\prompt{In}{incolor}{226}{\boxspacing}
\begin{Verbatim}[commandchars=\\\{\}]
\PY{n+nf}{ggplot}\PY{p}{(}\PY{p}{)} \PY{o}{+}
  \PY{n+nf}{geom\PYZus{}line}\PY{p}{(}\PY{n}{data}\PY{o}{=}\PY{n}{df\PYZus{}mtl}\PY{p}{,}\PY{n+nf}{aes}\PY{p}{(}\PY{n}{x} \PY{o}{=} \PY{n}{time}\PY{p}{,}\PY{n}{y} \PY{o}{=} \PY{n}{observed}\PY{p}{,} \PY{n}{color} \PY{o}{=} \PY{l+s}{\PYZdq{}}\PY{l+s}{Observed\PYZdq{}}\PY{p}{)}\PY{p}{,} \PY{n}{size} \PY{o}{=} \PY{l+m}{1}\PY{p}{)} \PY{o}{+}
  \PY{n+nf}{geom\PYZus{}line}\PY{p}{(}\PY{n}{data}\PY{o}{=}\PY{n}{df\PYZus{}mtl}\PY{p}{,}\PY{n+nf}{aes}\PY{p}{(}\PY{n}{x} \PY{o}{=} \PY{n}{time}\PY{p}{,}\PY{n}{y} \PY{o}{=} \PY{n}{adjusted\PYZus{}level}\PY{p}{,} \PY{n}{color} \PY{o}{=} \PY{l+s}{\PYZdq{}}\PY{l+s}{Adjusted Level\PYZdq{}}\PY{p}{)}\PY{p}{,} \PY{n}{size} \PY{o}{=} \PY{l+m}{1}\PY{p}{,}\PY{n}{alpha}\PY{o}{=}\PY{l+m}{0.6}\PY{p}{)} \PY{o}{+}
  \PY{n+nf}{scale\PYZus{}color\PYZus{}manual}\PY{p}{(}\PY{n}{values} \PY{o}{=} \PY{n+nf}{c}\PY{p}{(}\PY{l+s}{\PYZdq{}}\PY{l+s}{Observed\PYZdq{}} \PY{o}{=} \PY{l+s}{\PYZdq{}}\PY{l+s}{black\PYZdq{}}\PY{p}{,} \PY{l+s}{\PYZdq{}}\PY{l+s}{Adjusted Level\PYZdq{}} \PY{o}{=} \PY{l+s}{\PYZdq{}}\PY{l+s}{blue\PYZdq{}}\PY{p}{)}\PY{p}{)} \PY{o}{+}
  \PY{n+nf}{labs}\PY{p}{(}
    \PY{n}{title} \PY{o}{=} \PY{l+s}{\PYZdq{}}\PY{l+s}{Observed vs Adjusted Level StructTS (MTL)\PYZdq{}}\PY{p}{,}
    \PY{n}{x} \PY{o}{=} \PY{l+s}{\PYZdq{}}\PY{l+s}{Time\PYZdq{}}\PY{p}{,}
    \PY{n}{y} \PY{o}{=} \PY{l+s}{\PYZdq{}}\PY{l+s}{Value\PYZdq{}}\PY{p}{,}
    \PY{n}{color} \PY{o}{=} \PY{l+s}{\PYZdq{}}\PY{l+s}{Legend\PYZdq{}}
  \PY{p}{)} \PY{o}{+}
  \PY{n+nf}{theme\PYZus{}minimal}\PY{p}{(}\PY{p}{)}\PY{o}{+}
\end{Verbatim}
\end{tcolorbox}

    \begin{Verbatim}[commandchars=\\\{\}, frame=single, framerule=2mm, rulecolor=\color{outerrorbackground}]
Error in parse(text = input): <text>:12:0: unexpected end of input
10:   ) +
11:   theme\_minimal()+
   \^{}
Traceback:

    \end{Verbatim}

    \begin{tcolorbox}[breakable, size=fbox, boxrule=1pt, pad at break*=1mm,colback=cellbackground, colframe=cellborder]
\prompt{In}{incolor}{230}{\boxspacing}
\begin{Verbatim}[commandchars=\\\{\}]
\PY{n+nf}{ggplot}\PY{p}{(}\PY{p}{)} \PY{o}{+}
  \PY{n+nf}{geom\PYZus{}line}\PY{p}{(}\PY{n}{data}\PY{o}{=}\PY{n}{df\PYZus{}mtl}\PY{p}{,}\PY{n+nf}{aes}\PY{p}{(}\PY{n}{x} \PY{o}{=} \PY{n}{time}\PY{p}{,}\PY{n}{y} \PY{o}{=} \PY{n}{adjusted\PYZus{}trend}\PY{p}{,} \PY{n}{color} \PY{o}{=} \PY{l+s}{\PYZdq{}}\PY{l+s}{Adjusted Trend\PYZdq{}}\PY{p}{)}\PY{p}{,} \PY{n}{size} \PY{o}{=} \PY{l+m}{1}\PY{p}{)} \PY{o}{+}
  \PY{n+nf}{scale\PYZus{}color\PYZus{}manual}\PY{p}{(}\PY{n}{values} \PY{o}{=} \PY{n+nf}{c}\PY{p}{(}\PY{l+s}{\PYZdq{}}\PY{l+s}{Adjusted Trend\PYZdq{}} \PY{o}{=} \PY{l+s}{\PYZdq{}}\PY{l+s}{black\PYZdq{}}\PY{p}{)}\PY{p}{)} \PY{o}{+}
  \PY{n+nf}{labs}\PY{p}{(}
    \PY{n}{title} \PY{o}{=} \PY{l+s}{\PYZdq{}}\PY{l+s}{Trend using StructTS (MTL)\PYZdq{}}\PY{p}{,}
    \PY{n}{x} \PY{o}{=} \PY{l+s}{\PYZdq{}}\PY{l+s}{Time\PYZdq{}}\PY{p}{,}
    \PY{n}{y} \PY{o}{=} \PY{l+s}{\PYZdq{}}\PY{l+s}{Value\PYZdq{}}\PY{p}{,}
    \PY{n}{color} \PY{o}{=} \PY{l+s}{\PYZdq{}}\PY{l+s}{Legend\PYZdq{}}
  \PY{p}{)} \PY{o}{+}
  \PY{n+nf}{theme\PYZus{}minimal}\PY{p}{(}\PY{p}{)}\PY{o}{+}

\PY{n+nf}{theme}\PY{p}{(}
    \PY{n}{panel.grid.major} \PY{o}{=} \PY{n+nf}{element\PYZus{}line}\PY{p}{(}\PY{n}{color} \PY{o}{=} \PY{l+s}{\PYZdq{}}\PY{l+s}{gray\PYZdq{}}\PY{p}{,} \PY{n}{linewidth} \PY{o}{=} \PY{l+m}{1}\PY{p}{)}\PY{p}{,}
    \PY{n}{panel.grid.minor} \PY{o}{=} \PY{n+nf}{element\PYZus{}line}\PY{p}{(}\PY{n}{color} \PY{o}{=} \PY{l+s}{\PYZdq{}}\PY{l+s}{lightgray\PYZdq{}}\PY{p}{,} \PY{n}{linewidth} \PY{o}{=} \PY{l+m}{0.5}\PY{p}{)}\PY{p}{,}
    \PY{n}{axis.line} \PY{o}{=} \PY{n+nf}{element\PYZus{}line}\PY{p}{(}\PY{n}{color} \PY{o}{=} \PY{l+s}{\PYZdq{}}\PY{l+s}{black\PYZdq{}}\PY{p}{,} \PY{n}{linewidth} \PY{o}{=} \PY{l+m}{1}\PY{p}{)}\PY{p}{,}  \PY{c+c1}{\PYZsh{} Make axis lines thicker and black}
    \PY{n}{axis.ticks} \PY{o}{=} \PY{n+nf}{element\PYZus{}line}\PY{p}{(}\PY{n}{color} \PY{o}{=} \PY{l+s}{\PYZdq{}}\PY{l+s}{black\PYZdq{}}\PY{p}{,} \PY{n}{linewidth} \PY{o}{=} \PY{l+m}{0.8}\PY{p}{)}\PY{p}{,}  \PY{c+c1}{\PYZsh{} Make ticks more visible}
    \PY{n}{axis.text} \PY{o}{=} \PY{n+nf}{element\PYZus{}text}\PY{p}{(}\PY{n}{size} \PY{o}{=} \PY{l+m}{12}\PY{p}{,} \PY{n}{color} \PY{o}{=} \PY{l+s}{\PYZdq{}}\PY{l+s}{black\PYZdq{}}\PY{p}{)}\PY{p}{,}  \PY{c+c1}{\PYZsh{} Adjust axis text size and color}
    \PY{n}{axis.title} \PY{o}{=} \PY{n+nf}{element\PYZus{}text}\PY{p}{(}\PY{n}{size} \PY{o}{=} \PY{l+m}{14}\PY{p}{,} \PY{n}{color} \PY{o}{=} \PY{l+s}{\PYZdq{}}\PY{l+s}{black\PYZdq{}}\PY{p}{)}  \PY{c+c1}{\PYZsh{} Adjust axis title size and color}
    \PY{p}{)}
\end{Verbatim}
\end{tcolorbox}

    \begin{Verbatim}[commandchars=\\\{\}]
Don't know how to automatically pick scale for object of type
\textcolor{ansi-blue}{<ts>}. Defaulting to continuous.
    \end{Verbatim}

    \begin{center}
    \adjustimage{max size={0.9\linewidth}{0.9\paperheight}}{T2_FK_files/T2_FK_45_1.png}
    \end{center}
    { \hspace*{\fill} \\}
    
    Valores dos parâmetros:

    \begin{tcolorbox}[breakable, size=fbox, boxrule=1pt, pad at break*=1mm,colback=cellbackground, colframe=cellborder]
\prompt{In}{incolor}{212}{\boxspacing}
\begin{Verbatim}[commandchars=\\\{\}]
\PY{n+nf}{cat}\PY{p}{(}\PY{l+s}{\PYZdq{}}\PY{l+s}{Estimated Observation Noise Variance (sigma\PYZca{}2\PYZus{}epsilon):\PYZdq{}}\PY{p}{,} \PY{n}{fit\PYZus{}mtl}\PY{o}{\PYZdl{}}\PY{n}{coef}\PY{p}{[}\PY{l+s}{\PYZdq{}}\PY{l+s}{epsilon\PYZdq{}}\PY{p}{]}\PY{p}{,} \PY{l+s}{\PYZdq{}}\PY{l+s}{\PYZbs{}n\PYZdq{}}\PY{p}{)}
\PY{n+nf}{cat}\PY{p}{(}\PY{l+s}{\PYZdq{}}\PY{l+s}{Estimated State Noise Variance (sigma\PYZca{}2\PYZus{}eta):\PYZdq{}}\PY{p}{,} \PY{n}{fit\PYZus{}mtl}\PY{o}{\PYZdl{}}\PY{n}{coef}\PY{p}{[}\PY{l+s}{\PYZdq{}}\PY{l+s}{level\PYZdq{}}\PY{p}{]}\PY{p}{,} \PY{l+s}{\PYZdq{}}\PY{l+s}{\PYZbs{}n\PYZdq{}}\PY{p}{)}
\PY{n+nf}{cat}\PY{p}{(}\PY{l+s}{\PYZdq{}}\PY{l+s}{Estimated State Noise Variance (sigma\PYZca{}2\PYZus{}qsi):\PYZdq{}}\PY{p}{,} \PY{n}{fit\PYZus{}mtl}\PY{o}{\PYZdl{}}\PY{n}{coef}\PY{p}{[}\PY{l+s}{\PYZdq{}}\PY{l+s}{slope\PYZdq{}}\PY{p}{]}\PY{p}{,} \PY{l+s}{\PYZdq{}}\PY{l+s}{\PYZbs{}n\PYZdq{}}\PY{p}{)}
\end{Verbatim}
\end{tcolorbox}

    \begin{Verbatim}[commandchars=\\\{\}]
Estimated Observation Noise Variance (sigma\^{}2\_epsilon): 0.9959496
Estimated State Noise Variance (sigma\^{}2\_eta): 1.17139
Estimated State Noise Variance (sigma\^{}2\_qsi): 0.006570937
    \end{Verbatim}

    Agora vamos fazer previsões e suavização:

    \begin{tcolorbox}[breakable, size=fbox, boxrule=1pt, pad at break*=1mm,colback=cellbackground, colframe=cellborder]
\prompt{In}{incolor}{213}{\boxspacing}
\begin{Verbatim}[commandchars=\\\{\}]
\PY{n}{y\PYZus{}mtl} \PY{o}{\PYZlt{}\PYZhy{}} \PY{n+nf}{simul\PYZus{}y\PYZus{}mtl}\PY{p}{(}\PY{n+nb+bp}{T}\PY{o}{=}\PY{l+m}{100}\PY{p}{,}\PY{l+m}{10}\PY{p}{,}\PY{l+m}{0.1}\PY{p}{,}\PY{l+m}{1}\PY{p}{,}\PY{n}{n\PYZus{}seed}\PY{o}{=}\PY{l+m}{1}\PY{p}{)}
\end{Verbatim}
\end{tcolorbox}

    Ajustando o modelo com StructTS

    \begin{tcolorbox}[breakable, size=fbox, boxrule=1pt, pad at break*=1mm,colback=cellbackground, colframe=cellborder]
\prompt{In}{incolor}{214}{\boxspacing}
\begin{Verbatim}[commandchars=\\\{\}]
\PY{n}{fit} \PY{o}{\PYZlt{}\PYZhy{}} \PY{n+nf}{StructTS}\PY{p}{(}\PY{n}{y\PYZus{}mtl}\PY{p}{,} \PY{n}{type} \PY{o}{=} \PY{l+s}{\PYZdq{}}\PY{l+s}{trend\PYZdq{}}\PY{p}{)}
\end{Verbatim}
\end{tcolorbox}

    definindo o número de passos a frente:

    \begin{tcolorbox}[breakable, size=fbox, boxrule=1pt, pad at break*=1mm,colback=cellbackground, colframe=cellborder]
\prompt{In}{incolor}{216}{\boxspacing}
\begin{Verbatim}[commandchars=\\\{\}]
\PY{n}{h} \PY{o}{\PYZlt{}\PYZhy{}} \PY{l+m}{20}
\end{Verbatim}
\end{tcolorbox}

    Criando o objeto com as previsões:

    \begin{tcolorbox}[breakable, size=fbox, boxrule=1pt, pad at break*=1mm,colback=cellbackground, colframe=cellborder]
\prompt{In}{incolor}{221}{\boxspacing}
\begin{Verbatim}[commandchars=\\\{\}]
\PY{n}{forecast\PYZus{}obj} \PY{o}{\PYZlt{}\PYZhy{}} \PY{n+nf}{forecast}\PY{p}{(}\PY{n}{fit}\PY{p}{,} \PY{n}{h} \PY{o}{=} \PY{n}{h}\PY{p}{,} \PY{n}{level} \PY{o}{=} \PY{l+m}{90}\PY{p}{)}
\end{Verbatim}
\end{tcolorbox}

    Fazendo a suavização

    \begin{tcolorbox}[breakable, size=fbox, boxrule=1pt, pad at break*=1mm,colback=cellbackground, colframe=cellborder]
\prompt{In}{incolor}{222}{\boxspacing}
\begin{Verbatim}[commandchars=\\\{\}]
\PY{n}{smoothed} \PY{o}{\PYZlt{}\PYZhy{}} \PY{n+nf}{tsSmooth}\PY{p}{(}\PY{n}{fit}\PY{p}{)}
\end{Verbatim}
\end{tcolorbox}

    Criando dataframe para o ggplot

    \begin{tcolorbox}[breakable, size=fbox, boxrule=1pt, pad at break*=1mm,colback=cellbackground, colframe=cellborder]
\prompt{In}{incolor}{223}{\boxspacing}
\begin{Verbatim}[commandchars=\\\{\}]
\PY{n}{df} \PY{o}{\PYZlt{}\PYZhy{}} \PY{n+nf}{data.frame}\PY{p}{(}
  \PY{n}{Tempo} \PY{o}{=} \PY{l+m}{1}\PY{o}{:}\PY{n+nb+bp}{T}\PY{p}{,}
  \PY{n}{Observado} \PY{o}{=} \PY{n}{y\PYZus{}mtl}\PY{p}{,}
  \PY{n}{Suavizado} \PY{o}{=} \PY{n}{smoothed}\PY{p}{[}\PY{p}{,} \PY{l+m}{1}\PY{p}{]}
\PY{p}{)}
\end{Verbatim}
\end{tcolorbox}

    \begin{tcolorbox}[breakable, size=fbox, boxrule=1pt, pad at break*=1mm,colback=cellbackground, colframe=cellborder]
\prompt{In}{incolor}{224}{\boxspacing}
\begin{Verbatim}[commandchars=\\\{\}]
\PY{n}{df\PYZus{}pred} \PY{o}{\PYZlt{}\PYZhy{}} \PY{n+nf}{data.frame}\PY{p}{(}
  \PY{n}{Tempo} \PY{o}{=} \PY{p}{(}\PY{n+nb+bp}{T} \PY{o}{+} \PY{l+m}{1}\PY{p}{)}\PY{o}{:}\PY{p}{(}\PY{n+nb+bp}{T} \PY{o}{+} \PY{n}{h}\PY{p}{)}\PY{p}{,}
  \PY{n}{Previsao} \PY{o}{=} \PY{n}{forecast\PYZus{}obj}\PY{o}{\PYZdl{}}\PY{n}{mean}\PY{p}{,}
  \PY{n}{Lower} \PY{o}{=} \PY{n}{forecast\PYZus{}obj}\PY{o}{\PYZdl{}}\PY{n}{lower}\PY{p}{[}\PY{p}{,} \PY{l+m}{1}\PY{p}{]}\PY{p}{,}
  \PY{n}{Upper} \PY{o}{=} \PY{n}{forecast\PYZus{}obj}\PY{o}{\PYZdl{}}\PY{n}{upper}\PY{p}{[}\PY{p}{,} \PY{l+m}{1}\PY{p}{]}
\PY{p}{)}
\end{Verbatim}
\end{tcolorbox}

    \begin{tcolorbox}[breakable, size=fbox, boxrule=1pt, pad at break*=1mm,colback=cellbackground, colframe=cellborder]
\prompt{In}{incolor}{231}{\boxspacing}
\begin{Verbatim}[commandchars=\\\{\}]
\PY{n+nf}{ggplot}\PY{p}{(}\PY{p}{)} \PY{o}{+}
  \PY{n+nf}{geom\PYZus{}line}\PY{p}{(}\PY{n}{data} \PY{o}{=} \PY{n}{df}\PY{p}{,} \PY{n+nf}{aes}\PY{p}{(}\PY{n}{x} \PY{o}{=} \PY{n}{Tempo}\PY{p}{,} \PY{n}{y} \PY{o}{=} \PY{n}{Observado}\PY{p}{)}\PY{p}{,} \PY{n}{color} \PY{o}{=} \PY{l+s}{\PYZdq{}}\PY{l+s}{black\PYZdq{}}\PY{p}{,} \PY{n}{size} \PY{o}{=} \PY{l+m}{1}\PY{p}{)} \PY{o}{+}
  \PY{n+nf}{geom\PYZus{}line}\PY{p}{(}\PY{n}{data} \PY{o}{=} \PY{n}{df}\PY{p}{,} \PY{n+nf}{aes}\PY{p}{(}\PY{n}{x} \PY{o}{=} \PY{n}{Tempo}\PY{p}{,} \PY{n}{y} \PY{o}{=} \PY{n}{Suavizado}\PY{p}{)}\PY{p}{,} \PY{n}{color} \PY{o}{=} \PY{l+s}{\PYZdq{}}\PY{l+s}{blue\PYZdq{}}\PY{p}{,} \PY{n}{size} \PY{o}{=} \PY{l+m}{1}\PY{p}{)} \PY{o}{+}
  \PY{n+nf}{geom\PYZus{}line}\PY{p}{(}\PY{n}{data} \PY{o}{=} \PY{n}{df\PYZus{}pred}\PY{p}{,} \PY{n+nf}{aes}\PY{p}{(}\PY{n}{x} \PY{o}{=} \PY{n}{Tempo}\PY{p}{,} \PY{n}{y} \PY{o}{=} \PY{n}{Previsao}\PY{p}{)}\PY{p}{,} \PY{n}{color} \PY{o}{=} \PY{l+s}{\PYZdq{}}\PY{l+s}{red\PYZdq{}}\PY{p}{,} \PY{n}{size} \PY{o}{=} \PY{l+m}{1}\PY{p}{)} \PY{o}{+}
  \PY{n+nf}{geom\PYZus{}ribbon}\PY{p}{(}\PY{n}{data} \PY{o}{=} \PY{n}{df\PYZus{}pred}\PY{p}{,} \PY{n+nf}{aes}\PY{p}{(}\PY{n}{x} \PY{o}{=} \PY{n}{Tempo}\PY{p}{,} \PY{n}{ymin} \PY{o}{=} \PY{n}{Lower}\PY{p}{,} \PY{n}{ymax} \PY{o}{=} \PY{n}{Upper}\PY{p}{)}\PY{p}{,} \PY{n}{fill} \PY{o}{=} \PY{l+s}{\PYZdq{}}\PY{l+s}{red\PYZdq{}}\PY{p}{,} \PY{n}{alpha} \PY{o}{=} \PY{l+m}{0.2}\PY{p}{)} \PY{o}{+}
  \PY{n+nf}{labs}\PY{p}{(}\PY{n}{title} \PY{o}{=} \PY{l+s}{\PYZdq{}}\PY{l+s}{Modelo de Nível Local: Observado, Suavizado e Previsão\PYZdq{}}\PY{p}{,}
       \PY{n}{x} \PY{o}{=} \PY{l+s}{\PYZdq{}}\PY{l+s}{Tempo\PYZdq{}}\PY{p}{,} \PY{n}{y} \PY{o}{=} \PY{l+s}{\PYZdq{}}\PY{l+s}{Valor\PYZdq{}}\PY{p}{)} \PY{o}{+}
  \PY{n+nf}{theme\PYZus{}minimal}\PY{p}{(}\PY{p}{)}\PY{o}{+}

\PY{n+nf}{theme}\PY{p}{(}
    \PY{n}{panel.grid.major} \PY{o}{=} \PY{n+nf}{element\PYZus{}line}\PY{p}{(}\PY{n}{color} \PY{o}{=} \PY{l+s}{\PYZdq{}}\PY{l+s}{gray\PYZdq{}}\PY{p}{,} \PY{n}{linewidth} \PY{o}{=} \PY{l+m}{1}\PY{p}{)}\PY{p}{,}
    \PY{n}{panel.grid.minor} \PY{o}{=} \PY{n+nf}{element\PYZus{}line}\PY{p}{(}\PY{n}{color} \PY{o}{=} \PY{l+s}{\PYZdq{}}\PY{l+s}{lightgray\PYZdq{}}\PY{p}{,} \PY{n}{linewidth} \PY{o}{=} \PY{l+m}{0.5}\PY{p}{)}\PY{p}{,}
    \PY{n}{axis.line} \PY{o}{=} \PY{n+nf}{element\PYZus{}line}\PY{p}{(}\PY{n}{color} \PY{o}{=} \PY{l+s}{\PYZdq{}}\PY{l+s}{black\PYZdq{}}\PY{p}{,} \PY{n}{linewidth} \PY{o}{=} \PY{l+m}{1}\PY{p}{)}\PY{p}{,}  \PY{c+c1}{\PYZsh{} Make axis lines thicker and black}
    \PY{n}{axis.ticks} \PY{o}{=} \PY{n+nf}{element\PYZus{}line}\PY{p}{(}\PY{n}{color} \PY{o}{=} \PY{l+s}{\PYZdq{}}\PY{l+s}{black\PYZdq{}}\PY{p}{,} \PY{n}{linewidth} \PY{o}{=} \PY{l+m}{0.8}\PY{p}{)}\PY{p}{,}  \PY{c+c1}{\PYZsh{} Make ticks more visible}
    \PY{n}{axis.text} \PY{o}{=} \PY{n+nf}{element\PYZus{}text}\PY{p}{(}\PY{n}{size} \PY{o}{=} \PY{l+m}{12}\PY{p}{,} \PY{n}{color} \PY{o}{=} \PY{l+s}{\PYZdq{}}\PY{l+s}{black\PYZdq{}}\PY{p}{)}\PY{p}{,}  \PY{c+c1}{\PYZsh{} Adjust axis text size and color}
    \PY{n}{axis.title} \PY{o}{=} \PY{n+nf}{element\PYZus{}text}\PY{p}{(}\PY{n}{size} \PY{o}{=} \PY{l+m}{14}\PY{p}{,} \PY{n}{color} \PY{o}{=} \PY{l+s}{\PYZdq{}}\PY{l+s}{black\PYZdq{}}\PY{p}{)}  \PY{c+c1}{\PYZsh{} Adjust axis title size and color}
    \PY{p}{)}
\end{Verbatim}
\end{tcolorbox}

    \begin{center}
    \adjustimage{max size={0.9\linewidth}{0.9\paperheight}}{T2_FK_files/T2_FK_61_0.png}
    \end{center}
    { \hspace*{\fill} \\}
    
    \section{DLM}\label{dlm}

    Como sempre fazemos :

    \begin{tcolorbox}[breakable, size=fbox, boxrule=1pt, pad at break*=1mm,colback=cellbackground, colframe=cellborder]
\prompt{In}{incolor}{243}{\boxspacing}
\begin{Verbatim}[commandchars=\\\{\}]
\PY{n+nb+bp}{T}\PY{o}{=}\PY{l+m}{100}
\end{Verbatim}
\end{tcolorbox}

    \begin{tcolorbox}[breakable, size=fbox, boxrule=1pt, pad at break*=1mm,colback=cellbackground, colframe=cellborder]
\prompt{In}{incolor}{244}{\boxspacing}
\begin{Verbatim}[commandchars=\\\{\}]
\PY{n}{y\PYZus{}mnl} \PY{o}{\PYZlt{}\PYZhy{}} \PY{n+nf}{simul\PYZus{}y\PYZus{}mnl}\PY{p}{(}\PY{n+nb+bp}{T}\PY{o}{=}\PY{n+nb+bp}{T}\PY{p}{,}\PY{l+m}{1}\PY{p}{,}\PY{l+m}{0.5}\PY{p}{,}\PY{l+m}{1}\PY{p}{)}
\end{Verbatim}
\end{tcolorbox}

    The function dlmModPoly in the dlm package in R creates a state-space
model for a polynomial trend process. It helps define local level (MNL)
and local trend (MTL) models by setting up the system matrices required
for the Kalman filter.

    \begin{tcolorbox}[breakable, size=fbox, boxrule=1pt, pad at break*=1mm,colback=cellbackground, colframe=cellborder]
\prompt{In}{incolor}{245}{\boxspacing}
\begin{Verbatim}[commandchars=\\\{\}]
\PY{n}{build\PYZus{}mnl} \PY{o}{\PYZlt{}\PYZhy{}} \PY{n+nf}{function}\PY{p}{(}\PY{n}{theta}\PY{p}{)} \PY{p}{\PYZob{}}
    \PY{n+nf}{dlmModPoly}\PY{p}{(}\PY{n}{order} \PY{o}{=} \PY{l+m}{1}\PY{p}{,} \PY{n}{dV} \PY{o}{=} \PY{n}{theta}\PY{p}{[}\PY{l+m}{1}\PY{p}{]}\PY{p}{,} \PY{n}{dW} \PY{o}{=} \PY{n}{theta}\PY{p}{[}\PY{l+m}{2}\PY{p}{]}\PY{p}{)}
\PY{p}{\PYZcb{}}
\end{Verbatim}
\end{tcolorbox}

    estimando o modelo via MLE

    \begin{tcolorbox}[breakable, size=fbox, boxrule=1pt, pad at break*=1mm,colback=cellbackground, colframe=cellborder]
\prompt{In}{incolor}{246}{\boxspacing}
\begin{Verbatim}[commandchars=\\\{\}]
\PY{n}{fit} \PY{o}{\PYZlt{}\PYZhy{}} \PY{n+nf}{dlmMLE}\PY{p}{(}\PY{n}{y\PYZus{}mnl}\PY{p}{,} \PY{n}{parm} \PY{o}{=} \PY{n+nf}{c}\PY{p}{(}\PY{l+m}{100}\PY{p}{,} \PY{l+m}{2}\PY{p}{)}\PY{p}{,} \PY{n}{build\PYZus{}mnl}\PY{p}{,} \PY{n}{lower} \PY{o}{=} \PY{n+nf}{rep}\PY{p}{(}\PY{l+m}{1e\PYZhy{}4}\PY{p}{,} \PY{l+m}{2}\PY{p}{)}\PY{p}{)}
\end{Verbatim}
\end{tcolorbox}

    parm:Initial values for the parameters to be estimated

lower: constraint in param values

    Recuperando os valores estimados dos hyperparâmetros:

    \begin{tcolorbox}[breakable, size=fbox, boxrule=1pt, pad at break*=1mm,colback=cellbackground, colframe=cellborder]
\prompt{In}{incolor}{247}{\boxspacing}
\begin{Verbatim}[commandchars=\\\{\}]
\PY{n}{mod\PYZus{}mnl} \PY{o}{\PYZlt{}\PYZhy{}} \PY{n+nf}{build\PYZus{}mnl}\PY{p}{(}\PY{n}{fit}\PY{o}{\PYZdl{}}\PY{n}{par}\PY{p}{)}
\PY{n+nf}{drop}\PY{p}{(}\PY{n+nf}{V}\PY{p}{(}\PY{n}{mod\PYZus{}mnl}\PY{p}{)}\PY{p}{)}
\PY{n+nf}{drop}\PY{p}{(}\PY{n+nf}{W}\PY{p}{(}\PY{n}{mod\PYZus{}mnl}\PY{p}{)}\PY{p}{)}
\end{Verbatim}
\end{tcolorbox}

    0.718626480364571

    
    0.436154015222723

    
    The inverse of the Hessian matrix of the negative loglikelihood function
evaluated at the MLEs is, by standard maximum likelihood theory, an
estimate of the asymptotic variance matrix of the maximum likelihood
estimators

    obtendo a hessiana avaliada no MLE

    \begin{tcolorbox}[breakable, size=fbox, boxrule=1pt, pad at break*=1mm,colback=cellbackground, colframe=cellborder]
\prompt{In}{incolor}{249}{\boxspacing}
\begin{Verbatim}[commandchars=\\\{\}]
\PY{n}{hs} \PY{o}{\PYZlt{}\PYZhy{}} \PY{n+nf}{hessian}\PY{p}{(}\PY{n+nf}{function}\PY{p}{(}\PY{n}{x}\PY{p}{)} \PY{n+nf}{dlmLL}\PY{p}{(}\PY{n}{y\PYZus{}mnl}\PY{p}{,} \PY{n+nf}{build\PYZus{}mnl}\PY{p}{(}\PY{n}{x}\PY{p}{)}\PY{p}{)}\PY{p}{,} \PY{n}{fit}\PY{o}{\PYZdl{}}\PY{n}{par}\PY{p}{)}
\end{Verbatim}
\end{tcolorbox}

    Checando se é positiva definida

    \begin{tcolorbox}[breakable, size=fbox, boxrule=1pt, pad at break*=1mm,colback=cellbackground, colframe=cellborder]
\prompt{In}{incolor}{250}{\boxspacing}
\begin{Verbatim}[commandchars=\\\{\}]
\PY{n+nf}{all}\PY{p}{(}\PY{n+nf}{eigen}\PY{p}{(}\PY{n}{hs}\PY{p}{,} \PY{n}{only.values} \PY{o}{=} \PY{k+kc}{TRUE}\PY{p}{)}\PY{o}{\PYZdl{}}\PY{n}{values} \PY{o}{\PYZgt{}} \PY{l+m}{0}\PY{p}{)}
\end{Verbatim}
\end{tcolorbox}

    TRUE

    
    Calculando o Inverso da Hessiana

    \begin{tcolorbox}[breakable, size=fbox, boxrule=1pt, pad at break*=1mm,colback=cellbackground, colframe=cellborder]
\prompt{In}{incolor}{251}{\boxspacing}
\begin{Verbatim}[commandchars=\\\{\}]
\PY{n}{aVar} \PY{o}{\PYZlt{}\PYZhy{}} \PY{n+nf}{solve}\PY{p}{(}\PY{n}{hs}\PY{p}{)}
\end{Verbatim}
\end{tcolorbox}

    \begin{tcolorbox}[breakable, size=fbox, boxrule=1pt, pad at break*=1mm,colback=cellbackground, colframe=cellborder]
\prompt{In}{incolor}{252}{\boxspacing}
\begin{Verbatim}[commandchars=\\\{\}]
\PY{n+nf}{cat}\PY{p}{(}\PY{l+s}{\PYZdq{}}\PY{l+s}{Standard Error of dV:\PYZdq{}}\PY{p}{,} \PY{n+nf}{sqrt}\PY{p}{(}\PY{n+nf}{diag}\PY{p}{(}\PY{n}{aVar}\PY{p}{)}\PY{p}{[}\PY{l+m}{1}\PY{p}{]}\PY{p}{)}\PY{p}{,} \PY{l+s}{\PYZdq{}}\PY{l+s}{\PYZbs{}n\PYZdq{}}\PY{p}{)}
\PY{n+nf}{cat}\PY{p}{(}\PY{l+s}{\PYZdq{}}\PY{l+s}{Standard Error of dW:\PYZdq{}}\PY{p}{,} \PY{n+nf}{sqrt}\PY{p}{(}\PY{n+nf}{diag}\PY{p}{(}\PY{n}{aVar}\PY{p}{)}\PY{p}{[}\PY{l+m}{2}\PY{p}{]}\PY{p}{)}\PY{p}{,} \PY{l+s}{\PYZdq{}}\PY{l+s}{\PYZbs{}n\PYZdq{}}\PY{p}{)}
\end{Verbatim}
\end{tcolorbox}

    \begin{Verbatim}[commandchars=\\\{\}]
Standard Error of dV: 0.1579155
Standard Error of dW: 0.1435128
    \end{Verbatim}

    Aplicando o filtro e suavização:

    \begin{tcolorbox}[breakable, size=fbox, boxrule=1pt, pad at break*=1mm,colback=cellbackground, colframe=cellborder]
\prompt{In}{incolor}{256}{\boxspacing}
\begin{Verbatim}[commandchars=\\\{\}]
\PY{n}{smooth\PYZus{}mnl} \PY{o}{\PYZlt{}\PYZhy{}} \PY{n+nf}{dlmSmooth}\PY{p}{(}\PY{n}{y\PYZus{}mnl}\PY{p}{,} \PY{n}{mod\PYZus{}mnl}\PY{p}{)}
\PY{n}{filter\PYZus{}mnl} \PY{o}{\PYZlt{}\PYZhy{}} \PY{n+nf}{dlmFilter}\PY{p}{(}\PY{n}{y\PYZus{}mnl}\PY{p}{,}\PY{n}{mod\PYZus{}mnl}\PY{p}{)}
\end{Verbatim}
\end{tcolorbox}

    Extração dos estados suavizados

    \begin{tcolorbox}[breakable, size=fbox, boxrule=1pt, pad at break*=1mm,colback=cellbackground, colframe=cellborder]
\prompt{In}{incolor}{258}{\boxspacing}
\begin{Verbatim}[commandchars=\\\{\}]
\PY{n}{mu\PYZus{}hat} \PY{o}{\PYZlt{}\PYZhy{}} \PY{n+nf}{drop}\PY{p}{(}\PY{n}{smooth\PYZus{}mnl}\PY{o}{\PYZdl{}}\PY{n}{s}\PY{p}{)}
\end{Verbatim}
\end{tcolorbox}

    Gerando as previsões com dlmForecast

    \begin{tcolorbox}[breakable, size=fbox, boxrule=1pt, pad at break*=1mm,colback=cellbackground, colframe=cellborder]
\prompt{In}{incolor}{259}{\boxspacing}
\begin{Verbatim}[commandchars=\\\{\}]
\PY{n}{fore\PYZus{}mnl} \PY{o}{\PYZlt{}\PYZhy{}} \PY{n+nf}{dlmForecast}\PY{p}{(}\PY{n}{filter\PYZus{}mnl}\PY{p}{,} \PY{n}{nAhead} \PY{o}{=} \PY{l+m}{10}\PY{p}{)}
\end{Verbatim}
\end{tcolorbox}

    matriz de valores esperados para futuras obs

    \begin{tcolorbox}[breakable, size=fbox, boxrule=1pt, pad at break*=1mm,colback=cellbackground, colframe=cellborder]
\prompt{In}{incolor}{260}{\boxspacing}
\begin{Verbatim}[commandchars=\\\{\}]
\PY{n}{f} \PY{o}{\PYZlt{}\PYZhy{}} \PY{n}{fore\PYZus{}mnl}\PY{o}{\PYZdl{}}\PY{n}{f}
\end{Verbatim}
\end{tcolorbox}

    lista de variâncias de futuras observações

    \begin{tcolorbox}[breakable, size=fbox, boxrule=1pt, pad at break*=1mm,colback=cellbackground, colframe=cellborder]
\prompt{In}{incolor}{261}{\boxspacing}
\begin{Verbatim}[commandchars=\\\{\}]
\PY{n}{Q} \PY{o}{\PYZlt{}\PYZhy{}} \PY{n}{fore\PYZus{}mnl}\PY{o}{\PYZdl{}}\PY{n}{Q}
\end{Verbatim}
\end{tcolorbox}

    Calculando o intervalo preditivo (banda de 50\%)

    \begin{tcolorbox}[breakable, size=fbox, boxrule=1pt, pad at break*=1mm,colback=cellbackground, colframe=cellborder]
\prompt{In}{incolor}{262}{\boxspacing}
\begin{Verbatim}[commandchars=\\\{\}]
\PY{n}{hwidth} \PY{o}{\PYZlt{}\PYZhy{}} \PY{n+nf}{qnorm}\PY{p}{(}\PY{l+m}{0.25}\PY{p}{,} \PY{n}{lower} \PY{o}{=} \PY{k+kc}{FALSE}\PY{p}{)} \PY{o}{*} \PY{n+nf}{sqrt}\PY{p}{(}\PY{n+nf}{unlist}\PY{p}{(}\PY{n}{Q}\PY{p}{)}\PY{p}{)}
\end{Verbatim}
\end{tcolorbox}

    Criando o intervalo de previsão

    \begin{tcolorbox}[breakable, size=fbox, boxrule=1pt, pad at break*=1mm,colback=cellbackground, colframe=cellborder]
\prompt{In}{incolor}{263}{\boxspacing}
\begin{Verbatim}[commandchars=\\\{\}]
\PY{n}{lower} \PY{o}{\PYZlt{}\PYZhy{}} \PY{n}{f} \PY{o}{\PYZhy{}} \PY{n}{hwidth}
\PY{n}{upper} \PY{o}{\PYZlt{}\PYZhy{}} \PY{n}{f} \PY{o}{+} \PY{n}{hwidth}
\end{Verbatim}
\end{tcolorbox}

    Criar uma data frame para facilitar o uso no ggplot2

    \begin{tcolorbox}[breakable, size=fbox, boxrule=1pt, pad at break*=1mm,colback=cellbackground, colframe=cellborder]
\prompt{In}{incolor}{264}{\boxspacing}
\begin{Verbatim}[commandchars=\\\{\}]
\PY{n}{df\PYZus{}forecast} \PY{o}{\PYZlt{}\PYZhy{}} \PY{n+nf}{data.frame}\PY{p}{(}
  \PY{n}{Time} \PY{o}{=} \PY{p}{(}\PY{n+nf}{length}\PY{p}{(}\PY{n}{y\PYZus{}mnl}\PY{p}{)} \PY{o}{+} \PY{l+m}{1}\PY{p}{)}\PY{o}{:}\PY{p}{(}\PY{n+nf}{length}\PY{p}{(}\PY{n}{y\PYZus{}mnl}\PY{p}{)} \PY{o}{+} \PY{l+m}{10}\PY{p}{)}\PY{p}{,}
  \PY{n}{Forecasted} \PY{o}{=} \PY{n}{f}\PY{p}{,}
  \PY{n}{Lower} \PY{o}{=} \PY{n}{lower}\PY{p}{,}
  \PY{n}{Upper} \PY{o}{=} \PY{n}{upper}
\PY{p}{)}
\end{Verbatim}
\end{tcolorbox}

    Criando o gráfico

    \begin{tcolorbox}[breakable, size=fbox, boxrule=1pt, pad at break*=1mm,colback=cellbackground, colframe=cellborder]
\prompt{In}{incolor}{267}{\boxspacing}
\begin{Verbatim}[commandchars=\\\{\}]
\PY{n+nf}{ggplot}\PY{p}{(}\PY{n}{df\PYZus{}forecast}\PY{p}{,} \PY{n+nf}{aes}\PY{p}{(}\PY{n}{x} \PY{o}{=} \PY{n}{Time}\PY{p}{)}\PY{p}{)} \PY{o}{+}
  \PY{c+c1}{\PYZsh{} Adiciona as previsões como uma linha vermelha}
  \PY{n+nf}{geom\PYZus{}line}\PY{p}{(}\PY{n+nf}{aes}\PY{p}{(}\PY{n}{y} \PY{o}{=} \PY{n}{Forecasted}\PY{p}{,} \PY{n}{color} \PY{o}{=} \PY{l+s}{\PYZdq{}}\PY{l+s}{Forecasted\PYZdq{}}\PY{p}{)}\PY{p}{,} \PY{n}{size} \PY{o}{=} \PY{l+m}{1}\PY{p}{)} \PY{o}{+}
  
  \PY{c+c1}{\PYZsh{} Adiciona as bandas de previsão (intervalo de 50\PYZpc{})}
  \PY{n+nf}{geom\PYZus{}ribbon}\PY{p}{(}\PY{n+nf}{aes}\PY{p}{(}\PY{n}{ymin} \PY{o}{=} \PY{n}{Lower}\PY{p}{,} \PY{n}{ymax} \PY{o}{=} \PY{n}{Upper}\PY{p}{)}\PY{p}{,} \PY{n}{fill} \PY{o}{=} \PY{l+s}{\PYZdq{}}\PY{l+s}{red\PYZdq{}}\PY{p}{,} \PY{n}{alpha} \PY{o}{=} \PY{l+m}{0.3}\PY{p}{)} \PY{o}{+}
  
  \PY{c+c1}{\PYZsh{} Adiciona as observações reais da série (y\PYZus{}mnl)}
  \PY{n+nf}{geom\PYZus{}line}\PY{p}{(}\PY{n}{data} \PY{o}{=} \PY{n+nf}{data.frame}\PY{p}{(}\PY{n}{Time} \PY{o}{=} \PY{l+m}{1}\PY{o}{:}\PY{n+nf}{length}\PY{p}{(}\PY{n}{y\PYZus{}mnl}\PY{p}{)}\PY{p}{,} \PY{n}{Observed} \PY{o}{=} \PY{n}{y\PYZus{}mnl}\PY{p}{)}\PY{p}{,}
            \PY{n+nf}{aes}\PY{p}{(}\PY{n}{y} \PY{o}{=} \PY{n}{Observed}\PY{p}{,} \PY{n}{color} \PY{o}{=} \PY{l+s}{\PYZdq{}}\PY{l+s}{Observed\PYZdq{}}\PY{p}{)}\PY{p}{,} \PY{n}{size} \PY{o}{=} \PY{l+m}{1}\PY{p}{,} \PY{n}{linetype} \PY{o}{=} \PY{l+s}{\PYZdq{}}\PY{l+s}{solid\PYZdq{}}\PY{p}{)} \PY{o}{+}
  
  \PY{c+c1}{\PYZsh{} Adiciona a linha de suavização (suavizado)}
  \PY{n+nf}{geom\PYZus{}line}\PY{p}{(}\PY{n}{data} \PY{o}{=} \PY{n+nf}{data.frame}\PY{p}{(}\PY{n}{Time} \PY{o}{=} \PY{n+nf}{time}\PY{p}{(}\PY{n}{smooth\PYZus{}mnl}\PY{o}{\PYZdl{}}\PY{n}{s}\PY{p}{)}\PY{p}{,} \PY{n}{Smoothed} \PY{o}{=} \PY{n}{smooth\PYZus{}mnl}\PY{o}{\PYZdl{}}\PY{n}{s}\PY{p}{)}\PY{p}{,}
            \PY{n+nf}{aes}\PY{p}{(}\PY{n}{y} \PY{o}{=} \PY{n}{Smoothed}\PY{p}{,} \PY{n}{color} \PY{o}{=} \PY{l+s}{\PYZdq{}}\PY{l+s}{Smoothed\PYZdq{}}\PY{p}{)}\PY{p}{,} \PY{n}{size} \PY{o}{=} \PY{l+m}{1}\PY{p}{,} \PY{n}{linetype} \PY{o}{=} \PY{l+s}{\PYZdq{}}\PY{l+s}{dashed\PYZdq{}}\PY{p}{)} \PY{o}{+}
  
  \PY{c+c1}{\PYZsh{} Personalizando os eixos e título}
  \PY{n+nf}{labs}\PY{p}{(}
    \PY{n}{title} \PY{o}{=} \PY{l+s}{\PYZdq{}}\PY{l+s}{Forecasted Values with Prediction Intervals\PYZdq{}}\PY{p}{,}
    \PY{n}{x} \PY{o}{=} \PY{l+s}{\PYZdq{}}\PY{l+s}{Time\PYZdq{}}\PY{p}{,}
    \PY{n}{y} \PY{o}{=} \PY{l+s}{\PYZdq{}}\PY{l+s}{Level\PYZdq{}}\PY{p}{,}
    \PY{n}{color} \PY{o}{=} \PY{l+s}{\PYZdq{}}\PY{l+s}{Legend\PYZdq{}}
  \PY{p}{)} \PY{o}{+}
  
  \PY{c+c1}{\PYZsh{} Customizando a paleta de cores}
  \PY{n+nf}{scale\PYZus{}color\PYZus{}manual}\PY{p}{(}\PY{n}{values} \PY{o}{=} \PY{n+nf}{c}\PY{p}{(}\PY{l+s}{\PYZdq{}}\PY{l+s}{Forecasted\PYZdq{}} \PY{o}{=} \PY{l+s}{\PYZdq{}}\PY{l+s}{red\PYZdq{}}\PY{p}{,} \PY{l+s}{\PYZdq{}}\PY{l+s}{Observed\PYZdq{}} \PY{o}{=} \PY{l+s}{\PYZdq{}}\PY{l+s}{black\PYZdq{}}\PY{p}{,} \PY{l+s}{\PYZdq{}}\PY{l+s}{Smoothed\PYZdq{}} \PY{o}{=} \PY{l+s}{\PYZdq{}}\PY{l+s}{blue\PYZdq{}}\PY{p}{)}\PY{p}{)} \PY{o}{+}
  
  \PY{c+c1}{\PYZsh{} Tema minimalista}
  \PY{n+nf}{theme\PYZus{}minimal}\PY{p}{(}\PY{p}{)} \PY{o}{+}
    \PY{n+nf}{theme}\PY{p}{(}
    \PY{n}{panel.grid.major} \PY{o}{=} \PY{n+nf}{element\PYZus{}line}\PY{p}{(}\PY{n}{color} \PY{o}{=} \PY{l+s}{\PYZdq{}}\PY{l+s}{gray\PYZdq{}}\PY{p}{,} \PY{n}{linewidth} \PY{o}{=} \PY{l+m}{1}\PY{p}{)}\PY{p}{,}
    \PY{n}{panel.grid.minor} \PY{o}{=} \PY{n+nf}{element\PYZus{}line}\PY{p}{(}\PY{n}{color} \PY{o}{=} \PY{l+s}{\PYZdq{}}\PY{l+s}{lightgray\PYZdq{}}\PY{p}{,} \PY{n}{linewidth} \PY{o}{=} \PY{l+m}{0.5}\PY{p}{)}\PY{p}{,}
    \PY{n}{axis.line} \PY{o}{=} \PY{n+nf}{element\PYZus{}line}\PY{p}{(}\PY{n}{color} \PY{o}{=} \PY{l+s}{\PYZdq{}}\PY{l+s}{black\PYZdq{}}\PY{p}{,} \PY{n}{linewidth} \PY{o}{=} \PY{l+m}{1}\PY{p}{)}\PY{p}{,}  \PY{c+c1}{\PYZsh{} Make axis lines thicker and black}
    \PY{n}{axis.ticks} \PY{o}{=} \PY{n+nf}{element\PYZus{}line}\PY{p}{(}\PY{n}{color} \PY{o}{=} \PY{l+s}{\PYZdq{}}\PY{l+s}{black\PYZdq{}}\PY{p}{,} \PY{n}{linewidth} \PY{o}{=} \PY{l+m}{0.8}\PY{p}{)}\PY{p}{,}  \PY{c+c1}{\PYZsh{} Make ticks more visible}
    \PY{n}{axis.text} \PY{o}{=} \PY{n+nf}{element\PYZus{}text}\PY{p}{(}\PY{n}{size} \PY{o}{=} \PY{l+m}{12}\PY{p}{,} \PY{n}{color} \PY{o}{=} \PY{l+s}{\PYZdq{}}\PY{l+s}{black\PYZdq{}}\PY{p}{)}\PY{p}{,}  \PY{c+c1}{\PYZsh{} Adjust axis text size and color}
    \PY{n}{axis.title} \PY{o}{=} \PY{n+nf}{element\PYZus{}text}\PY{p}{(}\PY{n}{size} \PY{o}{=} \PY{l+m}{14}\PY{p}{,} \PY{n}{color} \PY{o}{=} \PY{l+s}{\PYZdq{}}\PY{l+s}{black\PYZdq{}}\PY{p}{)}\PY{p}{,}  \PY{c+c1}{\PYZsh{} Adjust axis title size and color}
    \PY{n}{legend.position} \PY{o}{=} \PY{l+s}{\PYZdq{}}\PY{l+s}{top\PYZdq{}}\PY{p}{)}
\end{Verbatim}
\end{tcolorbox}

    \begin{center}
    \adjustimage{max size={0.9\linewidth}{0.9\paperheight}}{T2_FK_files/T2_FK_98_0.png}
    \end{center}
    { \hspace*{\fill} \\}
    
    Análise dos resíduos

    \begin{tcolorbox}[breakable, size=fbox, boxrule=1pt, pad at break*=1mm,colback=cellbackground, colframe=cellborder]
\prompt{In}{incolor}{268}{\boxspacing}
\begin{Verbatim}[commandchars=\\\{\}]
\PY{n}{df\PYZus{}residual} \PY{o}{\PYZlt{}\PYZhy{}} \PY{n+nf}{data.frame}\PY{p}{(}\PY{n}{resid} \PY{o}{=} \PY{n+nf}{residuals}\PY{p}{(}\PY{n}{filter\PYZus{}mnl}\PY{p}{,} \PY{n}{sd} \PY{o}{=} \PY{k+kc}{FALSE}\PY{p}{)}\PY{p}{,}\PY{n}{zero}\PY{o}{=}\PY{n+nf}{rep}\PY{p}{(}\PY{l+m}{0}\PY{p}{,}\PY{n+nb+bp}{T}\PY{p}{)}\PY{p}{,}\PY{n}{tempo}\PY{o}{\PYZlt{}\PYZhy{}}\PY{n+nf}{seq}\PY{p}{(}\PY{l+m}{1}\PY{o}{:}\PY{n+nb+bp}{T}\PY{p}{)}\PY{p}{)}
\end{Verbatim}
\end{tcolorbox}

    \begin{tcolorbox}[breakable, size=fbox, boxrule=1pt, pad at break*=1mm,colback=cellbackground, colframe=cellborder]
\prompt{In}{incolor}{271}{\boxspacing}
\begin{Verbatim}[commandchars=\\\{\}]
\PY{n+nf}{ggplot}\PY{p}{(}\PY{p}{)} \PY{o}{+}
\PY{n+nf}{geom\PYZus{}line}\PY{p}{(}\PY{n}{data} \PY{o}{=} \PY{n}{df\PYZus{}residual}\PY{p}{,} \PY{n+nf}{aes}\PY{p}{(}\PY{n}{x} \PY{o}{=} \PY{n}{tempo}\PY{p}{,} \PY{n}{y} \PY{o}{=} \PY{n}{resid}\PY{p}{)}\PY{p}{,} \PY{n}{color} \PY{o}{=} \PY{l+s}{\PYZdq{}}\PY{l+s}{black\PYZdq{}}\PY{p}{,} \PY{n}{size} \PY{o}{=} \PY{l+m}{1}\PY{p}{)} \PY{o}{+}
\PY{n+nf}{geom\PYZus{}line}\PY{p}{(}\PY{n}{data}\PY{o}{=}\PY{n}{df\PYZus{}residual}\PY{p}{,} \PY{n+nf}{aes}\PY{p}{(}\PY{n}{x}\PY{o}{=}\PY{n}{tempo}\PY{p}{,}\PY{n}{y}\PY{o}{=}\PY{n}{zero}\PY{p}{)}\PY{p}{)}\PY{o}{+}
\PY{n+nf}{geom\PYZus{}point}\PY{p}{(}\PY{n}{data}\PY{o}{=}\PY{n}{df\PYZus{}residual}\PY{p}{,}\PY{n+nf}{aes}\PY{p}{(}\PY{n}{x} \PY{o}{=} \PY{n}{tempo} \PY{p}{,} \PY{n}{y} \PY{o}{=} \PY{n}{resid}\PY{p}{)}\PY{p}{,} \PY{n}{color} \PY{o}{=} \PY{l+s}{\PYZdq{}}\PY{l+s}{black\PYZdq{}}\PY{p}{,} \PY{n}{size} \PY{o}{=} \PY{l+m}{6} \PY{p}{,}\PY{n}{alpha}\PY{o}{=}\PY{l+m}{0.3}\PY{p}{)} \PY{o}{+}

\PY{n+nf}{labs}\PY{p}{(}\PY{n}{title} \PY{o}{=} \PY{l+s}{\PYZdq{}}\PY{l+s}{Modelo de Nível Local: Erro de previsão\PYZdq{}}\PY{p}{,}
   \PY{n}{x} \PY{o}{=} \PY{l+s}{\PYZdq{}}\PY{l+s}{Tempo\PYZdq{}}\PY{p}{,} \PY{n}{y} \PY{o}{=} \PY{l+s}{\PYZdq{}}\PY{l+s}{resid\PYZdq{}}\PY{p}{)} \PY{o}{+}
\PY{n+nf}{theme\PYZus{}minimal}\PY{p}{(}\PY{p}{)}\PY{o}{+}
\PY{n+nf}{theme}\PY{p}{(}\PY{n}{panel.grid.major} \PY{o}{=} \PY{n+nf}{element\PYZus{}line}\PY{p}{(}\PY{n}{color} \PY{o}{=} \PY{l+s}{\PYZdq{}}\PY{l+s}{gray\PYZdq{}}\PY{p}{,} \PY{n}{linewidth} \PY{o}{=} \PY{l+m}{1}\PY{p}{)}\PY{p}{,}
\PY{n}{panel.grid.minor} \PY{o}{=} \PY{n+nf}{element\PYZus{}line}\PY{p}{(}\PY{n}{color} \PY{o}{=} \PY{l+s}{\PYZdq{}}\PY{l+s}{lightgray\PYZdq{}}\PY{p}{,} \PY{n}{linewidth} \PY{o}{=} \PY{l+m}{0.5}\PY{p}{)}\PY{p}{,}
\PY{n}{axis.line} \PY{o}{=} \PY{n+nf}{element\PYZus{}line}\PY{p}{(}\PY{n}{color} \PY{o}{=} \PY{l+s}{\PYZdq{}}\PY{l+s}{black\PYZdq{}}\PY{p}{,} \PY{n}{linewidth} \PY{o}{=} \PY{l+m}{1}\PY{p}{)}\PY{p}{,}  \PY{c+c1}{\PYZsh{} Make axis lines thicker and black}
\PY{n}{axis.ticks} \PY{o}{=} \PY{n+nf}{element\PYZus{}line}\PY{p}{(}\PY{n}{color} \PY{o}{=} \PY{l+s}{\PYZdq{}}\PY{l+s}{black\PYZdq{}}\PY{p}{,} \PY{n}{linewidth} \PY{o}{=} \PY{l+m}{0.8}\PY{p}{)}\PY{p}{,}  \PY{c+c1}{\PYZsh{} Make ticks more visible}
\PY{n}{axis.text} \PY{o}{=} \PY{n+nf}{element\PYZus{}text}\PY{p}{(}\PY{n}{size} \PY{o}{=} \PY{l+m}{12}\PY{p}{,} \PY{n}{color} \PY{o}{=} \PY{l+s}{\PYZdq{}}\PY{l+s}{black\PYZdq{}}\PY{p}{)}\PY{p}{,}  \PY{c+c1}{\PYZsh{} Adjust axis text size and color}
\PY{n}{axis.title} \PY{o}{=} \PY{n+nf}{element\PYZus{}text}\PY{p}{(}\PY{n}{size} \PY{o}{=} \PY{l+m}{14}\PY{p}{,} \PY{n}{color} \PY{o}{=} \PY{l+s}{\PYZdq{}}\PY{l+s}{black\PYZdq{}}\PY{p}{)}\PY{p}{)}
\end{Verbatim}
\end{tcolorbox}

    \begin{center}
    \adjustimage{max size={0.9\linewidth}{0.9\paperheight}}{T2_FK_files/T2_FK_101_0.png}
    \end{center}
    { \hspace*{\fill} \\}
    
    testando se os resíduos são normais:

    \begin{tcolorbox}[breakable, size=fbox, boxrule=1pt, pad at break*=1mm,colback=cellbackground, colframe=cellborder]
\prompt{In}{incolor}{274}{\boxspacing}
\begin{Verbatim}[commandchars=\\\{\}]
\PY{n+nf}{shapiro.test}\PY{p}{(}\PY{n}{df\PYZus{}residual}\PY{o}{\PYZdl{}}\PY{n}{resid}\PY{p}{)}
\PY{n+nf}{cat}\PY{p}{(}\PY{l+s}{\PYZdq{}}\PY{l+s}{media\PYZus{}residuos:\PYZdq{}}\PY{p}{,}\PY{n+nf}{mean}\PY{p}{(}\PY{n}{df\PYZus{}residual}\PY{o}{\PYZdl{}}\PY{n}{resid}\PY{p}{)}\PY{p}{,}\PY{l+s}{\PYZsq{}}\PY{l+s}{\PYZbs{}n\PYZsq{}}\PY{p}{)}
\PY{n+nf}{cat}\PY{p}{(}\PY{l+s}{\PYZdq{}}\PY{l+s}{var\PYZus{}residuos:\PYZdq{}}\PY{p}{,}\PY{n+nf}{var}\PY{p}{(}\PY{n}{df\PYZus{}residual}\PY{o}{\PYZdl{}}\PY{n}{resid}\PY{p}{)}\PY{p}{,}\PY{l+s}{\PYZsq{}}\PY{l+s}{\PYZbs{}n\PYZsq{}}\PY{p}{)}
\end{Verbatim}
\end{tcolorbox}

    
    \begin{Verbatim}[commandchars=\\\{\}]

	Shapiro-Wilk normality test

data:  df\_residual\$resid
W = 0.98873, p-value = 0.5638

    \end{Verbatim}

    
    \begin{Verbatim}[commandchars=\\\{\}]
media\_residuos: -0.03384876
var\_residuos: 0.9988415
    \end{Verbatim}

    \section{Implementando a MLE}\label{implementando-a-mle}

    o objetivo dessa secção é comparar a estimação manual e a minha
implementação de estimação

    Gerando a série e fazendo a estimação usando pacote

    \begin{tcolorbox}[breakable, size=fbox, boxrule=1pt, pad at break*=1mm,colback=cellbackground, colframe=cellborder]
\prompt{In}{incolor}{275}{\boxspacing}
\begin{Verbatim}[commandchars=\\\{\}]
\PY{n+nb+bp}{T}\PY{o}{=}\PY{l+m}{1000}
\PY{n}{y\PYZus{}mnl} \PY{o}{\PYZlt{}\PYZhy{}} \PY{n+nf}{simul\PYZus{}y\PYZus{}mnl}\PY{p}{(}\PY{n+nb+bp}{T}\PY{o}{=}\PY{n+nb+bp}{T}\PY{p}{,}\PY{l+m}{1}\PY{p}{,}\PY{l+m}{0.5}\PY{p}{,}\PY{l+m}{1}\PY{p}{)}
\PY{n}{fit} \PY{o}{\PYZlt{}\PYZhy{}} \PY{n+nf}{dlmMLE}\PY{p}{(}\PY{n}{y\PYZus{}mnl}\PY{p}{,} \PY{n}{parm} \PY{o}{=} \PY{n+nf}{c}\PY{p}{(}\PY{l+m}{100}\PY{p}{,} \PY{l+m}{2}\PY{p}{)}\PY{p}{,} \PY{n}{build\PYZus{}mnl}\PY{p}{,} \PY{n}{lower} \PY{o}{=} \PY{n+nf}{rep}\PY{p}{(}\PY{l+m}{1e\PYZhy{}4}\PY{p}{,} \PY{l+m}{2}\PY{p}{)}\PY{p}{)}
\end{Verbatim}
\end{tcolorbox}

    Estimação manual

    definindo o espaço paramétrico:

    \begin{tcolorbox}[breakable, size=fbox, boxrule=1pt, pad at break*=1mm,colback=cellbackground, colframe=cellborder]
\prompt{In}{incolor}{276}{\boxspacing}
\begin{Verbatim}[commandchars=\\\{\}]
\PY{n}{sigma\PYZus{}epsilon2\PYZus{}vals} \PY{o}{\PYZlt{}\PYZhy{}} \PY{n+nf}{seq}\PY{p}{(}\PY{l+m}{0.1}\PY{p}{,}\PY{l+m}{4}\PY{p}{,} \PY{n}{length.out} \PY{o}{=} \PY{l+m}{100}\PY{p}{)}
\PY{n}{sigma\PYZus{}eta2\PYZus{}vals} \PY{o}{\PYZlt{}\PYZhy{}} \PY{n+nf}{seq}\PY{p}{(}\PY{l+m}{0.1}\PY{p}{,} \PY{l+m}{4}\PY{p}{,} \PY{n}{length.out} \PY{o}{=} \PY{l+m}{100}\PY{p}{)}
\end{Verbatim}
\end{tcolorbox}

    função de verossimilhança:

    \begin{tcolorbox}[breakable, size=fbox, boxrule=1pt, pad at break*=1mm,colback=cellbackground, colframe=cellborder]
\prompt{In}{incolor}{280}{\boxspacing}
\begin{Verbatim}[commandchars=\\\{\}]
\PY{n}{logLik\PYZus{}mnl} \PY{o}{\PYZlt{}\PYZhy{}} \PY{n+nf}{function}\PY{p}{(}\PY{n}{theta}\PY{p}{)} \PY{p}{\PYZob{}}
  
  \PY{c+c1}{\PYZsh{}definindo parâmetros}
  \PY{n}{sigma\PYZus{}epsilon2} \PY{o}{\PYZlt{}\PYZhy{}} \PY{n}{theta}\PY{p}{[}\PY{l+m}{1}\PY{p}{]}
  \PY{n}{sigma\PYZus{}eta2} \PY{o}{\PYZlt{}\PYZhy{}} \PY{n}{theta}\PY{p}{[}\PY{l+m}{2}\PY{p}{]}
  
  \PY{c+c1}{\PYZsh{} fazendo a filtragem usando minha função do trabalho 1}
  \PY{n}{fk\PYZus{}results} \PY{o}{\PYZlt{}\PYZhy{}} \PY{n+nf}{mnl\PYZus{}fk}\PY{p}{(}\PY{n+nb+bp}{T} \PY{o}{=} \PY{n+nf}{length}\PY{p}{(}\PY{n}{y\PYZus{}mnl}\PY{p}{)}\PY{p}{,} \PY{n}{sigma\PYZus{}epsilon2}\PY{p}{,} \PY{n}{sigma\PYZus{}eta2}\PY{p}{,} \PY{n}{a0} \PY{o}{=} \PY{l+m}{0}\PY{p}{,} \PY{n}{p0} \PY{o}{=} \PY{l+m}{100}\PY{p}{,} \PY{n}{y\PYZus{}mnl}\PY{p}{)}
  
  \PY{c+c1}{\PYZsh{} extraindo (v\PYZus{}t) e a variância (F\PYZus{}t)}
  \PY{n}{v\PYZus{}t} \PY{o}{\PYZlt{}\PYZhy{}} \PY{n}{fk\PYZus{}results}\PY{o}{\PYZdl{}}\PY{n}{v\PYZus{}t}
  \PY{n}{F\PYZus{}t} \PY{o}{\PYZlt{}\PYZhy{}} \PY{n}{fk\PYZus{}results}\PY{o}{\PYZdl{}}\PY{n+nb+bp}{F}
  
  \PY{c+c1}{\PYZsh{} Computando a verossimilhança}
  \PY{n}{n} \PY{o}{\PYZlt{}\PYZhy{}} \PY{n+nf}{length}\PY{p}{(}\PY{n}{y\PYZus{}mnl}\PY{p}{)}
  \PY{n}{logLik} \PY{o}{\PYZlt{}\PYZhy{}} \PY{o}{\PYZhy{}} \PY{p}{(}\PY{n}{n} \PY{o}{/} \PY{l+m}{2}\PY{p}{)} \PY{o}{*} \PY{n+nf}{log}\PY{p}{(}\PY{l+m}{2} \PY{o}{*} \PY{k+kc}{pi}\PY{p}{)} \PY{o}{\PYZhy{}} \PY{l+m}{0.5} \PY{o}{*} \PY{n+nf}{sum}\PY{p}{(}\PY{n+nf}{log}\PY{p}{(}\PY{n+nf}{abs}\PY{p}{(}\PY{n}{F\PYZus{}t}\PY{p}{)}\PY{p}{)}\PY{p}{)} \PY{o}{\PYZhy{}} \PY{l+m}{0.5} \PY{o}{*} \PY{n+nf}{sum}\PY{p}{(}\PY{p}{(}\PY{n}{v\PYZus{}t}\PY{o}{\PYZca{}}\PY{l+m}{2}\PY{p}{)} \PY{o}{/} \PY{n}{F\PYZus{}t}\PY{p}{)}
  
  \PY{n+nf}{return}\PY{p}{(}\PY{n}{logLik}\PY{p}{)}
\PY{p}{\PYZcb{}}
\end{Verbatim}
\end{tcolorbox}

    Calculando a verossimilhança para todas as combinações dos
hyperparâmetros:

    \begin{tcolorbox}[breakable, size=fbox, boxrule=1pt, pad at break*=1mm,colback=cellbackground, colframe=cellborder]
\prompt{In}{incolor}{277}{\boxspacing}
\begin{Verbatim}[commandchars=\\\{\}]
\PY{n}{likelihood\PYZus{}matrix} \PY{o}{\PYZlt{}\PYZhy{}} \PY{n+nf}{outer}\PY{p}{(}\PY{n}{sigma\PYZus{}epsilon2\PYZus{}vals}\PY{p}{,} \PY{n}{sigma\PYZus{}eta2\PYZus{}vals}\PY{p}{,} 
                           \PY{n+nf}{Vectorize}\PY{p}{(}\PY{n+nf}{function}\PY{p}{(}\PY{n}{se}\PY{p}{,} \PY{n}{sw}\PY{p}{)} \PY{n+nf}{logLik\PYZus{}mnl}\PY{p}{(}\PY{n+nf}{c}\PY{p}{(}\PY{n}{se}\PY{p}{,} \PY{n}{sw}\PY{p}{)}\PY{p}{)}\PY{p}{)}\PY{p}{)}
\end{Verbatim}
\end{tcolorbox}

    transformando em um df para usar o ggplot2

    \begin{tcolorbox}[breakable, size=fbox, boxrule=1pt, pad at break*=1mm,colback=cellbackground, colframe=cellborder]
\prompt{In}{incolor}{282}{\boxspacing}
\begin{Verbatim}[commandchars=\\\{\}]
\PY{n}{likelihood\PYZus{}df} \PY{o}{\PYZlt{}\PYZhy{}} \PY{n+nf}{expand.grid}\PY{p}{(}\PY{n}{sigma\PYZus{}epsilon2} \PY{o}{=} \PY{n}{sigma\PYZus{}epsilon2\PYZus{}vals}\PY{p}{,} 
                             \PY{n}{sigma\PYZus{}eta2} \PY{o}{=} \PY{n}{sigma\PYZus{}eta2\PYZus{}vals}\PY{p}{)}
\end{Verbatim}
\end{tcolorbox}

    adicionando a verossimilhança no df

    \begin{tcolorbox}[breakable, size=fbox, boxrule=1pt, pad at break*=1mm,colback=cellbackground, colframe=cellborder]
\prompt{In}{incolor}{284}{\boxspacing}
\begin{Verbatim}[commandchars=\\\{\}]
\PY{n}{likelihood\PYZus{}df}\PY{o}{\PYZdl{}}\PY{n}{logLik} \PY{o}{\PYZlt{}\PYZhy{}} \PY{n+nf}{as.vector}\PY{p}{(}\PY{n}{likelihood\PYZus{}matrix}\PY{p}{)}
\end{Verbatim}
\end{tcolorbox}

    comparando os resultado

    pacote:

    \begin{tcolorbox}[breakable, size=fbox, boxrule=1pt, pad at break*=1mm,colback=cellbackground, colframe=cellborder]
\prompt{In}{incolor}{286}{\boxspacing}
\begin{Verbatim}[commandchars=\\\{\}]
\PY{n+nf}{cat}\PY{p}{(}\PY{l+s}{\PYZdq{}}\PY{l+s}{Estimated sigma\PYZca{}2\PYZus{}epsilon:\PYZdq{}}\PY{p}{,} \PY{n}{fit}\PY{o}{\PYZdl{}}\PY{n}{par}\PY{p}{[}\PY{l+m}{1}\PY{p}{]}\PY{p}{,} \PY{l+s}{\PYZdq{}}\PY{l+s}{\PYZbs{}n\PYZdq{}}\PY{p}{)}
\PY{n+nf}{cat}\PY{p}{(}\PY{l+s}{\PYZdq{}}\PY{l+s}{Estimated sigma\PYZca{}2\PYZus{}eta:\PYZdq{}}\PY{p}{,} \PY{n}{fit}\PY{o}{\PYZdl{}}\PY{n}{par}\PY{p}{[}\PY{l+m}{2}\PY{p}{]}\PY{p}{,} \PY{l+s}{\PYZdq{}}\PY{l+s}{\PYZbs{}n\PYZdq{}}\PY{p}{)}
\end{Verbatim}
\end{tcolorbox}

    \begin{Verbatim}[commandchars=\\\{\}]
Estimated sigma\^{}2\_epsilon: 1.074177
Estimated sigma\^{}2\_eta: 0.5647697
    \end{Verbatim}

    manual:

    \begin{tcolorbox}[breakable, size=fbox, boxrule=1pt, pad at break*=1mm,colback=cellbackground, colframe=cellborder]
\prompt{In}{incolor}{289}{\boxspacing}
\begin{Verbatim}[commandchars=\\\{\}]
\PY{n}{mle\PYZus{}estimates} \PY{o}{\PYZlt{}\PYZhy{}} \PY{n}{likelihood\PYZus{}df} \PY{o}{\PYZpc{}\PYZgt{}\PYZpc{}}
  \PY{n+nf}{filter}\PY{p}{(}\PY{n}{logLik} \PY{o}{==} \PY{n+nf}{max}\PY{p}{(}\PY{n}{logLik}\PY{p}{)}\PY{p}{)} \PY{o}{\PYZpc{}\PYZgt{}\PYZpc{}}
  \PY{n+nf}{select}\PY{p}{(}\PY{n}{sigma\PYZus{}epsilon2}\PY{p}{,} \PY{n}{sigma\PYZus{}eta2}\PY{p}{,} \PY{n}{logLik}\PY{p}{)}

\PY{n+nf}{names}\PY{p}{(}\PY{n}{mle\PYZus{}estimates}\PY{p}{)}\PY{o}{\PYZlt{}\PYZhy{}}\PY{n+nf}{c}\PY{p}{(}\PY{l+s}{\PYZdq{}}\PY{l+s}{sigma2\PYZus{}epsilon\PYZdq{}}\PY{p}{,}\PY{l+s}{\PYZsq{}}\PY{l+s}{sigma2\PYZus{}eta\PYZsq{}}\PY{p}{,}\PY{l+s}{\PYZsq{}}\PY{l+s}{Max LogLikelihood\PYZsq{}}\PY{p}{)}

\PY{n+nf}{cat}\PY{p}{(}\PY{l+s}{\PYZdq{}}\PY{l+s}{Estimated sigma\PYZca{}2\PYZus{}epsilon:\PYZdq{}}\PY{p}{,} \PY{n+nf}{unlist}\PY{p}{(}\PY{n}{mle\PYZus{}estimates}\PY{p}{[}\PY{l+m}{1}\PY{p}{]}\PY{p}{)}\PY{p}{,} \PY{l+s}{\PYZdq{}}\PY{l+s}{\PYZbs{}n\PYZdq{}}\PY{p}{)}
\PY{n+nf}{cat}\PY{p}{(}\PY{l+s}{\PYZdq{}}\PY{l+s}{Estimated sigma\PYZca{}2\PYZus{}eta:\PYZdq{}}\PY{p}{,} \PY{n+nf}{unlist}\PY{p}{(}\PY{n}{mle\PYZus{}estimates}\PY{p}{[}\PY{l+m}{2}\PY{p}{]}\PY{p}{)}\PY{p}{,} \PY{l+s}{\PYZdq{}}\PY{l+s}{\PYZbs{}n\PYZdq{}}\PY{p}{)}
\end{Verbatim}
\end{tcolorbox}

    \begin{Verbatim}[commandchars=\\\{\}]
Estimated sigma\^{}2\_epsilon: 1.084848
Estimated sigma\^{}2\_eta: 0.5727273
    \end{Verbatim}

    \begin{tcolorbox}[breakable, size=fbox, boxrule=1pt, pad at break*=1mm,colback=cellbackground, colframe=cellborder]
\prompt{In}{incolor}{290}{\boxspacing}
\begin{Verbatim}[commandchars=\\\{\}]
\PY{n}{fig} \PY{o}{\PYZlt{}\PYZhy{}} \PY{n+nf}{plot\PYZus{}ly}\PY{p}{(}
  \PY{n}{x} \PY{o}{=} \PY{n+nf}{unique}\PY{p}{(}\PY{n}{likelihood\PYZus{}df}\PY{o}{\PYZdl{}}\PY{n}{sigma\PYZus{}epsilon2}\PY{p}{)}\PY{p}{,}  \PY{c+c1}{\PYZsh{} X\PYZhy{}axis (sigma\PYZus{}epsilon2 values)}
  \PY{n}{y} \PY{o}{=} \PY{n+nf}{unique}\PY{p}{(}\PY{n}{likelihood\PYZus{}df}\PY{o}{\PYZdl{}}\PY{n}{sigma\PYZus{}eta2}\PY{p}{)}\PY{p}{,}      \PY{c+c1}{\PYZsh{} Y\PYZhy{}axis (sigma\PYZus{}eta2 values)}
  \PY{n}{z} \PY{o}{=} \PY{n+nf}{matrix}\PY{p}{(}\PY{n}{likelihood\PYZus{}df}\PY{o}{\PYZdl{}}\PY{n}{logLik}\PY{p}{,} 
             \PY{n}{nrow} \PY{o}{=} \PY{n+nf}{length}\PY{p}{(}\PY{n+nf}{unique}\PY{p}{(}\PY{n}{likelihood\PYZus{}df}\PY{o}{\PYZdl{}}\PY{n}{sigma\PYZus{}epsilon2}\PY{p}{)}\PY{p}{)}\PY{p}{,} 
             \PY{n}{ncol} \PY{o}{=} \PY{n+nf}{length}\PY{p}{(}\PY{n+nf}{unique}\PY{p}{(}\PY{n}{likelihood\PYZus{}df}\PY{o}{\PYZdl{}}\PY{n}{sigma\PYZus{}eta2}\PY{p}{)}\PY{p}{)}\PY{p}{)}\PY{p}{,}  \PY{c+c1}{\PYZsh{} Reshape log\PYZhy{}likelihood data}
  \PY{n}{type} \PY{o}{=} \PY{l+s}{\PYZdq{}}\PY{l+s}{surface\PYZdq{}}\PY{p}{,}
  \PY{n}{colorscale} \PY{o}{=} \PY{l+s}{\PYZdq{}}\PY{l+s}{Inferno\PYZdq{}}  \PY{c+c1}{\PYZsh{} Makes high values vivid red}
\PY{p}{)}

\PY{n}{fig} \PY{o}{\PYZlt{}\PYZhy{}} \PY{n}{fig} \PY{o}{\PYZpc{}\PYZgt{}\PYZpc{}}
  \PY{n+nf}{layout}\PY{p}{(}
    \PY{n}{title} \PY{o}{=} \PY{l+s}{\PYZdq{}}\PY{l+s}{Log\PYZhy{}Likelihood Surface for MNL Model\PYZdq{}}\PY{p}{,}
    \PY{n}{scene} \PY{o}{=} \PY{n+nf}{list}\PY{p}{(}
      \PY{n}{xaxis} \PY{o}{=} \PY{n+nf}{list}\PY{p}{(}\PY{n}{title} \PY{o}{=} \PY{n+nf}{expression}\PY{p}{(}\PY{n}{sigma}\PY{p}{[}\PY{n}{epsilon}\PY{p}{]}\PY{o}{\PYZca{}}\PY{l+m}{2}\PY{p}{)}\PY{p}{)}\PY{p}{,}
      \PY{n}{yaxis} \PY{o}{=} \PY{n+nf}{list}\PY{p}{(}\PY{n}{title} \PY{o}{=} \PY{n+nf}{expression}\PY{p}{(}\PY{n}{sigma}\PY{p}{[}\PY{n}{eta}\PY{p}{]}\PY{o}{\PYZca{}}\PY{l+m}{2}\PY{p}{)}\PY{p}{)}\PY{p}{,}
      \PY{n}{zaxis} \PY{o}{=} \PY{n+nf}{list}\PY{p}{(}\PY{n}{title} \PY{o}{=} \PY{l+s}{\PYZdq{}}\PY{l+s}{Log\PYZhy{}Likelihood\PYZdq{}}\PY{p}{)}
    \PY{p}{)}
  \PY{p}{)}


\PY{n}{fig}
\end{Verbatim}
\end{tcolorbox}

    
    \begin{Verbatim}[commandchars=\\\{\}]
HTML widgets cannot be represented in plain text (need html)
    \end{Verbatim}

    
    Heatmap:

    \begin{tcolorbox}[breakable, size=fbox, boxrule=1pt, pad at break*=1mm,colback=cellbackground, colframe=cellborder]
\prompt{In}{incolor}{293}{\boxspacing}
\begin{Verbatim}[commandchars=\\\{\}]
\PY{n+nf}{ggplot}\PY{p}{(}\PY{n}{likelihood\PYZus{}df}\PY{p}{,} \PY{n+nf}{aes}\PY{p}{(}\PY{n}{x} \PY{o}{=} \PY{n}{sigma\PYZus{}epsilon2}\PY{p}{,} \PY{n}{y} \PY{o}{=} \PY{n}{sigma\PYZus{}eta2}\PY{p}{,} \PY{n}{fill} \PY{o}{=} \PY{n}{logLik}\PY{p}{)}\PY{p}{)} \PY{o}{+}
  \PY{n+nf}{geom\PYZus{}tile}\PY{p}{(}\PY{p}{)} \PY{o}{+}
  \PY{n+nf}{scale\PYZus{}fill\PYZus{}viridis\PYZus{}c}\PY{p}{(}\PY{p}{)}\PY{o}{+}
  \PY{n+nf}{labs}\PY{p}{(}\PY{n}{title} \PY{o}{=} \PY{l+s}{\PYZdq{}}\PY{l+s}{Log\PYZhy{}Likelihood Surface for MNL Model\PYZdq{}}\PY{p}{,}
       \PY{n}{x} \PY{o}{=} \PY{l+s}{\PYZsq{}}\PY{l+s}{sigma\PYZca{}2 epsilon\PYZsq{}}\PY{p}{,}
       \PY{n}{y} \PY{o}{=} \PY{l+s}{\PYZsq{}}\PY{l+s}{sigma\PYZca{}2 eta\PYZsq{}}\PY{p}{,}
       \PY{n}{fill} \PY{o}{=} \PY{l+s}{\PYZdq{}}\PY{l+s}{Log\PYZhy{}Likelihood\PYZdq{}}\PY{p}{)} \PY{o}{+}
      \PY{n+nf}{theme\PYZus{}minimal}\PY{p}{(}\PY{p}{)}\PY{o}{+}
    \PY{n+nf}{theme}\PY{p}{(}\PY{n}{panel.grid.major} \PY{o}{=} \PY{n+nf}{element\PYZus{}line}\PY{p}{(}\PY{n}{color} \PY{o}{=} \PY{l+s}{\PYZdq{}}\PY{l+s}{gray\PYZdq{}}\PY{p}{,} \PY{n}{linewidth} \PY{o}{=} \PY{l+m}{1}\PY{p}{)}\PY{p}{,}
    \PY{n}{panel.grid.minor} \PY{o}{=} \PY{n+nf}{element\PYZus{}line}\PY{p}{(}\PY{n}{color} \PY{o}{=} \PY{l+s}{\PYZdq{}}\PY{l+s}{lightgray\PYZdq{}}\PY{p}{,} \PY{n}{linewidth} \PY{o}{=} \PY{l+m}{0.5}\PY{p}{)}\PY{p}{,}
    \PY{n}{axis.line} \PY{o}{=} \PY{n+nf}{element\PYZus{}line}\PY{p}{(}\PY{n}{color} \PY{o}{=} \PY{l+s}{\PYZdq{}}\PY{l+s}{black\PYZdq{}}\PY{p}{,} \PY{n}{linewidth} \PY{o}{=} \PY{l+m}{1}\PY{p}{)}\PY{p}{,}  \PY{c+c1}{\PYZsh{} Make axis lines thicker and black}
    \PY{n}{axis.ticks} \PY{o}{=} \PY{n+nf}{element\PYZus{}line}\PY{p}{(}\PY{n}{color} \PY{o}{=} \PY{l+s}{\PYZdq{}}\PY{l+s}{black\PYZdq{}}\PY{p}{,} \PY{n}{linewidth} \PY{o}{=} \PY{l+m}{0.8}\PY{p}{)}\PY{p}{,}  \PY{c+c1}{\PYZsh{} Make ticks more visible}
    \PY{n}{axis.text} \PY{o}{=} \PY{n+nf}{element\PYZus{}text}\PY{p}{(}\PY{n}{size} \PY{o}{=} \PY{l+m}{12}\PY{p}{,} \PY{n}{color} \PY{o}{=} \PY{l+s}{\PYZdq{}}\PY{l+s}{black\PYZdq{}}\PY{p}{)}\PY{p}{,}  \PY{c+c1}{\PYZsh{} Adjust axis text size and color}
    \PY{n}{axis.title} \PY{o}{=} \PY{n+nf}{element\PYZus{}text}\PY{p}{(}\PY{n}{size} \PY{o}{=} \PY{l+m}{14}\PY{p}{,} \PY{n}{color} \PY{o}{=} \PY{l+s}{\PYZdq{}}\PY{l+s}{black\PYZdq{}}\PY{p}{)}\PY{p}{)}
\end{Verbatim}
\end{tcolorbox}

    \begin{center}
    \adjustimage{max size={0.9\linewidth}{0.9\paperheight}}{T2_FK_files/T2_FK_126_0.png}
    \end{center}
    { \hspace*{\fill} \\}
    
    \begin{tcolorbox}[breakable, size=fbox, boxrule=1pt, pad at break*=1mm,colback=cellbackground, colframe=cellborder]
\prompt{In}{incolor}{296}{\boxspacing}
\begin{Verbatim}[commandchars=\\\{\}]
\PY{n}{jupyter} \PY{n}{nbconvert} \PY{o}{\PYZhy{}}\PY{o}{\PYZhy{}}\PY{n}{to} \PY{n}{pdf} \PY{n}{T2\PYZus{}Fk.pynb}
\end{Verbatim}
\end{tcolorbox}

    \begin{Verbatim}[commandchars=\\\{\}, frame=single, framerule=2mm, rulecolor=\color{outerrorbackground}]
Error in parse(text = input): <text>:1:9: unexpected symbol
1: jupyter nbconvert
            \^{}
Traceback:

    \end{Verbatim}

    \begin{tcolorbox}[breakable, size=fbox, boxrule=1pt, pad at break*=1mm,colback=cellbackground, colframe=cellborder]
\prompt{In}{incolor}{ }{\boxspacing}
\begin{Verbatim}[commandchars=\\\{\}]

\end{Verbatim}
\end{tcolorbox}


    % Add a bibliography block to the postdoc
    
    
    
\end{document}
